\subsection{Operations in the source code}
\subsubsection{Object names}

\begin{verbatim}
char name_component_properties[80] = "TRACER_PROPERTIES"; \\
char name_component_properties_component[80] =
                                  "TRACER_PROPERTIES999";
\end{verbatim}

\subsubsection{Construction}

The component properties can be created from reading a RFD file.
The read function is $int MATReadComponentProperties(char *data,
int found, FILE * f);$. Destruction by MAT list destruction $void
DestroyMaterialPropertiesList(void)$.

\subsubsection{Data Access}

Access to CP objects is possible either from MAT list or from MAT
table.

\begin{itemize}
  \item from MAT list by name

  Data object

  $cp = GetTracerPropertiesObject(name_component_properties_component,
  cp);$
\\
  Object properties by $set$ and $get$ functions like:

  $set\_tp\_solubility(cp, d)$  \ \ /*save the parameter value \\
  $get\_tp\_solubility(cp)$   \ \ \ \   /*read the parameter value

  \item from MAT table by group, phase, component numbers via MPT:

  $double\ GetTracerDiffusionCoefficient(long index, long phase, long component);$
\end{itemize}



\begin{verbatim}
#Reference
[1]Habber, A. (2001): Direckte und Inverse Modellierung
reaktiver Transportprozesse in kl\"{u}ftig-por\"{o}sen Medien,
Uni-Dissertation, Uni-Hannover

[2]Kohlmeier, M. et al. (2001): Modling of Coupled
Geohydraulic/Geomechanical/Geochemical Processes in Salt Rock,
Technical Report, Universit\"{a}t Hannover

\end{verbatim}
