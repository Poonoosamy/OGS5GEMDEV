\section{Numerics}

\begin{tabular*}{8.35cm}{|p{2.5cm}|p{5cm}|} \hline
Object acronym & NUM \\
C++ class      & Numerics \\
Source files   & rf\_num\_new.h/cpp \\
\hline
File extension & *.num \\
Object keyword & \#NUMERICS \\
\hline
\end{tabular*}

\Developer{
%----------------------------------------------------------
\subsection{Theory}
}
This object is used to handle parameters of linear/non-linear solver, numerical integration
 and numerical relaxation.
%-------------------------------------------------------------------------------
\subsection{\texttt{\bf\#NUMERICS}}

\begin{verbatim}
#NUMERICS
 $PCS_TYPE
  LIQUID_FLOW
 $LINEAR_SOLVER
; method error_tolerance max_iterations theta precond storage
  2      0 1.e-008       1000           1.0   100       4
$COUPLING_ITERATIONS
 CPL_NAME1 3 1.0e-3
#NUMERICS
 $PCS_TYPE
  HEAT_TRANSPORT
 $LINEAR_SOLVER
; method error_tolerance max_iterations theta precond storage
  2      0 1.e-012       1000           0.5   100       4
CPL_NAME2 3 1.0e-3
#NUMERICS
 $PCS_TYPE
  DEFORMATION
 $NON_LINEAR_SOLVER
; method error_tolerance max_iterations relaxation
  NEWTON 1e-2 1e-10      100            0.0
 $LINEAR_SOLVER
; method error_tolerance max_iterations  theta   precond  storage
  2      0      1.e-011       5000          1.0   100       4
#STOP
\end{verbatim}
where \texttt{PCS\_TYPE} refers to a process. Data followed after keyword \texttt{LINEAR\_SOLVER} define
   parameters to control the convergence of a linear solver as given in Table \ref{tab_num1}.
{\ttfamily
  \small
\begin{table}[!htb]
\centering
\begin{tabular}{|l|c|l|}
  \hline
  Name &Number & Meaning  \\
  \hline
  \hline % 1st row
  Method &  1  &SpGAUSS, direct solver\\
  &  2  &SpBICGSTAB\\
  &  3  &SpBICG\\
 &  4  &SpQMRCGSTAB\\
  &  5  &SpCG\\
  &  6  &SpCGNR\\
  &  7  &CGS\\
  &  8  &SpRichard\\
  &  9  &SpJOR\\
  &  10  &SpSOR\\
 \hline
  Error &  0  &Absolutely error, $\|\bm r\|<\epsilon$\\
 &  1  &$\|\bm r\|<\epsilon\|\bm b\|$\\
 &  2  &$\|\bm r_n\|<\epsilon\|\bm r_{n-1}\|$\\
 &  3  &if  $\|\bm r_{n}\|>1$, $\|\bm r_n\|<\epsilon\|\bm r_{n-1}\|$, else $\|\bm r\|<\epsilon$\\
 &  4  &$\|\bm r_n\|<\epsilon\|\bm x\|$\\
 &  5  &$\|\bm r_n\|<\epsilon\ max(|\bm x\|,\,|\bm b\|, \,|\bm r_{n-1}\|)$\\
 &  6  &other\\
\hline
  Theta &  0  &Relaxation parameter, $\theta\in[0,1]$.  \\
 \hline
  Preconditioner &  0  &No preconditioner  \\
   &  1  &Jacobi  \\
  &  100  &ILU  \\
 \hline
  Storage &  2  &Symmetry  \\
 Storage &  4  &Asymmetry  \\
 \hline
 \hline
\end{tabular}
\caption{Data by keyword \textbf{\$LINEAR\_SOLVER}}
\label{tab_num1}
\end{table}
}
Note: In current version, only SpBICGSTAB is well tested.

The coupling loop is controlled by keyword \mbox{\$COUPLING\_ITERATIONS} followed by acronym of the names of
the coupled processes, maximum number of iterations, and the tolerance.  For example, if a THM coupled problem
is being modeled, the \mbox{CPL\_NAME1} can be \mbox{THM} and the \mbox{CPL\_NAME2} can be \mbox{TH}. If keyword
\mbox{\$COUPLING\_ITERATIONS} is not given for a simulation, the default value, maximum number of iterations being 1, will be used.


Keyword \textbf{\$NON\_LINEAR\_SOLVER} leads the configuration of the basic nonlinear solver as given in
 Table \ref{tab_num2}.
{\ttfamily
  \small
\begin{table}[!htb]
\centering
\begin{tabular}{|l|c|l|}
  \hline
  Name &Number & Meaning  \\
  \hline
  \hline % 1st row
  Method &  PICARD &Picard iteration\\
  &  NEWTON  &Newton-Raphson methods\\
 \hline
  Error &  float number  &tolerance for global Newton-Raphson step\\
 \hline
  Tolerance &  float number  &tolerance for Picard or local Newton-Raphson step\\
\hline
 \hline
\end{tabular}
\caption{Data by keyword \textbf{\$NON\_LINEAR\_SOLVER}}
\label{tab_num2}
\end{table}
}
Note: In current version, Newton-Raphson method is only valid for deformation and overland flow analysis.

Keyword \mbox{\$COUPLING\_ITERATIONS} is always located at the first \mbox{\#NUMERICS} object if staggered scheme is
 used for a coupled processes.
\LastModified{WW - \today}
