\section{Boundary Conditions}

\begin{tabular*}{8.35cm}{|p{2.5cm}|p{5cm}|} \hline
Object acronym & BC \\
C++ class      & CBoundaryCondition \\
Source files   & rf\_bc\_new.h/cpp \\
\hline
File extension & *.bc \\
Object keyword & \#BOUNDARY\_CONDITION \\
\hline
\end{tabular*}

\Developer{
%----------------------------------------------------------
\subsection{Theory}

The solution of an initial-boundary-value-problem (e.g. partial
differential equations for flow and transport problems) requires
the specification of initial and boundary conditions. In this
section we describe the most common boundary conditions and their
physical meaning. Methods for implementation of boundary
conditions in discrete equations are discussed in the part of
numerical methods. Examples of common flow boundary conditions are
e.g. inlet, outlet, wall, prescribed pressure values.

Constant pressure boundary condition: The constant pressure
boundary condition is used in situations where exact details of
the flow distribution are unknown but the boundary values of
pressure are known.

No-slip boundary condition at a wall: This boundary condition is
most common encountered in confined flow problems. The no-slip
condition is the appropriate one for the velocity components at
solid walls. The normal velocity component can simply be set to
zero at this boundary. \index{condition - boundary - no-slip}

The following table gives an overview on common boundary condition
types and its mathematical representation.

\begin{table}[htb!]
\caption{Boundary conditions types}
\begin{center}
\begin{tabular}{|l|c|l|}
\hline
Type of BC & Mathematical Meaning & Physical Meaning \\
\hline \hline
Dirichlet type  & $\psi$               & prescribed value \\
                &                      & potential surface \\
\hline
Neumann type    & $\nabla\psi$         & prescribed flux \\
                &                      & stream surface \\
\hline
Cauchy type     & $\psi+A\nabla\psi$   & resistance between  \\
                &                      & potential and stream surface \\
\hline
\end{tabular}
\end{center}
\end{table}
\index{condition - boundary - Dirichlet} \index{condition -
boundary - Neumann } \index{condition - boundary - Cauchy}
\index{flux}

To describe conditions at boundaries we can use flux expressions
of conservation quantities.

\begin{table}[htb!]
\caption{Fluxes through surface boundaries}
\begin{center}
\begin{tabular}{|l|l|}
\hline
Quantity & Flux term \\
\hline \hline
Mass &   $\rho\bf v$ \\
\hline
Momentum & $\rho{\bf vv}-\sigma$ \\
\hline
Energy   & $\rho e\bf v - \lambda\nabla T$ \\
\hline
\end{tabular}
\end{center}
\end{table}

}

%-------------------------------------------------------------------------------
\subsection{\texttt{\bf\#BOUNDARY\_CONDITION}}
\label{sec:bc}
%\subsubsection{Keyword structure}

\begin{verbatim}
#BOUNDARY_CONDITION
 $PCS_TYPE // physical process
  LIQUID_FLOW       // H process (incompressible flow)
  UNCONFINED_FLOW   // H process (incompressible flow)
  GAS_FLOW          // H process (compressible flow)
  TWO_PHASE_FLOW    // H2 process (incompressible/compressible flow)
  COMPONENTAL_FLOW  // H2 process (incompressible/compressible flow)
  RIVER_FLOW        // H process (incompressible flow)
  RICHARDS_FLOW     // H process (incompressible flow)
  OVERLAND_FLOW     // H process (incompressible flow)
  HEAT_TRANSPORT    // T process (single/multi-phase flow)
  DEFORMATION       // M process (single/multi-phase flow)
  MASS_TRANSPORT    // C process (single/multi-phase flow)
  GROUNDWATER_FLOW  // H process (incompressible flow)
 $PRIMARY_VARIABLE
  PRESSURE1      // flow (phase)
  SATURATION2
  TEMPERATURE1   // heat transport
  DISPLACEMENT_X1 // deformation  (radial direction for axisymmetry)
  DISPLACEMENT_Y1
  DISPLACEMENT_Z1  //(axial direction for axisymmetry)
  CONCENTRATION1 // mass transport
  CONCENTRATIONx
 $GEO_TYPE // geometry
  POINT     name
  POLYLINE  name
  SURFACE   name
 $DIS_TYPE // value distribution
  CONSTANT  value
  LINEAR
  DATA_BASE file_name // EXCEL(CSV) file
 $TIM_TYPE // time dependencies
  CURVE number
 $FCT_TYPE // 4.2.14(OK)
  fct_name
\end{verbatim}


\begin{tabular*}{12.773cm}{|p{3.cm}|p{1.5cm}|p{7cm}|} \hline
Parameter          & Acronym & Meaning \\ \hline \hline
%
\texttt{PCS\_TYPE} & PCS &  Reference to a process \\
\texttt{GEO\_TYPE} & GEO &  Reference to a geometric object \\
\texttt{DIS\_TYPE} & DIS &  Distribution of source terms values \\
\texttt{TIM\_TYPE} & TIM &  Time dependencies of source terms \\
\texttt{FCT\_TYPE} & FCT &  Functional dependencies of source terms \\
\hline
\end{tabular*}

\subsubsection{\texttt{\$PCS\_TYPE}}

\begin{tabular*}{12.773cm}{|p{3.cm}|p{1.5cm}|p{7cm}|} \hline
Parameter          & Value & Meaning \\ \hline \hline
%
\texttt{PRESSUREx}        & phase & Source term for fluid phase x \\
\texttt{DISPLACEMENTx\_X} & phase & Load force for solid phase x \\
\texttt{DISPLACEMENTx\_Y} & phase & ... \\
\texttt{DISPLACEMENTx\_Z} & phase & ... \\
\texttt{TEMPERATUREx}     & phase & Source term for temperature \\
\texttt{CONCENTRATIONx}   & component & Source term for component mass \\
\hline
\end{tabular*}

\subsubsection{\texttt{\$GEO\_TYPE}}

\begin{tabular*}{12.773cm}{|p{3.cm}|p{8.9cm}|} \hline
Parameter          & Meaning \\ \hline \hline
%
\texttt{POINT}     & Name of point \\
\texttt{POLYLINE}  & Name of polyline \\
\texttt{SURFACE}   & Name of surface \\
\hline
\end{tabular*}

\subsubsection{\texttt{\$DIS\_TYPE}}

\begin{tabular*}{12.773cm}{|p{3.cm}|p{8.9cm}|} \hline
Parameter           & Meaning \\ \hline \hline
%
\texttt{CONSTANT}   & constant value \\
\texttt{LINEAR}     & ?WW \\
\texttt{DATA\_BASE} & distributed data from file \\
\hline
\end{tabular*}

\subsubsection{\texttt{\$TIM\_TYPE}}

\begin{tabular*}{12.773cm}{|p{3.cm}|p{8.9cm}|} \hline
Parameter           & Meaning \\ \hline \hline
%
\texttt{CURVE}      & TIM curve number \\
\hline
\end{tabular*}

\Examples{
%-------------------------------------------------------------------------------
\subsection{Examples}

%-------------------------------------------------------------------------------
\subsubsection{Boundary condition at point}

\begin{verbatim}
benchmark: h_line.bc
#BOUNDARY_CONDITION
 $PCS_TYPE
  LIQUID_FLOW
 $PRIMARY_VARIABLE
  PRESSURE1
 $GEO_TYPE
  POINT POINT1
 $DIS_TYPE
  CONSTANT 2.000000e+004
#STOP
\end{verbatim}

This boundary condition is defined for a process with primary variable
\texttt{PRESSURE1}. Geometrically the boundary condition is linked to point 1
named \texttt{POINT1}.

\begin{verbatim}
benchmark: h_line.gli
#POINTS
0 0.0 0.0 0.0  0.0
1 1.000000e+002  0.000000e+000  0.000000e+000
#STOP
\end{verbatim}

%-------------------------------------------------------------------------------
\subsubsection{Boundary condition along polyline}

\begin{verbatim}
benchmark: hm_tri.bc
#BOUNDARY_CONDITION
 $PCS_TYPE
  DEFORMATION
 $PRIMARY_VARIABLE
  DISPLACEMENT_Y1
 $GEO_TYPE
  POLYLINE BOTTOM
 $DIS_TYPE
  CONSTANT 0.0
#STOP
\end{verbatim}

This boundary condition is defined for a process with primary variable
\texttt{DISPLACEMENT\_Y1}. Geometrically the source term is linked to a polyline
named \texttt{BOTTOM}.

\begin{verbatim}
benchmark: hm_tri.gli
#POINTS
2 0.0 -1.0 0.0
3 0.1 -1.0 0.0
#POLYLINE
 $NAME
  BOTTOM
 $TYPE
  2
 $EPSILON
  1.0e-4
 $POINTS
 2
 3
#STOP
\end{verbatim}

The polyline \texttt{BOTTOM} is defined by two points 2 and 3.

%-------------------------------------------------------------------------------
\subsubsection{Boundary condition at surface}

\begin{verbatim}
benchmark: h_tet3.bc
#BOUNDARY_CONDITION
 $PCS_TYPE
  LIQUID_FLOW
 $PRIMARY_VARIABLE
  PRESSURE1
 $GEO_TYPE
  SURFACE BC_PRESSURE
 $DIS_TYPE
  CONSTANT 0.0
#STOP
\end{verbatim}

This boundary condition is defined for a process with primary
variable \texttt{PRESSURE1}. Geometrically the boundary condition
is linked to a surfaced named \texttt{BC\_PRESSURE}, which is
defined by a polyline.

\begin{verbatim}
benchmark: h_tet3.gli
#POINTS
2 -10.  10. 0.
3 -10. -10. 0.
4 -10. -10. 10.
5 -10.  10. 10.
#POLYLINE
 $NAME
  BC_PRESSURE
 $POINTS
  2
  3
  4
  5
#SURFACE
 $NAME
  BC_PRESSURE
 $POLYLINES
  BC_PRESSURE
 $EPSILON
  0.01
\end{verbatim}

}

\LastModified{OK - \today}
