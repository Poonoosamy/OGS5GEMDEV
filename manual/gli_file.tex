\section{GEOLib Data}

\begin{verbatim}
filename.gli
\end{verbatim}

The GLI file is the input file for GeoLib. Geometric objects are:
\begin{center}
\begin{tabular}{|l|l|l|c|}
\hline
Objects   & Keyword & Class                & Dimension \\
\hline \hline
%
Points    & \texttt{\#POINT}    & \texttt{CGLPoint}    & 0 \\
Lines     & \texttt{\#LINES}    & \texttt{CGLLine}     & 1 \\
Polylines & \texttt{\#POLYLINE} & \texttt{CGLPolyline} & 1 \\
Surfaces  & \texttt{\#SURFACE}  & \texttt{CGLSurface}  & 2 \\
Volumes   & \texttt{\#VOLUME}   & \texttt{CGLSurface}  & 3 \\
\hline
\end{tabular}
\end{center}

%------------------------------------------------------------------
\subsection{GEO types}
\label{sec:geo_types}

\begin{tabular*}{12.773cm}{|p{3.cm}|p{8.9cm}|} \hline
Parameter          & Meaning \\ \hline \hline
%
\texttt{POINT}     & Name of point \\
\texttt{POLYLINE}  & Name of polyline \\
\texttt{SURFACE}   & Name of surface \\
\texttt{VOLUME }   & Name of volume \\
\texttt{DOMAIN }   & whole domain \\
\hline
\end{tabular*}

%------------------------------------------------------------------
\subsection{Points}

Keyword: \texttt{\#POINTS}

\small
\begin{verbatim}
#POINTS
0 0.0 0.0 0.0 $MD 0.3 $NAME P0
1 1.0 0.0 0.0 $MD 0.3
2 1.0 3.0 0.0 $MD 0.3
3 0.3 0.0 0.0 $MD 0.3
4 0.0 0.0 3.0 $MD 0.3
5 1.0 0.0 3.0 $MD 0.3
6 1.0 3.0 3.0 $MD 0.3
7 0.0 3.0 3.0 $MD 0.3
\end{verbatim}
\normalsize

\small
\hrule

\begin{verbatim}
point number | x   y   z   | point properties with subkeywords

0              0.0 0.0 0.0   $MD 0.3 $NAME P0

\end{verbatim}

\hrule
\normalsize

\begin{center}
\begin{tabular}{|l|l|}
\hline
Properties   & Meaning \\
\hline \hline
%
\$MD            & Mesh density  \\
\$NAME          & Name of the point  \\
\hline
\end{tabular}
\end{center}

%------------------------------------------------------------------
\subsection{Lines}

Keyword: \texttt{\#LINES}

Lines are built from 2 points.

\small
\begin{verbatim}
#LINES
0  0 1
1  1 2
2  2 3
...
9  7 2
\end{verbatim}
\normalsize

\small
\hrule
\begin{minipage}[t]{4cm}
\begin{verbatim}
line number | point1 point2 (numbers)

0             0      1

\end{verbatim}
\end{minipage}
\hrule
\normalsize

%------------------------------------------------------------------
\subsection{Polylines}

Keyword: \texttt{\#POLYLINE}

Polylines are built from lines specified in sub-keyword
\texttt{\$LINES}

\small
\begin{verbatim}
#POLYLINE
 $NAME
  LOWER_FACE
 $EPSILON
  0.000000e+000
 $POINTS ; List of points identified by ID number
  22
  23
  24
  25
  22
 $POINT_VECTOR
  file_name.ply
 $MAT_GROUP
  mat_group_number
\end{verbatim}
\normalsize

\begin{center}
\begin{tabular}{|l|l|}
\hline
Sub-keyword & Objective \\
\hline \hline
%
\texttt{\$NAME}    &  name for identification \\
\texttt{\$TYPE}    &  type for use \\
\texttt{\$EPSILON} &  $\epsilon$ environment \\
\texttt{\$POINTS}   &  list of points building the polyline \\
\texttt{\$LINES}   &  list of lines building the polyline \\
\texttt{\$POINT\_VECTOR}   &  the name of one PLY file \\
\texttt{\$MAT\_GROUP}   &  connected material group \\
 \hline
%\caption{Polyline properties}
\end{tabular}
\end{center}

%------------------------------------------------------------------
\subsection{Surfaces}

Keyword: \texttt{\#SURFACE}

Surfaces are built from polylines specified in sub-keyword
\texttt{\$POLYLINES}. Surfaces should be completely closed by the
set of polylines.

\small
\begin{verbatim}
#SURFACE
 $NAME
  SOURROUND
 $POLYLINES ; List of polylines identified by name
  NORTHERN_FACE
  EASTERN_FACE
  SOUTHERN_FACE
  WESTERN_FACE
 $TIN
  SOURROUND.tin
 $MAT_GROUP
  mat_group_number
\end{verbatim}
\normalsize

\begin{center}
\begin{tabular}{|l|l|}
\hline
Sub-keyword & Objective \\
\hline \hline
%
\texttt{\$NAME}    &  name for identification \\
\texttt{\$TYPE}    &  type for use \\
\texttt{\$POLYLINES}  &  list of polylines building the surface \\
\texttt{\$TIN}    &   file name of a TIN belonging to the surface \\
\texttt{\$MAT\_GROUP}   &  connected material group \\
\hline
%\caption{Polyline properties}
\end{tabular}
\end{center}

Where \$TYPE takes integer value. Two special values of it are:
\begin{itemize}
    \item 3: Flat surface with any normal direction
    \item 100: Cylindrical surface between two cross round
    sections A and B. Four extra data are required to determine
    this surface as:
    gli points index of center of sections A and sections B, the
    radius of the cylinder, and a tolerance to select the element
    nodes close to the cylindrical surface.
\end{itemize}

Here is an example of a cylindrical surface with radius of 0.485:
 \small
\begin{verbatim}
#SURFACE
 $NAME
  SFS_TUNNEL
 $TYPE
   100
    0 4 0.485 1.0e-4
\end{verbatim}
\normalsize


%------------------------------------------------------------------
\subsection{Volumes}

Keyword: \texttt{\#VOLUME}

Volumes are built from surfaces specified in sub-keyword
\texttt{\$SURFACES}. Volumes should be completely closed by the
set of surfaces.

\small
\begin{verbatim}
#VOLUME
 $NAME
  CUBOID_DOMAIN
 $SURFACES ; List of surfaces identified by name
  BOTTOM
  TOP
  SOURROUND
 $MAT_GROUP
  mat_group_number
\end{verbatim}
\normalsize

\begin{center}
\begin{tabular}{|l|l|}
\hline
Sub-keyword & Objective \\
\hline \hline
%
\texttt{\$NAME}    &  name for identification \\
\texttt{\$TYPE}    &  type for use \\
\texttt{\$SURFACES}  &  list of surfaces building the volume \\
\texttt{\$MAT\_GROUP}   &  connected material group \\
\hline
%\caption{Polyline properties}
\end{tabular}
\end{center}

\vfill \LastModified{OK 17.11.2004}
