\section{Initial Conditions}
Initial conditions can be specified for saturation, fluid pressure,
temperature and concentration. Saturation, fluid pressure and temperature
refer to phases and concentration refers to components. The available
menus are controled by the selected physico-chemical model as well as
by the specified number of phases and components.

%----------------------------------------------------------
\subsection{Keywords}

The following keywords are available to specify initial conditions
for fluid flow, heat transport and mass transport.

\begin{verbatim}
#INITIAL_CONDITIONS_PRESSURE
#INITIAL_CONDITIONS_WATER_CONTENT
#INITIAL_CONDITIONS_SATURATION
#INITIAL_CONDITIONS_CONCENTRATION
#INITIAL_CONDITIONS_SOLUTE_CONCENTRATION
#INITIAL_CONDITIONS_SORBED_CONCENTRATION
#INITIAL_CONDITIONS_TEMPERATURE
#INITIAL_CONDITIONS_IMMOBILE_SOLUTE_CONCENTRATION
\end{verbatim}
Initial conditions for different phases and components are specified by
repeated use of the keyword. If the keyword is not repeated but more than
one initial conditions are specified under one keyword these conditions
are applied to the first phase.


%----------------------------------------------------------
\subsection{Methods}
\begin{table}[h]
%\caption{}
\begin{tabular}{|l|l|l|}
\hline Method & Subject & Au
\\
\hline \hline
0 & Constant value to all nodes & AH
\\
1 & Constant value for individual node given by node number & AH
\\
2 & Constant value for individual nodes given by coordinates & AH
\\
3 & Linear distribution between two nodes given by node numbers & AH
\\
4 & Linear distribution between two nodes given by coordinates & AH
\\
5 & Linear depth distribution to all nodes & AH/OK
\\
6 & Hydrostatic distribution to all nodes by z-coordinate ???? & AH
\\
7 & Specified distribution around node given by node number & AH
\\
8 & Specified distribution around node given by node coordinates & AH
\\
9 & Constant value at a plain given by coordinates & AH
\\
10 & Linear distribution between horizontal planes given by z-coordinates & AH
\\
11 & Bilinear distribution to all nodes within a rectangle given by node coordinates & RK
\\
20 & Constant value inside polygon, 2-D method $\Subset\Supset$ POLYLINE object & OK
\\
21 & Constant value along a polygon, 3-D method $\Subset\Supset$ POLYLINE object & OK
\\
\hline
\end{tabular}
\end{table}
\newpage
%----------------------------------------------------------
\subsection{Examples}
\hrule
Method 0:
A constant value is applied to all nodes.
%
\small
\begin{verbatim}
RFD - File:
#INITIAL_CONDITIONS_PRESSURE ; keyword
0 0 1.e5 ; method, mode, value
#INITIAL_CONDITIONS_CONCENTRATION ; keyword
0 0 0. ; method, mode, value
\end{verbatim}
\normalsize

\hrule
Method 1:
A constant value is applied to a node given by node number.
%
\small
\begin{verbatim}
RFD - File:
#INITIAL_CONDITIONS_SATURATION ; keyword
1 0 0 0.15 ; method, mode, node, value
\end{verbatim}
\normalsize

\hrule
Method 2:
A constant value is applied to a node given by coordinates.
%
\small
\begin{verbatim}
RFD - File:
#INITIAL_CONDITIONS_PRESSURE ; keyword
2 0 ; method, mode
100 1 0 1.e5 ; x, y, z, value
\end{verbatim}
\normalsize

\hrule
Method 3:
A linear value distribution is applied to nodes between two nodes given by
node numbers. Skipmode specifies to which nodes no values are assigned:\newline
If skipmode = 1: value is applied to every node on line \newline
if skipmode = 2: value is applied to 1., 3., 5.....node on line etc.
%
\small
\begin{verbatim}
RFD - File:
#INITIAL_CONDITIONS_PRESSURE ; keyword
3  0 ; method, mode
0  100 ; node1, node2
1  1.693567e+005  1.693567e+005 ; skipmode, value1, value2
\end{verbatim}
\normalsize

\hrule
Method 4:
A linear value distribution is applied to all nodes between two nodes given by
coordinates.
%
\small
\begin{verbatim}
RFD - File:
#REFERENCE_CONDITIONS
 9.810000  0.000000  101325.000000
#INITIAL_CONDITIONS_CONCENTRATION ; keyword
 4 0 ; method, mode
 1.500000e+02  0.000000e+00  1.500000e+02 200. ; x1, y1, z1, value1
 4.500000e+02  0.000000e+00  1.500000e+02 200. ; x2, y2, z2, value2
\end{verbatim}
\normalsize

\hrule
Method 5:
A linear depth distribution is applied to all nodes according to: \newline
$value = reference\_value + (node\_elevation - reference\_elevation) * gradient$
%
\small
\begin{verbatim}
RFD - File:
#INITIAL_CONDITIONS_PRESSURE  ; keyword
5 0 ; method, mode
150. 1.e5  -9810. ; reference_elevation, reference_value, gradient
\end{verbatim}
\normalsize

\hrule
Method 6:
Hydrostatic distribution to all nodes by z-coordinate according to: \newline
$value = (value1 - z1)*rho*g) - (node\_elevation - z1)*rho*g$ \newline
gek\"{u}rzt: $value = rho * g * (value1 - node\_elevation)$ \newline
ist nie hydrostatisch !!!!!!!
%
\small
\begin{verbatim}
RFD - File:
#INITIAL_CONDITIONS_PRESSURE  ; keyword
6 0 ; method, mode
100. 100. ; z1, value1
\end{verbatim}
\normalsize

\hrule
Method 7:
A distribution is assigned to nodes around a node given by node coordinates according
to the distribution\_type chosen: \newline
 0:  Cube with constant value \newline
 1:  Diamond with constant value \newline
 2:  Sphere with constant value \newline
-1:  Triangle or cone with value distribution $value = value1 - value1*dist/radius$ \newline
-2:  Circle or sphere with value distribution \newline
-3:  Gaussian distribution $value=value1*exp((-0.5)*dist*dist/radius/radius)$ \newline
-4:  Cylinder ???
%
\small
\begin{verbatim}
RFD - File:
#INITIAL_CONDITIONS_CONCENTRATION ; keyword
7 0 ; method, mode
-1 ; distribution_type,
0 ; node
100 50 ;  radius, value1
\end{verbatim}
\normalsize

\hrule
Method 8:
A distribution is assigned to nodes around a node given by node coordinates according
to the distribution\_type chosen: \newline
 0:  Cube with constant value \newline
 1:  Diamond with constant value \newline
 2:  Sphere with constant value \newline
-1:  Triangle or cone with value distribution $value = value1 - value1*dist/radius$ \newline
-2:  Circle or sphere with value distribution \newline
-3:  Gaussian distribution $value=value1*exp((-0.5)*dist*dist/radius/radius)$ \newline
-4:  Cylinder ???
%
\small
\begin{verbatim}
RFD - File:
#INITIAL_CONDITIONS_CONCENTRATION ; keyword
8 0 ; method, mode
-1 ; distribution_type
0 10 10 ; x, y, z,
2. 100. ;  radius, value1
\end{verbatim}
\normalsize

\hrule
Method 9:
A constant value is applied to all nodes in a plain given by coordinates.
%
\small
\begin{verbatim}
RFD - File:
#INITIAL_CONDITIONS_PRESSURE ; keyword
9 0 ; method, mode
0 0 0 ; x1, y1, z1
0 0 1 ; x2, y2, z2
0 1 0 ; x3, y3, z3
1.e5 ; value1
\end{verbatim}
\normalsize

\hrule
Method 10:
A linear depth distribution is applied to all nodes in between a specified number
of horizontal planes given by the z coordinates according to: \newline
$value = value1 + |node\_elevation -z1| * (value2 - value1) \quad / \quad |z2 - z1| $
%
\small
\begin{verbatim}
RFD - File:
#INITIAL_CONDITIONS_PRESSURE ; keyword
10 0 ; method, mode
2 0 100 50 0 ; number_of_plains, z1, value1, z2, value2
\end{verbatim}
\normalsize

\hrule
Method 11:
A bilinear distribution is applied to all nodes within a rectangle given by
node coordinates.
%
\small
\begin{verbatim}
RFD - File:
#INITIAL_CONDITIONS_CONCENTRATION ; keyword
11 0 ; method, mode
 0 0 0 100 ; x1, y1, z1, value1
50 0 0  50 ; x2, y2, z2, value2
 0 1 0 100 ; x3, y3, z3, value3
50 1 0  50 ; x4, y4, z4, value4
\end{verbatim}
\normalsize

\hrule
Method 20:
A constant value is assigned to all nodes within and close to a polyline (see method 21).
Initial conditions data are in the RFD file, whereas POLYLINE data are in the RFM file.\newline
(mode and distribution\_type are not used).
%
\small
\begin{verbatim}
RFD - File:
#INITIAL_CONDITIONS_TEMPERATURE ; keyword
 20 0 Conc1 0 340.0 ; method, mode, polyline_name, distribution_type, value
 20 0 Conc2 0 340.0 ; method, mode, polyline_name, distribution_type, value
RFM - File:
#POLYLINE
Conc1
4 ; number of polyline points
2 ; polyline type
2.566993e+005 -8.162393e+002 0.000000e+000 0.000000e+000 5.000000e+001
2.566993e+005 -2.264957e+002 0.000000e+000 0.000000e+000 5.000000e+001
2.515904e+005 -2.111111e+002 0.000000e+000 0.000000e+000 5.000000e+001
2.514379e+005 -8.111111e+002 0.000000e+000 0.000000e+000 1.000000e-001
0.000000e+000 ; epsilon
1 ; closed
#POLYLINE
Conc2
4 ; number of polyline points
2 ; polyline type
2.619608e+005 -8.213675e+002 0.000000e+000 0.000000e+000 1.000000e-001
...
\end{verbatim}
\normalsize

\hrule
Method 21:
A constant value is assigned to all nodes close to a polyline (3-D method ????).
Initial conditions data are in the RFD file, whereas POLYLINE data are in the RFM file.\newline
(mode, distribution\_type and radius are not used(epsilon in rfm)).
%
\small
\begin{verbatim}
RFD - File:
#INITIAL_CONDITIONS_TEMPERATURE ; keyword
 21 0 Conc1 0 340.0 1 ; method, mode, polyline_name, distribution_type, value, radius
 21 0 Conc2 0 340.0 1 ; method, mode, polyline_name, distribution_type, value, radius
\end{verbatim}
\normalsize

%----------------------------------------------------------
\subsection{GUI}



\LastModified{MB - \today}
