\section{Data Output}
\label{sec:out}

\begin{tabular*}{8.35cm}{|p{2.5cm}|p{5cm}|} \hline
Object acronym & OUT \\
C++ class      & COutput \\
Source files   & rf\_out.h/cpp \\
\hline
File extension & *.out \\
Object keyword & \#OUTPUT \\
\hline
\end{tabular*}

Three types of data output are
available:
\begin{itemize}
    \item VAR\_TYPE: Output for variables (primary and secondary).
    If no PCS\_TYPE and no MSH\_TYPE are specified, all PCS are
    screened for the given variables (standard case).
    \item PCS\_TYPE: Output for processes. Output only for the
    given PCS.
    \item MSH\_TYPE: Output for meshes. Output only for the given
    MSH.
\end{itemize}

%-------------------------------------------------------------------------------
\subsection{\texttt{\bf\#OUTPUT}}

%\subsubsection{Keyword structure}

\begin{verbatim}
#VERSION       //Given:  show version in all output file names
#OUTPUT
 $PCS_TYPE // physical process
  LIQUID_FLOW       // H process (incompressible flow)
  UNCONFINED_FLOW   // H process (incompressible flow)
  GAS_FLOW          // H process (compressible flow)
  TWO_PHASE_FLOW    // H2 process (incompressible/compressible flow)
  COMPONENTAL_FLOW  // H2 process (incompressible/compressible flow)
  RIVER_FLOW        // H process (incompressible flow)
  RICHARDS_FLOW     // H process (incompressible flow)
  OVERLAND_FLOW     // H process (incompressible flow)
  HEAT_TRANSPORT    // T process (single/multi-phase flow)
  DEFORMATION       // M process (single/multi-phase flow)
  MASS_TRANSPORT    // C process (single/multi-phase flow)
  GROUNDWATER_FLOW  // H process (incompressible flow)
  FLUID_MOMENTUM
 $MSH_TYPE // mesh
  msh_name //4.2.14(OK)
 $NOD_VALUES // specified node quantities
  PRESSUREx
  SATURATIONx
  TEMPERATURE1
  DISPLACEMENT1_X
  DISPLACEMENT1_Y
  DISPLACEMENT1_Z
  CONCENTRATION1
  CONCENTRATIONx
  STRESS_XX
  STRESS_XY
  STRESS_YY
  STRESS_ZZ
  STRESS_XZ
  STRESS_YZ
  STRAIN_XX
  STRAIN_XY
  STRAIN_YY
  STRAIN_ZZ
  STRAIN_XZ
  STRAIN_YZ
  STRAIN_PLS
  VELOCITY1_X
  VELOCITY1_Y
  VELOCITY1_Z
 $ELE_VALUES // specified element quantities
  VELOCITY1_X
  VELOCITY1_Y
  VELOCITY1_Z
  MASS_FLUX1_X
  MASS_FLUX1_Y
  MASS_FLUX1_Z
 $GEO_TYPE // geometry
  POINT    name
  POLYLINE name
  SURFACE  name
  VOLUME   name
  DOMAIN
  LAYER //4.3.20
 $TIM_TYPE // output times
  time1
  ...
  timex
 $DAT_TYPE // output file format
  TECPLOT
  ROCKFLOW // 4.2.14(OK)
  VTK      // 4.3.XX (GK)
 $DIS_TYPE // 4.2.14(OK)
  AVERAGE
 $AMPLIFIER // to amplify output data
  scale
#STOP
\end{verbatim}

In the case of  axisymmetrical deformation problem, the output of
stresses, strains and displacements have the following meanings:
\begin{gather}
   \stress_{xx}=\stress_{rr}, \quad  \stress_{yy}=\stress_{\theta\theta},\quad
  \stress_{zz}=\stress_{zz}, \stress_{xy}=\stress_{rz} \nonumber\\
   \strain_{xx}=\strain_{rr}, \quad  \strain_{yy}=\strain_{\theta\theta},\quad
  \strain_{zz}=\strain_{zz}, \strain_{xy}=\strain_{rz} \nonumber\\
  u_{x}=u_{r},\quad u_z=u_z \nonumber
\end{gather}

\begin{tabular*}{12.773cm}{|p{3.cm}|p{1.5cm}|p{7cm}|} \hline
Subkeyword          & Acronym & Meaning \\ \hline \hline
%
\texttt{PCS\_TYPE}   & PCS &  Specified process for output \\
\texttt{MSH\_TYPE}   & MSH &  Specified mesh for output \\
\texttt{NOD\_VALUES} & NOD &  Specified node values for output \\
\texttt{ELE\_VALUES} & ELE &  Specified element values for output \\
\texttt{GEO\_TYPE}   & GEO &  Related geometric objects \\
\texttt{TIM\_TYPE}   & TIM &  Specified output times \\
\texttt{DAT\_TYPE}   & DAT &  Output file format \\
\texttt{DIS\_TYPE}   & DIS &  Output methods (e.g. averaging) \\
\hline
\end{tabular*}

\subsubsection{\texttt{\$PCS\_TYPE}}

see section \ref{sec:pcs}.

\subsubsection{\texttt{\$MSH\_TYPE}}

see section \ref{sec:msh}.

\subsubsection{\texttt{\$NOD\_VALUES}}

see section \ref{sec:nod_values}

\subsubsection{\texttt{\$GEO\_TYPE}}

see section \ref{sec:geo_types}

%OK
\small
\begin{verbatim}
LAYER //4.3.20
\end{verbatim}
\normalsize \vspace{-2mm}
%
This specification produces layer output for regional processes
(e.g. regional soil model).

\subsubsection{\texttt{\$TIM\_TYPE}}

\begin{tabular*}{12.773cm}{|p{3.cm}|p{8.9cm}|} \hline
Parameter        & Meaning \\ \hline \hline
%
\texttt{...}     & List of output times \\
\texttt{STEPS}   & Interval of output steps \\
\hline
\end{tabular*}

\subsubsection{\texttt{\$DAT\_TYPE}}

\begin{tabular*}{12.773cm}{|p{3.cm}|p{8.9cm}|} \hline
Parameter        & Meaning \\ \hline \hline
%
\texttt{TECPLOT}  & Tecplot file format (tec file) \\
\texttt{ROCKFLOW} & RockFlow file format (rfo file) \\
\texttt{VTK} & Paraview file format (vtk file) \\
\hline
\end{tabular*}

\subsubsection{\texttt{\$DIS\_TYPE}}

\begin{tabular*}{12.773cm}{|p{3.cm}|p{8.9cm}|} \hline
Parameter        & Meaning \\ \hline \hline
%
\texttt{AVERAGE} & nodal average \\
\hline
\end{tabular*}


\Examples{
%===============================================================================
\subsection{Examples}

%-------------------------------------------------------------------------------
\subsubsection{VAR\_TYPE: Output for variables}

%...............................................................................
\subsubsection*{Output files}

The names of the OUT files are generated automatically:

\begin{tabular*}{12.773cm}{|p{3.cm}|p{8.9cm}|} \hline
Parameter          & File name \\ \hline \hline
%
Domain      & node values: \texttt{filename\_dom\_nod.tec} \\
Domain      & element values: \texttt{filename\_dom\_ele.tec} \\
Time curves & \texttt{filename\_time\_GEOName.tec} \\
Profiles    & \texttt{filename\_GEOName\_TIMStepNumber.tec} \\
\hline
\end{tabular*}

%...............................................................................
\subsubsection*{Domain output}

Data output of \texttt{PRESSURE1} at $t$ = 4.320000e+003 sec for
whole domain.

\begin{verbatim}
benchmark: h_tet3.out
#OUTPUT // domain
 $NOD_VALUES
  PRESSURE1
 $GEO_TYPE
  DOMAIN
 $DAT_TYPE
  TECPLOT
 $TIM_TYPE
  4.320000e+003
#STOP
\end{verbatim}

Data output of \texttt{PRESSURE1} each time step for whole domain.

\begin{verbatim}
#OUTPUT // domain
 $NOD_VALUES
  PRESSURE1
 $GEO_TYPE
  DOMAIN
 $DAT_TYPE
  TECPLOT
 $TIM_TYPE
  STEPS 1
#STOP
\end{verbatim}

%...............................................................................
\subsubsection{Time curve output}

Data output of \texttt{PRESSURE1} and \texttt{TEMPERATURE1} in
\texttt{POINT2} for all time steps.

\begin{verbatim}
benchmark: ht_line.out
#OUTPUT // time curve
 $NOD_VALUES
  PRESSURE1
  TEMPERATURE1
 $GEO_TYPE
  POINT POINT2
 $DAT_TYPE
  TECPLOT
#STOP
\end{verbatim}

Data output of node average at surface \texttt{OUT} for all time steps.

\begin{verbatim}
benchmark: h2_line.out
#OUTPUT // profile
 $NOD_VALUES
  CONCENTRATION
 $GEO_TYPE
  SURFACE OUT
 $TIM_TYPE
  STEPS 1
 $DIS_TYPE
  AVERAGE
#STOP
\end{verbatim}

%...............................................................................
\subsubsection{Profile output}

Data output of \texttt{PRESSURE1} and \texttt{SATURATION2} along
polyline \texttt{OUT} at times $t$ = 5e+5, 1e+6, 5e+6, 1e+7, 5e+7,
1e+8 and 2e+8 sec.

\begin{verbatim}
benchmark: h2_line.out
#OUTPUT // profile
 $NOD_VALUES
  PRESSURE1
  SATURATION2
 $GEO_TYPE
  POLYLINE OUT
 $DAT_TYPE
  TECPLOT
 $TIM_TYPE
  5e+5
  1e+6
  5e+6
  1e+7
  5e+7
  1e+8
  2e+8
#STOP
\end{verbatim}

%-------------------------------------------------------------------------------
\subsubsection{MSH\_TYPE: Output for meshes}

%...............................................................................
\subsubsection*{Output files}

The names of the OUT files are generated automatically:

\begin{tabular*}{12.773cm}{|p{3.cm}|p{8.9cm}|} \hline
Parameter          & File name \\ \hline \hline
%
Domain      & node values: \texttt{filename\_dom\_MSHName\_nod.tec} \\
Domain      & element values: \texttt{filename\_dom\_MSHName\_ele.tec} \\
Time curves & \texttt{filename\_time\_MSHName\_GEOName.tec} \\
Profiles    & \texttt{filename\_MSHName\_GEOName\_TIMStepNumber.tec} \\
\hline
\end{tabular*}

Output for two meshes (regional soil model).

\begin{verbatim}
benchmark: 2.out
#OUTPUT
 $MSH_TYPE
  SURFACE0
 $NOD_VALUES
  PRESSURE1
  SATURATION1
 ...
#OUTPUT
 $MSH_TYPE
  SURFACE1
 $NOD_VALUES
  PRESSURE1
  SATURATION1
 ...
#STOP
\end{verbatim}


%-------------------------------------------------------------------------------
\subsubsection{PCS\_TYPE: Output for meshes}

%...............................................................................
\subsubsection*{Output files}

The names of the OUT files are generated automatically:

\begin{tabular*}{12.773cm}{|p{3.cm}|p{8.9cm}|} \hline
Parameter          & File name \\ \hline \hline
%
Domain      & node values: \texttt{filename\_dom\_PCSName\_nod.tec} \\
Domain      & element values: \texttt{filename\_dom\_PCSName\_ele.tec} \\
Time curves & \texttt{filename\_time\_PCSName\_GEOName.tec} \\
Profiles    & \texttt{filename\_PCSName\_GEOName\_TIMStepNumber.tec} \\
\hline
\end{tabular*}


}

\LastModified{OK - 30.12.2005}
