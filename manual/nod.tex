\Developer{
\section{NOD Node Data}

\textbf{Data Object - Nodes}

\footnotesize
\begin{verbatim}
typedef struct {        /* Knotenkoordinaten */
    double x;           /* Koordinaten */
    double y;
    double z;
    long node_start_number;     /* Knotennummer im Startnetz */
    double source;      /* Quelle */
    long index;         /* aktueller Index nach Umnummerierung, sonst -1 */
    long *elems1d;      /* zugehoerige 1D-Elemente */
    long *elems2d;      /* zugehoerige 2D-Elemente */
    long *elems3d;      /* zugehoerige 3D-Elemente */
    int anz1d;          /* Anzahl der zugehoerigen 1D-Elemente */
    int anz2d;          /* Anzahl der zugehoerigen 2D-Elemente */
    int anz3d;          /* Anzahl der zugehoerigen 3D-Elemente */
    long *plains;       /* Wird zum Aufbau des Flaechenverzeichnisses (Startnetz) gebraucht */
    int anz_plains;     /* Waehrend der Adaption werden diese Daten nicht aktualisiert !!!!  */
    long *edges;        /* Wird zum Aufbau des Kantenverzeichnisses (Startnetz) gebraucht */
    int anz_edges;      /* Waehrend der Adaption werden diese Daten nicht aktualisiert !!!!  */
    long *newnodes;     /* benachbarte neugenerierte Knoten fuer 9-Knoten-Elemente */
    int anz_new_nodes;

    int status;         /* -1: regulaerer Innenknoten
                           -2: irregulaerer Kantenknoten
                           -3: regulaerer Randknoten
                           -4: irregulaerer Flaechenknoten */
    int corner_node;    /* corner_node flag = 1 if Eckknoten */
    int free_surface;   /* 1: bewegliche Knoten oben */
                        /* 2: bewegliche Knoten unten */
    double *nval;       /* modellabhaengige Knotenwerte, z.B. h und conc */
    void *nval_intern;            /* interne modellabhaengige Knotendaten */
    Randbedingung *randbedingung; /* Zeiger auf Randbedingungen */
} Knoten;

\end{verbatim}
}
