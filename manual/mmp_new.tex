\textcolor[rgb]{0.98,0.00,0.00}{\section{Porous Medium Properties}}

\begin{tabular*}{8.35cm}{|p{2.5cm}|p{5cm}|} \hline
Object acronym & MMP \\
C++ class      & CMediumProperties \\
Source files   & rf\_mmp.h/cpp \\
\hline
File extension & *.mmp \\
Object keyword & \#MEDIUM\_PROPERTIES \\
\hline
\end{tabular*}

\Developer{
%----------------------------------------------------------
\subsection{Theory}
}
%-------------------------------------------------------------------------------
\textcolor[rgb]{0.98,0.00,0.00}{\subsection{\texttt{\bf\#MEDIUM\_PROPERTIES - TORTUOSITY Changed}}}

%\subsubsection{Keyword structure}

\begin{verbatim}
#MEDIUM_PROPERTIES
// Data base
 $MEDIUM_TYPE
   CLAY
   SILT
   SAND
   GRAVEL
   CRYSTALLINE
// Geometric properties
 $GEOMETRY_DIMENSION
   dim // 1,2,3-D
 $GEOMETRY_AREA
   area // area for 1D element, thickness for 2D element
   FILE file.dat // input of distributed data (node or element wise)
 $GEO_TYPE
  POINT    point_name
  POLYLINE polyline_name
  SURFACE  surface_name
  VOLUME   volume_name
  LAYER    Layer_number
 $POROSITY
  model model_parameters
 $TORTUOSITY
  model model_parameters
  or
  ISOTROPIC   tortuosity
  ANISOTROPIC tortuosities
  FILE file_name
// Hydraulic properties
 $STORATIVITY
  model model_parameters
 $PERMEABILITY_TENSOR
  ISOTROPIC   permeability
  ORTHOTROPIC permeabilities
  ANISOTROPIC permeabilities
  FILE file_name
 $UNCONFINED
 $PERMEABILITY_FUNCTION_SATURATION
  model model_parameters for first fluid phase
  model model_parameters for second fluid phase
  ...
 $PERMEABILITY_FUNCTION_DEFORMATION
  model model_parameters
 $PERMEABILITY_FUNCTION_PRESSURE
  model model_parameters
 $PERMEABILITY_FUNCTION_STRESS
  model model_parameters
 $PERMEABILITY_FUNCTION_VELOCITY
  model model_parameters
 $PERMEABILITY_FRAC_APERTURE
  average_type roughness_correction
 $CAPILLARY_PRESSURE
  model model_parameters
 // Thermal properties
 $HEAT_DISPERSION
  model model_parameters
 // Mass transport properties
 $MASS_DISPERSION
  model model_parameters
 // Electric properties
 $ELECTRIC_CONDUCTIVITY
  model model_parameters
 // Multi-continua properties
 $FLUID_EXCHANGE
  model model_parameters
 $MASS_EXCHANGE
  model model_parameters
 $PERMEABILITY_DISTRIBUTION
  file_name
 $MANNING_COEFFICIENT
  value
 $FRACTURE_DATA
  number_of_fractures names_of_fractures
#STOP
\end{verbatim}

%==========================================================
\subsection{Medium model data input}
%----------------------------------------------------------
\subsubsection{Porosity}
\paragraph*{Case 11: Read in a file}
\begin{verbatim}
$POROSITY
 11 poro_layer1.dat
\end{verbatim}

%----------------------------------------------------------
%----------------------------------------------------------
\subsubsection{Friction coefficient for overland or channel flow}
\begin{verbatim}
$MANNING_COEFFICIENT
 0.15

$CHEZY_COEFFICIENT
 10
\end{verbatim}

%----------------------------------------------------------
%----------------------------------------------------------
\subsubsection{Permeability}
\paragraph*{PERMEABILITY\_TENSOR}
\begin{verbatim}
$PERMEABILITY_TENSOR
 FILE perm_layer1.dat
\end{verbatim}
%----------------------------------------------------------
%----------------------------------------------------------
\subsubsection{Discrete Fracture Permeability}
\begin{verbatim}
$PERMEABILITY_FRAC_APERTURE
  average_type roughness_correction
\end{verbatim}
\begin{itemize}
  \item average\_type: Arithmetic, Geometric, or Harmonic
  \item roughness\_correction: corr\_roughness OR no\_corr\_roughness
\end{itemize}
The average type describes how the average aperture (which is subsequently used to calculate the permeability) will be calculated. The permeability calculation is based on the cubic law. If $corr\_roughness$ is entered, this cubic law permeability will be corrected depending on the roughness of the fracture walls and the closure ratio of the fracture. This correction is based on:

Zimmerman RW, Bodvarsson GS (1996) Hydraulic Conductivity of Rock Fractures.
\emph{Transport in Porous Media} 23: 1-30

\emph{Note:} For this function to work polylines must be defined for the upper and lower fracture surface profiles. See \emph{frac\_test.gli}.

%----------------------------------------------------------
%----------------------------------------------------------
\subsubsection{Confined or unconfined flow}
Standard is the confined flow is modelled. No additional keyword
is required. If the mmp group is unconfined the keyword
\begin{verbatim}
$UNCONFINED
\end{verbatim}
has to be used.
%----------------------------------------------------------
%----------------------------------------------------------
\subsubsection{Relative permeability - saturation}
%----------------------------------------------------------


%-------------------------------------------------------------------------------
\paragraph*{Case 0: User-defined function by \#CURVE}
\begin{eqnarray}
  \PermRelS^\Phase= f(u)
\end{eqnarray}
\begin{verbatim}
$PERMEABILITY_SATURATION
 0 curve_number
\end{verbatim}

%-------------------------------------------------------------------------------
\paragraph*{Case 2: Linear function}

\paragraph*{Case 21: Linear function from saturation}
\begin{eqnarray}
\PermRelS^\Phase = 1-\Saturation^\Phase
\end{eqnarray}
\begin{verbatim}
$PERMEABILITY_SATURATION
 21
\end{verbatim}

%-------------------------------------------------------------------------------
\paragraph*{Case 4: van Genuchten (1980)}
\begin{eqnarray}
\SaturationEff
&=&
\frac{\Saturation^\Phase-\SaturationRes^\Phase}{\SaturationMax^\Phase-\SaturationRes^\Phase}
\\
\PermRelS^l &=& \SaturationEff^{1/2} \left[
1-(1-\SaturationEff^{1/m})^m \right]^2
\end{eqnarray}
\begin{verbatim}
$PERMEABILITY_SATURATION
 4  s_res  s_max  m
\end{verbatim}

%-------------------------------------------------------------------------------
\paragraph*{Case 14: van Genuchten 2 (1980)}
\begin{eqnarray}
\SaturationEff &=&
\frac{\Saturation^\Phase-\SaturationRes^\Phase}{\SaturationMax^\Phase-\SaturationRes^\Phase}
\\
\PermRelS^l &=& \SaturationEff^{0.5} \left[
1-(1-\SaturationEff^{1/m})^m \right]^2
\end{eqnarray}
\begin{verbatim}
$PERMEABILITY_SATURATION
 14  s_res  s_max  m
\end{verbatim}


%-------------------------------------------------------------------------------
\paragraph*{Case 14: van Genuchten (1980) for non-wettable phase}
\begin{eqnarray}
\SaturationEff &=&
\frac{\Saturation^\Phase-\SaturationRes^\Phase}{\SaturationMax^\Phase-\SaturationRes^\Phase}
\\
\PermRelS^g &=& (1-\SaturationEff)^{0.5} \left[
1-(1-\SaturationEff) \right]^{2m}
\end{eqnarray}
\begin{verbatim}
$PERMEABILITY_SATURATION
 15  s_res  s_max  m
\end{verbatim}

%----------------------------------------------------------
\subsubsection{Capillary pressure}

%-------------------------------------------------------------------------------
\textcolor[rgb]{0.98,0.00,0.00}{\paragraph*{Case 4: van Genuchten
(1980)}}
\begin{eqnarray}
p_c &=& \frac{\rho^lg}{\alpha}(\SaturationEff^{1/m}-1 )^{(1-m)}
\end{eqnarray}
\begin{verbatim}
 $CAPILLARY_PRESSURE
 4  alpha
 \end{verbatim}
%==========================================================
\Examples{
\newpage
%-------------------------------------------------------------------------------
%----------------------------------------------------------
\subsubsection{Discrete Fracture Data}
\begin{verbatim}
 $FRACTURE_DATA
  number_of_fractures names_of_fractures
\end{verbatim}
\begin{itemize}
  \item number\_of\_fractures: The number of discrete fractures in the model domain. Each fracture is handled as a separate entity.  
  \item names\_of\_fractures: The names of each of these fractures. For each fracture, polylines must be created representing the upper and lower fracture surface profiles. These polylines must have the names \emph{name\_of\_fracture\_top} and \emph{name\_of\_fracture\_bot}. See \emph{frac\_test.gli}.
\end{itemize}
%----------------------------------------------------------
\subsection{Examples}

%-------------------------------------------------------------------------------
\subsubsection{Single phase flow}

\begin{verbatim}
benchmark: h_line.mmp
#MEDIUM_PROPERTIES
 $GEOMETRY_DIMENSION
  1
 $GEOMETRY_AREA
  1.000000e+000
 $POROSITY
  1 2.000000e-001
 $TORTUOSITY
  1  1.000000e+000
 $PERMEABILITY_TENSOR
  ISOTROPIC 1.000000e-07
#STOP
\end{verbatim}


%-------------------------------------------------------------------------------
\subsubsection{Unconfined groundwater flow}

\begin{verbatim}
benchmark: beerze-reuzel.mmp
#MEDIUM_PROPERTIES
 $NAME
  Layer1
 $GEO_TYPE
  LAYER 1
 $GEOMETRY_DIMENSION
  2
 $GEOMETRY_AREA
  1.000000000000e+000
 $POROSITY
   11 sf1_1x.dat1
 $PERMEABILITY_TENSOR
   FILE   kd1_simgroq.dat1
 $UNCONFINED
\end{verbatim}

%-------------------------------------------------------------------------------
\subsubsection{Two phase flow}

\begin{verbatim}
benchmark: h2_line.mmp
#MEDIUM_PROPERTIES
 $GEOMETRY_DIMENSION
  1
 $GEOMETRY_AREA
  1.000000e+000
 $POROSITY
  1 2.000000e-001
 $TORTUOSITY
  1 1.000000e+000
 $PERMEABILITY_TENSOR
  ISOTROPIC 1.000000e-07
 $PERMEABILITY_SATURATION
  3 0.2 0.8 2.
  3 0.2 0.8 2.
 $CAPILLARY_PRESSURE
  0:
#STOP
\end{verbatim}

%-------------------------------------------------------------------------------
\subsubsection{Non-isothermal two phase flow}

\begin{verbatim}
benchmark: th2_line.mmp
#MEDIUM_PROPERTIES
 $GEOMETRY_DIMENSION
  1
 $GEOMETRY_AREA
  1.000000e-2
 $POROSITY
  1 0.407407407
 $TORTUOSITY
  1 0.8
 $PERMEABILITY_TENSOR
  ISOTROPIC: 8.22854E-20
 $PERMEABILITY_SATURATION
  21 0.0 0.9
  4  0.1 1.0 1.0
 $PERMEABILITY_DEFORMATION
  1 1.0 1.0 7.0 3293673.0 -0.165 3.0
 $CAPILLARY_PRESSURE
  0 5
#STOP
\end{verbatim}

%-------------------------------------------------------------------------------

\subsubsection{Richards flow}
\begin{verbatim}
benchmark: h_us_line.mmp #MEDIUM_PROPERTIES
 $GEOMETRY_DIMENSION
  3
 $GEOMETRY_AREA
  1.000000e+000
 $POROSITY
  1 2.000000e-001
 $TORTUOSITY
  1 1.000000e+000
 $PERMEABILITY_TENSOR
  ISOTROPIC 1.000000e-07
 $PERMEABILITY_SATURATION
  4  0  0.6452  0.791667
 $CAPILLARY_PRESSURE
  4  320
#STOP
\end{verbatim}

%-------------------------------------------------------------------------------
\subsubsection{Consolidation}

\begin{verbatim}
benchmark: hm_tri.mmp
#MEDIUM_PROPERTIES
 $GEOMETRY_DIMENSION
  2
 $GEOMETRY_AREA
  1.0
 $POROSITY
  1 0.000000e-001
 $TORTUOSITY
  1 1.000000e+000
 $PERMEABILITY_TENSOR
  ISOTROPIC 1.000000e-10
#STOP
\end{verbatim}

%-------------------------------------------------------------------------------
\subsubsection{Heat transport}

\begin{verbatim}
benchmark: ht_line.mmp
#MEDIUM_PROPERTIES
 $GEOMETRY_DIMENSION
  1
 $GEOMETRY_AREA
  1.0
 $POROSITY
  1 2.000000e-001
 $PERMEABILITY_TENSOR
  ISOTROPIC 1.000000e-11
 $HEAT_DISPERSION
  1 5.000000e+000 0.000000e+000
#STOP

\end{verbatim}

%-------------------------------------------------------------------------------
\subsubsection{Discrete Fracture Deformation}

\begin{verbatim}
benchmark: frac_test.mmp
#MEDIUM_PROPERTIES
 $GEOMETRY_DIMENSION
  2
 $GEOMETRY_AREA
  1.000000e+000
 $POROSITY
  1 1.000000e-001
 $TORTUOSITY
  1 1.000000e+000
 $PERMEABILITY_TENSOR
  ORTHOTROPIC 1 1
 $PERMEABILITY_FRAC_APERTURE
  Arithmetic corr_roughness
 $FRACTURE_DATA
  1 Frac0
#MEDIUM_PROPERTIES
 $GEOMETRY_DIMENSION
  2
 $GEOMETRY_AREA
  1.000000e+000
 $POROSITY
  1 1.000000e-005
 $TORTUOSITY
  1 1.000000e+000
 $PERMEABILITY_TENSOR
  ISOTROPIC 1.000000e-16
#STOP

\end{verbatim}
The two material groups represent the medium properties of the fracture and matrix material respectively.  
%-------------------------------------------------------------------------------
\subsubsection{Distributed data}

\begin{verbatim}
benchmark: fracnet02.mmp
#MEDIUM_PROPERTIES
 $PERMEABILITY_DISTRIBUTION
  permeabilities.txt
\end{verbatim}


} % Examples

\LastModified{OK - \today}
