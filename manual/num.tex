\section{Numerics}
There are many alternative methods to solve initial-boundary-value problems arising from flow and transport processes in
subsurface systems. In general these methods can be classified into analytical and numerical ones. Analytical solutions
can be obtained for a number of problems involving linear or quasi-linear equations and calculation domains of simple
geometry. For non-linear equation or problems with complex geometry or boundary conditions, exact solutions usually do not
 exist, and approximate solutions must be obtained. For such problems the use of numerical methods is advantageous. Rockflow
 uses the Finite Difference Method for time discretization and  uses Finite Element/Volume for spatial discretization.
 The Gelerkin wighted residual method, which is more general in application than variational method, is employed to
 provide the weak formula of the PDEs. The Gauss integral formula and modified Gauss integral formula for triangle
 element/tedrahedron element are used to numerical integrate the integrals arising from the weak formula. This section
 introduces the keywords and data that manage the numerical methods used in computation.



%-------------------------------------------------------------------------------
\subsubsection{\bf\$NUM\_TYPE}

\begin{tabular*}{12.773cm}{|p{3.cm}|p{1.5cm}|p{7cm}|} \hline
Parameter          &  & Meaning \\ \hline \hline
%
\bf FEM    &  & Finite element method \\
FDM    &  & Finite difference method \\
...    &  & e.g. particle tracking, MOC ... \\
\hline
\end{tabular*}

Comments: Default is FEM. The finite difference method is
available for 1-D Saint Venant equations.

%-------------------------------------------------------------------------------
\subsubsection{\bf\$NOL\_ITERATIONS}

\begin{tabular*}{12.773cm}{|p{3.cm}|p{1.5cm}|p{7cm}|} \hline
Parameter          & Value & Meaning \\ \hline \hline
%
number & \bf 1 & Number of non-linear iterations for this process \\
\hline
\end{tabular*}

%-------------------------------------------------------------------------------
\subsubsection{\bf\$CPL\_ITERATIONS}

\begin{tabular*}{12.773cm}{|p{3.cm}|p{1.5cm}|p{7cm}|} \hline
Parameter          & Value & Meaning \\ \hline \hline
%
number & \bf 1 & Number of coupling iterations over all processes per time step (partitioned process coupling) \\
\hline
\end{tabular*}





Keyword \textbf{\#NUMERICS} leads the configuration of the basic numerical method given in
Table \ref{tab_num1}.
{\ttfamily
  \small
\begin{table}[htb]
\begin{tabular}{|l|c|l|l|}
  \hline
  Position &Value & Data type & Meaning \\
  \hline
  \hline % 1st row
  1 &   &Integer  &Method for the weak formula\\
    & \textbf{0} &  &Bubnov-Galerkin-Method (Standard FEM)\\
    & 1 &         &Petrov-Galerkin-Method (Upwind FEM)\\
    & 2 &         &Eular-Taylor-Galerkin-Method (ETG) \\
  \hline % 2nd row
   2&   &String &Name, the identification of the object \\
  \hline % 3rd row
   3&   &Integer &Number of Gaussian integration points\\
  \hline % 4th row
   4&   &Float   &$\in [\textbf{0},1]$. Time collocation factor\\
  \hline % 5th row
   5&   &Float   &$\in [\textbf{0},1]$. Upwind parameter\\
  \hline % 6th row
   6&   & Integer   &Larange advection \\
    &  \textbf{0} &    &Larange scheme for advective terms disabled\\
    &  1 &    &Larange scheme enabled (Operator-Splitting)\\
  \hline % 7th row
   6&   & Float   &Larange quality \\
    &  [0,1] &    &Parameter of Larange scheme for advection terms\\
    &  \textbf{0.95} &    &To control approximation accuracy\\
  \hline % 8th row
\end{tabular}
\caption{Data by keyword \textbf{\#NUMERICS}.
 The default values are in bold face}
\label{tab_num1}
\end{table}
}



The following keywords can be used for different processes. Users can choose one or more than one of them for
specified problem:
\begin{center}
  \begin{verbatim}
   #NUMERICS_CONCENTRATION
   #NUMERICS_DEFORMATION
   #NUMERICS_GEOELECTRIC
   #NUMERICS_ITERATION
   #NUMERICS_PRESSURE
   #NUMERICS_SATURATION
   #NUMERICS_VELOCITY
 \end{verbatim}
\end{center}

Followed these keywords can be a number of sub-keywords, which are labelled by a \$ sign and
explained in the following.
\begin{description}
    \item[\$METHOD] Integer number
    {
      \begin{itemize}
      \item \textbf{0}: No calculation
      \item \textbf{1}: Finite element calation
      \item \textbf{3}: For transport calculation only. Using Lagrange scheme for advective transport in
                   2D fracture element. Compute everything else by FEM. Requires additional parameters.
      \item \textbf{4}: For transport calculation only. Using Lagrange scheme for 2D fracture elements.
                    Omit everything else. Requires additional parameters.
      \item \textbf{5}: For transport calculation only. Using Lagrange scheme for 2D elements.
                   Requires additional parameters.
      \item \textbf{7}: For unsaturated flow (pressure and saturation). The Richards' formula is used to calculate
                   pressure and saturation
      \item \textbf{8}: For unsaturated flow (pressure and saturation). The Celia-Richards' formula is used to calculate
                   pressure and saturation
      \item \textbf{9}: For transport calculation only. Finite element calculation with artificial diffusion in critical
                        regions. Reads additionally two float values. The first parameter $p_1$ specifies the
                        concentration difference with marks an element as critical one. If the parameter is negative,
                        concentration will be normalized with the span of all concentrations in the system before
                        comparison (recommended value is 0.25, i.e. the front should be smoothed less than four elements
                        wide. ). The second parameter $p_2$ specifies the additional diffusion multiplier. The diffusion
                        multiplier is computed from $p_2^{(c\_max\_min-p_1)/p_1}$, where $c\_max\_min$ is the normalized
                        concentration difference within the regarded element.
      \item \textbf{11}: Same as \textbf{9}.
      \end{itemize}
    }
    \item[\$GAUSS\_POINTS] Number of Gauss points in numerical integration.
    \item[\$NONLINEAR\_COUPLING] Integer number 1 or 0 to specifies whether the regarded quantity shall be
                                 regarded within the loop for non-linear iteration.
    \item[\$TRANSPORT\_IN\_PHASE] Number of phases for which transport processes shall be regarded. For
                  \textbf{-1}, all phases are regarded
    \item[\$PLASTIC\_DEFORMATION] If \textbf{1}, the plastic deformation is regarded. Obsoleted.
    \item[\$ITERATION\_METHOD]   \textbf{0} for Picard,  \textbf{1} for Newton-Raphson.
    \item[\$REFERENCE\_PRESSURE\_METHOD]  Index of phases. It specifies which phase pressure shall be used as reference
          in multi-phase flow calculations.
    \item[\$CALC\_OTHER\_SATURATION\_METHOD] Integer number (-1,$\cdots$, number of phases). It specifies
     the method to calculation the phase saturation in multi-phase flow.
    {
      \begin{enumerate}
      \item If it is \textbf{-1},  all phases are modified, so that they sum up to unity (Not recommended)
      \item The saturation of this phase will be computed from the other phase saturations. It is recommended
            to use the index of the least moveable phase.
      \end{enumerate}
    }
    \item[\$IRR\_NODES\_CORRECTION] no use
    \item[\$UPWINDING] Integer number to specifies the upwinding scheme for matrices of
    {
      \begin{description}
        \item[pressure ] field
        {
          \begin{enumerate}
          \item Upwinding of Gaussian integration points (scaled), additional parameter (double) factor is required.
          \item Upwinding of Gaussian integration points (unscaled), additional parameter (double) for upwinding
                factor is required.
          \end{enumerate}
        }
        \item[concentration] and temperature field
        {
          \begin{enumerate}
          \item Streamline upwinding, additional parameter (double) for upwinding factor is required.
          \item Streamline upwinding applied to advection matrices only,
               additional parameter (double) for upwinding  factor is required.
          \end{enumerate}
        }
        \item[saturation] field
        {
          \begin{enumerate}
          \item Streamline upwinding, additional parameter (double) for upwinding factor is required.
          \end{enumerate}
        }
      \end{description}
    }
    \item[\$MASS\_LUMPING] Integer number 1 or 0 to specifies whether mass lumping shall be used to reduce wiggles or not.
       This can be very useful for multi-phase flow  or transport calculations with very steep front.
    \item[\$PREDICTOR] Integer number 1 or 0 to specifies whether a linear predictor will be used to estimate solutions of
      the next time step. An additional float number $p_1\in [0,1]$ is required. If $p_1=1.0$, then full predictor is used.
    \item[\$RELAXATION] Integer number to specifies whether mass the scheme to compute the correction in non-linear iterations.
    {
      \begin{itemize}
       \item \textbf{0} No relaxation.
       \item \textbf{1} Fixed relaxation. An additional float number $p_1\in [0,1]$ is required.
       \item \textbf{2} Dynamic relaxation on the basis of field oscillations. Four additional float numbers $\in [0,1]$
       are required (recommended: 1.0, 0.25, 0.5, 0.5). Experimental option, not for common use.
       \item \textbf{3} Dynamic relaxation on the basis of single node oscillations. Two additional float numbers $\in [0,1]$
       are required (recommended: 1.0, 0.5). Experimental option, not for common use.
      \end{itemize}
     }
    \item[\$EXTRACT\_VALUES\_FROM\_PDE] Integer number 1 or 0. Only valid for the pressure field. It specifies whether
    the results of the last time step are extracted from the system of the linear equations while assembling it.
    For systems with very large vertical extend, this procedure reduces "small differences of large numbers" accuracy
    problems.
    \item[\$TIMESTEPPING] not yet.
    \item[\$TIMECOLLOCATION] Float number $\in [0,1]$. Specifies the time collocation. The time collocation value can be chosen globally for the
    regarded partial differential equation and differently for certain parts. These are distinguished by optional sub-sub-keywords
    (labelled with symbol \&) as follows:
    {
      \begin{description}
       \item [\&GLOBAL]The main time collocation (recommended value: 0.6).
       \item [\&UPWINDING] The time collocation for the upwinding scheme (recommended value: 0.0).
       \item [\&SOURCE]The time collocation for sinks and sources (recommended value: should be same as to that by
                       \textbf{\&GLOBAL}. For mass fluxes of compressible fluids, value 0.0 leads better stability )
       \item [\&COND\_BC]The time collocation for boundary condition (recommended value: 0.0).
       \item [\&OPEN\_BOUNDARY]The time collocation for free outflow boundary conditions (recommended value: 0.0).
      \end{description}
     }
   \item[\$INTEGRATION\_POINTS] Specifies how certain variables (given by sub-sub-keyword) are to be treated during
      integration. The sub-sub-keyword gives a boolean like integer to specify the method
    {
      \begin{description}
       \item [\&NAME 1] Using the values of the gravity center of the element for integration.
       \item [\&NAME 2] Using the values of Gaussian points of the element for integration.
       \item [\&SOURCE]The time collocation for sinks and sources (recommended value: should be same as to that by
                       \textbf{\&GLOBAL}. For mass fluxes of compressible fluids, value 0.0 leads better stability )
       \item [\&COND\_BC]The time collocation for boundary condition (recommended value: 0.0).
       \item [\&OPEN\_BOUNDARY]The time collocation for free outflow boundary conditions (recommended value: 0.0).
      \end{description}.
     }
     In the velocity calculation, the following sub-sub-keywords are available (Obsoleted ):
    {
      \begin{description}
       \item [\&DENSITY]
       \item [\&VISCOSITY]
       \item [\&REL\_PERM]
      \end{description}.
     }
   \item[\$MATRIX\_REBUILD] Specifies the circumstance under which the system matrices are rebuild. As setting up the system
          matrices is a substantial computational effort, choosing a conditional rebuild can significantly decrease
          computation time, though the reduced accuracy  must be considered. This method is not available for milticontinua
          calculations.

          All entries follows the same style as:

          \textbf{\&NAME method $p_1$, $p_2$, $\cdots$, $p_n$}

          where \textbf{\&NAME} specifies the identifier for the regarded matrix (depends on the FE-kernel), \textbf{method}
          is an integer value that characterizes that to evaluate the matrix rebuild, and $p_1$, $p_2$, $\cdots$, $p_n$
          are additional factors. The following \textbf{method}s are available.

    {
      \begin{itemize}
      \item \textbf{0}: Never rebuild
      \item \textbf{1, $p_1$}: Rebuild if the characteristic value diverges more than $p_1$. Criterion: $|X-X_{ref}|>p_1$
      \item \textbf{2, $p_1$}: Rebuild if the normalized characteristic value diverges more than $p_1$. Criterion: $2|X-X_{ref}|/|X+X_{ref}|>p_1$
      \item \textbf{3, $p_1$, $p_2$}: Rebuild if both criterion \textbf{1} and criterion \textbf{2} are satisfied.
      \item \textbf{11, $p_1$, $p_2$}: The rebuild is triggered by a probability approach. Probability $P$ is derived from the
                    evaluating $P=\left(|X-X_{ref}|/p_1\right)^{p_2}$
      \item \textbf{12,$p_1$, $p_2$}: The rebuild is triggered by a probability approach. Probability $P$ is derived from the
                    evaluating $P=\left(2|X-X_{ref}|/p_1/|X+X_{ref}|\right)^{p_2}$
      \item \textbf{13, $p_1$, $p_2$, $p_3$, $p_4$}: Rebuild if both criterion \textbf{11} (parameters \textbf{$p_1$} and \textbf{$p_2$}) and
                                             criterion \textbf{12} (parameters \textbf{$p_3$} and \textbf{$p_4$}) are satisfied.
      \end{itemize}
    }

    For the pressure kernel the following  \textbf{\&NAME} entries are recognized:
    {
      \begin{description}
       \item [\&ELEMENT\_MOBILITY\_CHANGE]
       \item [\&ELEMENT\_DENSITY\_CHANGE]
       \item [\&ELEMENT\_SATURATION\_CHANGE]
      \end{description}.
    }

    For the transport kernel the following  \textbf{\&NAME} entries are recognized:
    {
      \begin{description}
       \item [\&ELEMENT\_SATURATION\_CHANGE]
       \item [\&ELEMENT\_VELOCITY\_CHANGE]
      \end{description}.
    }

 %  \item[.] ...

\end{description}


Here is a sample of numerical configuration for pressure field:

\begin{center}
\shadowbox
 {
   \parbox[c]{6cm}
   {
    \#NUMERICS\_PRESSURE\\
    \$METHOD 1\\
    \$TIMECOLLOCATION\\
      \&GLOBAL 0 1.0\\
    \$UPWINDING 2 10\\
    \$MASS\_LUMPING 1\\
    \$REFERENCE\_PRESSURE\_METHOD 0\\
    \$PREDICTOR 1 1.\\
    \$RELAXATION 1 0.5\\
    \$GAUSS\_POINTS 1\\
   }
  }
\end{center}
