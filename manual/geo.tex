\section{Geometry}

%--------------------------------------------------------
\subsection{Polylines}

Data for geometric objects are contained in the RFM file. The
format for data input is as follows.

%--------------------------------------------------------
\subsubsection{Keyword}

\small
\begin{verbatim}
#POLYLINE
\end{verbatim}
\normalsize

%--------------------------------------------------------
\subsubsection{Example}

\small
\begin{verbatim}
#POLYLINE ; keyword
POLYLINE1 ; name - polyline identfier
4         ; number of polyline nodes
;           x             y             z         value
7.222222e-001 3.083144e+000 1.000000e+000 1.000000e+000
6.818783e+000 3.559226e+000 2.000000e+000 2.000000e+000
8.783069e+000 8.570615e+000 1.000000e+000 4.000000e+000
4.534392e+000 8.921412e+000 3.000000e+000 3.000000e+000
...
\end{verbatim}
\normalsize


%--------------------------------------------------------
\subsubsection{Implementation}

\small
\begin{verbatim}
read function:  int GEOReadPolylines (char *data,int found,FILE *f)
write function: void  GEOWritePolylines (FILE *f)
\end{verbatim}
\normalsize


Graphics-based data input is described in section
\ref{sec:polylines_graphics}.
