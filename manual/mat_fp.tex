\section{Fluid Properties}
\subsection{Keywords}
The following keywords have been introduced for fluid
properties:

\begin{tabular}{lp{6.5cm}}
\#FLUID\_PROPERTIES\_NEW & Announces fluid properties input,
indicates that the formulation including keywords is used.
Generally, if more than one phase is described, first the gas
phase is described, then the liquid phase.\\
\$DENSITY & Starts the input for the density case for the phase
density. The first number is the density case that will be used,
followed by the
parameters needed for that model.\\
\$VISCOSITY & Starts the input for the viscosity case. The first
number is the density case that will be used, followed by the
parameters needed for that model.\\
\$HEAT\_CAPACITY & The first number, 0, stands for constant model,
the second is the heat capacity of the fluid.\\
\$HEAT\_CONDUCTIVITY & The first number, 0, stands for constant
model,
the second is the heat conductivity of the fluid.\\
\end{tabular}

To illustrate this, two input examples are shown below.  The first
input example is an example of the old input, without
sub-keywords. The second example is an example with the new
keywords.  All new files should be written with the keywords, as
some model additions require them.  A description of the individual phase density and
viscosity cases can be found in the following sections.
\\

\begin{minipage}{12cm}
Input example 1: No sub-keywords
\hrule
\begin{verbatim}
#FLUID_PROPERTIES ; gas
 0  1.15 ; density function, parameter
 0  1.800000e-005 ; viscosity function, parameter
 3.000000e+003  1.000000e+000 ; heat capacity, heat conductivity
#FLUID_PROPERTIES ; liquid
 0  1.050000e+003 ; density function, parameter
 0  0.0012 ; viscosity function, parameter
 0.000000 ; real gas factor
 4.000000e+003  2.000000e+000 ; heat capacity, heat conductivity
\end{verbatim}
\hrule
\end{minipage}
\\
\begin{minipage}{12cm}
Input example 2: With sub-keywords
\hrule
\begin{verbatim}
#FLUID_PROPERTIES_NEW
$DENSITY
 0   1000.                      ; rho_0: reference density,
; 1   1000.  1e-4               ; rho_0,drho/dp:compressibility
; 2 3 10 1000.  0.2             ; rho_0,drho/dC:expansion coeff.
; 11 1000.  0.2  1e-4           ; rho_0,drho/dC,drho/dT:therm. expan. coeff.
; 12 1000.  1e-4                ; rho_0,drho/dT
; 13                            ; no parameters needed
; 14 1000. 10100 1e-4 285 1e-4  ; rho_0, p_0, drho_/dp, T_0, drho/dT
$VISCOSITY
 0   1e-3            ; my_0: reference viscosity
$HEAT_CAPACITY
 0   4800.           ; c_0: reference heat capacity
$HEAT_CONDUCTIVITY
 0   0.6             ; lamda_0: reference heat conductivity
\end{verbatim}
\hrule
\end{minipage}

The following sections describe the choice of models for the
keywords \$DENSITY and \$VISCOSITY.

\subsection{Fluid phase density - \$DENSITY}

\subsubsection*{Nomenclature}
\begin{tabular}{ll}
$\Conc$ & concentration\\
$\MolarMass_a$ & molar mass of air\\
$\MolarMass_w$ & molar mass of water\\
$\Pressure$ & pressure\\
$\VapourPressure$ & saturated vapor pressure\\
$\GasConstant$ & ideal gas constant\\
$\Temperature$ & temperature\\
$\FluidCompressibility$ & compressibility\\
$\ThermalExpansionCoefficient$ & thermal expansion coefficient\\
$\Phase$ & phase\\
$\Density$ & density\\
subscript 0 &  reference value\\
\end{tabular}

\subsubsection*{Cases}
It is important to state that there are two functions calculating
density, \texttt{CalcFluidDensity} and
\texttt{MATCalcFluidDensity}. The function
\texttt{MATCalcFluidDensity} does not have a variable parameter
list.  There are more density computation choices available
through this function.  Up to now, only models 100102 (TH
non-isotherm) and 10333 (unconfined flow) use this function, as
well as hydraulic and thermal processes.

Table \ref{table:density case} shows the available fluid density
calculation methods. It is important to state that for multiphase
applications, it is assumed that phase numbering starts from 0,
and that phase 0 is a gaseous phase.  Single phase calculations
are not influenced by this.  Case 0 is the constant density case.
Cases 1 and 4 compute density as a function of pressure. Cases 2,
3, and 10 compute density as a function of concentration. In case
3, more than one component can be present and case 10 uses a
reference concentration and a concentration expansion coefficient.
Case 11 computes density as a function of concentration and
temperature.Case 12 computes density as a function of temperature.
Case 13 computes density as a function of pressure and
temperature. Case 13 is limited to calculations of the gas phase
density. Additionally, one has to take care that the difference
between gas pressure and saturated vapour pressure is positive.

\begin{table}[!h]
\centering \caption{Fluid density calculation methods} $\odot:$
Cases only available with model 100102 and 10333 or process
approach.
\begin{tabular}{|c|c|}
\hline
\rule[-3mm]{0mm}{8mm}{\textbf{Case}} & \textbf{Function} \\
\hline
  0
  &
  $
  \Density^\Phase= \Density_0^\Phase
  $
  \\
  1
  &
  $
  \Density^\Phase(\Pressure)= \Density_0^\Phase(1
  +\FluidCompressibility^\Phase\Pressure^\Phase)
  $

  \\
  2
  &
  $
  \Density^\Phase(\Conc)= \Density_0^\Phase+max(C)
  $
  \\
  3
  &
  $
  \Density^\Phase(\Conc_i)= \Density_0^\Phase+max(\Density_i^\Phase(C_i), \Density_0^\Phase
  $
  \\
  4
  &
  $
  \Density^\Phase(\Pressure)= \Density_0^\Phase+\FluidCompressibility^\Phase\Pressure^\Phase
  $
  \\
  10
  &
  $
  \Density^\Phase(\Conc)=
  \Density_0^\Phase+\frac{\partial\Density^\Phase}{\partial\Conc}(\Conc-\Conc_0)
  $
  \\
  11
  &
  $
  \Density^\Phase(\Conc,\Temperature)= \Density_0^\Phase+max(\Conc, 0)\FluidCompressibility+max(\Temperature,
  0)\ThermalExpansionCoefficient
  $
  \\
 $ \odot\;12$
  &
  $
   \Density^\Phase(\Temperature)=
  \Density_0^\Phase+\frac{\partial\Density}{\partial\Temperature}(\Temperature-\Temperature_0)
  $
  \\
  $\odot\;13$
  &
  $
   \Density^g(\Pressure^g,\Temperature)=
  \frac{\MolarMass_a}{\GasConstant\Temperature}\Pressure^g+\frac{(\MolarMass_w -
  \MolarMass_a)}{\GasConstant\Temperature}\VapourPressure(\Temperature)
  $
   \\
  $\odot\;14$
  &
  $
   \Density^\Phase(\Pressure,\Temperature)=
  \Density_0^\Phase+\frac{\partial\Density}{\partial\Temperature}(\Temperature-\Temperature_0)
  +\frac{\partial\Density}{\partial\Pressure}(\Pressure-\Pressure_0)
  $
\\
\hline
\end{tabular}
\label{table:density case}
\end{table}


\subsubsection*{Data Input}
The table below summarizes the input for the keyword \$DENSITY.
One of the cases with corresponding values can be chosen.

\begin{table}[h]
\centering
\caption{Parameters for fluid density calculations}
\begin{tabular}{|l|lllll|}
\hline
\rule[-3mm]{0mm}{8mm}\textbf{Case} & \multicolumn{5}{|c|}{\textbf{Parameters}}\\
\hline
0 & $\Density_0$ [$kg/m^3$]&&&&\\
1 & $\Density_0$ [$kg/m^3$]& $\FluidCompressibility [1/Pa]$ & $\Pressure_0 [Pa]$&&\\
2 & $\Density_0$ [$kg/m^3$] & &&&\\
3 & $\Density_0$ [$kg/m^3$] & $\p\Density/\p\Conc [-] $&&&\\
4 & $\Density_0$ [$kg/m^3$]& $\FluidCompressibility [kg/(m^3 Pa)]$ &&&\\
10 & $\Density_0$ [$kg/m^3$] & $\p\Density/\p\Conc [-] $&&&\\
11 & $\Density_0$[$kg/m^3$] & $\FluidCompressibility [1/Pa]$& $\p\Density/\p\Conc [1/mol] $&&\\
12 & $\Density_0$ [$kg/m^3$] & $\ThermalExpansionCoefficient [1/K]$&&&\\
13 & &&&&\\
14 & $\Density_0$[$kg/m^3$] & $\Pressure_0 [Pa]$ & $\FluidCompressibility [1/Pa]$& $\Temperature_0 [K]$& $\ThermalExpansionCoefficient [1/K]$\\
\hline
\end{tabular}
\label{table:density parameters}
\end{table}


\subsubsection*{Data access}
The implementation of fluid density functions into the Rockflow
code is as follows.

\bigskip
\begin{table}[htb]
\centering
\begin{tabular}{|lll|}
\hline
\rule[-3mm]{0mm}{8mm}{\textbf{Description}}& \textbf{Function} & \textbf{RF Object}\\
\hline
Access to fluid density in kernel& rho=MATCalcFluidDensity()&MPC/ENT\\
\rule[0mm]{0mm}{8mm}Calculation of density & \texttt{MATCalcFluidDensity} & MAT\\
\rule[0mm]{0mm}{8mm}Overwriting & \texttt{MATGetNodeIndexTemperature=} &MOD\\
&\texttt{MODGetNodeIndexTemperature\_XX}&\\
\rule[0mm]{0mm}{8mm}Definition of model function for &&\\
access to node index &\texttt{MODGetNodeIndexTemperature\_XX}& MOD\\
\hline
\end{tabular}
\caption{Fluid density functions in Rockflow}
\end{table}
\bigskip


%\pagebreak
\subsection{Fluid Viscosity}
\subsubsection*{Nomenclature}
\begin{tabular}{ll}
$\Conc$ & concentration\\
$\Pressure$ & pressure\\
$\Temperature$ & temperature\\
$\Phase$ & phase\\
$\Viscosity$ & viscosity\\
subscript 0 &  reference value\\
subscript avg & average value\\
\end{tabular}

\subsubsection*{Cases}

Fluid viscosity can be computed in one of the ways illustrated in
Table \ref{Fluid viscosity methods}.  As for density, it is
important to note that  viscosity can be calculated with two
different functions in the code, \texttt{CalcFluidViscosity} and
\texttt{MATCalcFluidViscosity}.  \texttt{MATCalcFluidViscosity} is
only accessed by models 100102 and 10333 so far.  Case 0 is the
constant viscosity case. Cases 1 and 2 compute viscosity as a
function of pressure. Case 1 uses a curve, whereas case 2 uses a
coefficient for the variation of viscosity with pressure.  Case 7
computes the viscosity after Reichenberg (1971) as shown in Reid
et al. (1988), p. 420 as a function of pressure and temperature.
Equation \ref{eqn: Viscosity Reichenberg} complete the equation
shown in Table \ref{Fluid viscosity methods} for this case.
\\
\hrule
%\\
Additional Equations for Case 7
\begin{eqnarray}
\centering
\Viscosity_0 &=& 26.69 \sqrt{28.96} \frac{\sqrt{\Temperature}}{3.7^2} 1.0 10^{-6} 0.1\nonumber\\
A &=& \frac{1.9824 10^{-3}}{\frac{\Temperature}{126.2}}\exp(5.2683(\frac{\Temperature}{126.2})^{-0.5767})\nonumber\\
B &=& A ( 1.6552 \frac{\Temperature}{126.2} - 1.2760)\nonumber\\
C &=& \frac{0.1319}{\frac{\Temperature}{126.2}} \exp(3.7035(\frac{\Temperature}{126.2})^{-79.8678})\nonumber\\
D &=& \frac{2.9496}{\frac{\Temperature}{126.2}} \exp(2.9190(\frac{\Temperature}{126.2})^{-16.6169})\nonumber\\
\label{eqn: Viscosity Reichenberg}
\end{eqnarray}\nopagebreak
\hrule

Case 8 computes viscosity as a function of temperature following
Yaws et al. (1976) as shown in Reid et al. (1988), p. 441/455.
Case 9 also computes viscosity as a function of temperature
following a logarithmical distribution.  Case 10 computes
viscosity as a function of concentration and temperature.

\bigskip
\begin{table}[!h]
\centering \caption{Fluid viscosity calculation
methods}
$\odot:$ Cases only available with model 100102 and 10333
\begin{tabular}{|c|p{10cm}|}
\hline
\rule[-3mm]{0mm}{8mm}{\textbf{Case}} & \textbf{Function} \\
\hline
  0  &  $\Viscosity^\Phase(\Pressure)= \Viscosity_0^\Phase$
  \\[1mm]
  1  &  $\Viscosity^\Phase(\Pressure)= GetCurveValue (\mbox{get\_fp\_curve(fp)},
  0, \Pressure_{\mbox{avg}}, gueltig)$
  \\[1mm]
  2  &  $\Viscosity^\Phase(\Pressure)= \Viscosity_0^\Phase+
  \Pressure_{avg}\cdot\frac{d\Viscosity}{d\Pressure}$
  \\[1mm]
 $\odot$ 7 & $\Viscosity^\Phase(\Pressure,\Temperature)= \Viscosity_0 [1
+ \frac{A(\frac{\Pressure}{33.9
10^4})^{1.5}}{B(\frac{\Pressure}{33.9 10^4}) +
\frac{1}{C(\frac{\Pressure}{33.9 10^4})^D}}]$
 \\
$\odot$ 8 & $\Viscosity^\Phase(\Temperature) = 10^{-3} \exp(-2.471 10^1+\frac{4.209 10^3}{\Temperature}
+4.527 10^{-2}\Temperature-3.376 10^{-5}\Temperature^2)$
\\
$\odot$ 9 & $\Viscosity^\Phase(\Temperature) = 2.285\cdot10^{-5} + 1.01 \cdot10^{-3}\log{T}$
\\
  10 & $\Viscosity^\Phase(\Conc,\Temperature) =
  \frac{\Viscosity}{f1 + f2}$ $ f1= f(\Conc), f2=f(\Temperature)$
  \\
  \hline
\end{tabular}
 \label{Fluid viscosity
methods}
\end{table}

\subsubsection*{Data Input}

\begin{table}[h]
\begin{tabular}{|l|lll|}
\hline
\rule[-3mm]{0mm}{8mm}\textbf{Case} & \multicolumn{3}{|c|}{\textbf{Parameters}}\\
\hline
0 & $\Viscosity_0$ [$Pa\,s]$&&\\
1 & $\Viscosity_0$ [$Pa\,s]$& curve number &\\
2 & $\Viscosity_0$ [$Pa\,s]$ & $\p\Viscosity/\p\Pressure \;[Pa^{-1}s^{-1}]$&\\
10 & $\Viscosity_0$ [$Pa\,s]$ & $\p\Viscosity/\p\Conc \;[1/mol] $&$\p\Viscosity/\p\Pressure \;[Pa^{-1}s^{-1}]$\\
\hline
\end{tabular}
\label{table:density parameters}\caption{Parameters for viscosity
calculations}
\end{table}


\subsubsection*{Data Access}
To obtain fluid viscosity results, the structure shown in Table
\ref{viscosity table} is implemented in the program.
\begin{table}[htb]
\centering
\begin{tabular}{|lll|}
\hline
\rule[-3mm]{0mm}{8mm}{\textbf{Description}}& \textbf{Function} & \textbf{RF Object}\\
\hline
FE-Kernel needs fluid viscosity&& \\
for finite element matrix&&\\
calculation & \texttt{GetFluidViscosity} & MPL\\
&&\\
Calculation of viscosity using &&\\
one of the cases &&\\
as described above & \texttt{CalcFluidViscosity} & MAT\\
&&\\
%---
Specification of model&& \\
specific parameters for & \texttt{GetFluidViscosity =} & \\
fluid viscosity calculation & \texttt{THMGetFluidViscosity} & MOD\\
\hline
\end{tabular}
\caption{Fluid viscosity functions in Rockflow}\label{viscosity
table}
\end{table}
