\section{Processes}
\label{sec:pcs}

\begin{tabular*}{5.35cm}{|p{2.5cm}|p{2cm}|} \hline
Object acronym & PCS \\
C++ class  & CRFProcess \\
Source files   & rf\_pcs.h/cpp \\
\hline
File extension & *.pcs \\
Object keyword &  {\texttt{\#PROCESS}} \\
\hline
\end{tabular*}

\bigskip
The control--keyword {\texttt{\#PROCESS}} can be used to specify
physical/bio/chemical processes.

%-------------------------------------------------------------------------------
\subsection{\bf\texttt{\#PROCESS}}

\begin{verbatim}
#PROCESS
 $PCS_TYPE
  LIQUID_FLOW       // H process (incompressible flow)
  GROUNDWATER_FLOW  // H process (incompressible flow)
  RIVER_FLOW        // H process (incompressible flow)
  RICHARDS_FLOW     // H process (incompressible flow)
  OVERLAND_FLOW     // H process (incompressible flow)
  GAS_FLOW          // H process (compressible flow)
  TWO_PHASE_FLOW    // H2 process (incompressible/compressible flow)
  MULTI_PHASE_FLOW  // H2 process (Non-isothermal two-phase flow)
  COMPONENTAL_FLOW  // H2 process (incompressible/compressible flow)
  HEAT_TRANSPORT    // T process (single/multi-phase flow)
  DEFORMATION       // M process (single/multi-phase flow)
  MASS_TRANSPORT    // C process (single/multi-phase flow)
  FLUID_MOMENTUM
 $CPL_TYPE
  PARTITIONED       // default
  MONOLITHIC
 $NUM_TYPE
  FEM               // Finite-Element-Method default
  FDM               // Finite-Difference-Method (for line elements)
  NEW               // New FEM assemblier
  EXCAVATION        // Works with DEFORMATION process up to now
 $PRIMARY_VARIABLE  // to specify the primary variable name
  HEAD
 $MEMORY_TYPE       // Works with $NUM_TYPE=NEW
  1                 //0  Do not save local matrices and vectors in RAM; 1, Do
 $ELEMENT_MATRIX_OUTPUT  // Element output
   1                //0  Do not output local matrices and vectors; 1, Do
 $BOUNDARY_CONDITION_OUTPUT  // Given: output the boundary condition and source term nodes together with their values  
 $DEACTIVATED_SUBDOMAIN  // Select the elements of a subdomain that not required by this process
   1                     // Number of sub-domain
   ...                   // List of the indices of the selected subdomain. The indices are given in element data of mesh file.
                         //The indices are given in element data of mesh file.
   ...
$WRITE_READ_SOURCE_OR_NEUMANN_RHS
  // To write soure term or Neumann
  // BC after face or domain intergration
  // Use this keyword will save time if one exemple is run more frenquently.
  // e.g. First run: 1. Next: 2. Then you skip finding nodes on the specfied polylines,
  // faces or domains of source or Neumann BC definition and skip the corresponding
  // numerical integation as well.
   0     // Do nothing
   1     // Write
   2     // Read what has already written
$TIME_UNIT
  //HOUR, DAY, MONTH, YEAR
$MEDIUM_TYPE
  CONTINUUM  0.95
   ...
\end{verbatim}
%%%%%%WW
The axisymmetrical problems can be solved by give keyword
\textit{\mbox{\$AXISYMMETRY}} in .msh file (c.f. section
\ref{sec:msh}, \ref{sec:st}, \ref{sec:bc} and \ref{sec:out}).

The gravity term for hydraulic analysis is assumed to be always from
 the vertical direction. %%%%%%%%%WW


\subsubsection{Remarks}
A sub-keyword \mbox{\$RELOAD} is also available to control output/input for test purpose. If value of \mbox{\$RELOAD} is 1, i.e.,
\begin{verbatim}
.
.
.
$RELOAD
 1
\end{verbatim}
Files named by process\_name +"\_"+pcs\_type\_name+"\_primary\_value.asc" will be produced after simulation being finished.
They contain the results of primary viarables of the last time step. For deformation analysis, this value will make
  all Gauss point stresses stored in a binary file named as  process\_name +".sts".  Such initial value will
   be read with \mbox{\$RELOAD=2} if the simulation is restarted. Note, this is a test subkeyword,
    we do not warrant or guarantee the stability for its usage in current version.



\subsection{Node values}
\label{sec:nod_values}

\begin{tabular*}{12.773cm}{|p{3.cm}|p{1.5cm}|p{7cm}|} \hline
Parameter          & Value & Meaning \\ \hline \hline
%
\texttt{PRESSUREx}        & phase & Pressure of fluid phase x \\
\texttt{DISPLACEMENTx\_X} & phase & Displacement of solid phase x \\
\texttt{DISPLACEMENTx\_Y} & phase & ... \\
\texttt{DISPLACEMENTx\_Z} & phase & ... \\
\texttt{TEMPERATUREx}     & phase & Temperature \\
\texttt{CONCENTRATIONx}   & component & Mass concentration on component x \\
\hline
\end{tabular*}

%-------------------------------------------------------------------------------
\subsection{Coupling of processes}
\subsubsection{Partitioning}

Partitioning is the default scheme. Each specified process is
executed. The order of process execution (i.e. flow, mass transport,
heat transport and deformation) depends on the order in the PCS input file.

\Examples{
%-------------------------------------------------------------------------------
\subsection{Examples}
%-------------------------------------------------------------------------------
\subsubsection{(confined) Groundwater flow}
\begin{verbatim}
benchmarks: h_*.*
#PROCESS
 $PCS_TYPE
  LIQUID_FLOW
#STOP
\end{verbatim}
%-------------------------------------------------------------------------------
\subsubsection{(unconfined) Groundwater flow}
\begin{verbatim}
benchmarks: h_uc_*.* #PROCESS
 $PCS_TYPE
  GROUNDWATER_FLOW
#STOP
\end{verbatim}
%-------------------------------------------------------------------------------
\subsubsection{Gas flow}
\begin{verbatim}
benchmarks: h_gas_*.* #PROCESS
 $PCS_TYPE
  GAS_FLOW
#STOP
\end{verbatim}
%-------------------------------------------------------------------------------
\subsubsection{Two phase flow (partitioned)}
\begin{verbatim}
benchmarks: h2_*.*
#PROCESS
 $PCS_TYPE
  TWO_PHASE_FLOW
#PROCESS
 $PCS_TYPE
  TWO_PHASE_FLOW
#STOP
\end{verbatim}
%-------------------------------------------------------------------------------
\subsubsection{Multi-phase flow (monolithic)}
\begin{verbatim}
benchmarks: h2_*.*
#PROCESS
 $PCS_TYPE
  MULTI_PHASE_FLOW
#STOP
\end{verbatim}
This process has two primary variables, capillary pressure and gaseous pressure. In the code,
 capillary pressure gaseous pressure are denoted as \mbox{PRESSURE1} and \mbox{PRESSURE1}, respectively.
%-------------------------------------------------------------------------------
\subsubsection{Heat transport}
\begin{verbatim}
benchmarks: ht_*.*
#PROCESS
 $PCS_TYPE
  LIQUID_FLOW
#PROCESS
 $PCS_TYPE
  HEAT_TRANSPORT
#STOP
\end{verbatim}
%-------------------------------------------------------------------------------
\subsubsection{Deformation}
\label{sec:pcs_dm}
\begin{verbatim}
benchmarks: m_*.*
#PROCESS
 $PCS_TYPE
  DEFORMATION
#STOP
\end{verbatim}
\textbf{Note: For the monolithic approach of HM coupled problem,
 the process name for boundary conditions and source terms
  must be DEFORMATION\_FLOW} %%WW

For the case of excavation simulation, the deformation process data is given as
\begin{verbatim}
#PROCESS
 $PCS_TYPE
  DEFORMATION
  $NUM_TYPE
  EXCAVATION        // Works with DEFORMATION process up to now
#STOP
\end{verbatim}
Consequently, user has to specify the elements in the domain to be excavated.
This is can be done in .st file (see Section\ref{sec:st_excav}). The current version
  can deal with linear elastic deformation of excavation.

%-------------------------------------------------------------------------------
\subsubsection{Non-isothermal two-phase flow (partitioned)}
\begin{verbatim}
benchmarks: th2_*.*
#PROCESS
 $PCS_TYPE
  COMPONENTAL_FLOW
#PROCESS
 $PCS_TYPE
  COMPONENTAL_FLOW
#PROCESS
 $PCS_TYPE
  HEAT_TRANSPORT
#STOP
\end{verbatim}
%-------------------------------------------------------------------------------
\subsubsection{Mass transport}
\begin{verbatim}
benchmarks: c_*.*
#PROCESS
 $PCS_TYPE
  MASS_TRANSPORT
#STOP
\end{verbatim}
%-------------------------------------------------------------------------------
\subsubsection{River flow}
\begin{verbatim}
GeoSys-PCS: Processes ----- 3.9.09_6OK4_SV1
#PROCESS
 $PCS_TYPE
  RIVER_FLOW
 $NUM_TYPE
  FDM
#STOP
\end{verbatim}

%-------------------------------------------------------------------------------
\subsubsection{Unsaturated flow}
\begin{verbatim}
GeoSys-PCS: Processes
 $PCS_TYPE
  RICHARDS_FLOW
#STOP
\end{verbatim}

%-------------------------------------------------------------------------------
\subsubsection{Unsaturated flow with the dual porosity model} %%WW
\begin{verbatim}
GeoSys-PCS: Processes
 $PCS_TYPE
  RICHARDS_FLOW
$MEDIUM_TYPE
  CONTINUUM  0.95
#STOP
\end{verbatim}
%%WW
Keyword \mbox{\$MEDIUM\_TYPE} indicates that the process is for a unsaturated flow process with the dual porosity model.
While subkeyword \mbox{CONTINUUM} specifies the volumetric factor for matrix $w\_m$, which is 0.95 for this example.

%-------------------------------------------------------------------------------
\subsubsection{Monolithic schemes}
Is more than one PCS type specified, than these processes are treated in a monolithic coupling scheme
(see Example \ref{sec:ex_monolithic})
%-------------------------------------------------------------------------------
\subsubsection{Consolidation (monolithic)}
\label{sec:ex_monolithic}
\begin{verbatim}
benchmarks: hm_*.*
#PROCESS
 $PCS_TYPE
  LIQUID_FLOW
  DEFORMATION
 $CPL_TYPE
  MONOLITHIC
#STOP
\end{verbatim}

} % Examples


%\LastModified{OK August 13, 2004}
\LastModified{PCH \today}
\newpage
