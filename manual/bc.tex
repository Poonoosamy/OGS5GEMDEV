\section{Boundary Conditions}
\subsection{Keyword Description}

The following keywords are available to specify boundary conditions for fluid flow,
heat transport and mass transport.

\begin{verse}
 \#BOUNDARY\_CONDITIONS\_PRESSURE      \\
 \#BOUNDARY\_CONDITIONS\_SATURATION    \\
 \#BOUNDARY\_CONDITIONS\_TEMPERATURE   \\
 \#BOUNDARY\_CONDITIONS\_CONCENTRATION \\
 \#BOUNDARY\_CONDITIONS\_SORBED\_CONCENTRATION  \\
 \#BOUNDARY\_CONDITIONS\_SOLUTE\_CONCENTRATION  \\
 \#BOUNDARY\_CONDITIONS\_DISPLACEMENT\_X        \\
 \#BOUNDARY\_CONDITIONS\_DISPLACEMENT\_Y        \\
 \#BOUNDARY\_CONDITIONS\_DISPLACEMENT\_Z        \\
 \#BOUNDARY\_CONDITIONS\_FREE\_OUTFLOW          \\
\end{verse}

By repeatedly use of the keywords, boundary conditions for corresponding phases and components are specified.


\begin{tabular}{|l|p{2.75cm}|c|p{8.5cm}|}
  \hline
  Parameter &    RF Variable \quad      &  Values  & Meaning           \\
  \hline
%
  \hline
%
  N1        & {\footnotesize type} &    0     & Individual nodes by node number  \\
            &                      &    1     & Individual nodes by coordinates   \\
            &                      &    2     & Linear distribution between two nodes by coordinates   \\
            &                      &    3     & Linear distribution between two nodes by node number   \\
            &                      &    4     & Constant value at plain given by 3 points   \\
            &                      &    5     & Linear distribution along polygon   \\
            &                      &    6     & Constant value at plain given by polygon node numbers   \\
            &                      &    7     & Hydrostatic distribution at plain given by 3 points   \\
            &                      &    8     & Constant value at plain given by polygon coordinates   \\
            &                      &    9     & Bilinear distribution within rectangle by coordinates   \\
            &                      &    10    & Bilinear distribution at plain given by coordinates   \\
            &                      &    11/0  & Individual nodes by node number (eps)   \\
            &                      &    12/5  & Linear distribution along polygon / coordinates (eps)   \\
            &                      &    13    & Linear distribution along polygon / nodes (eps)   \\
            &                      &    14    & Hydrostatic distribution in polygon / coordinates (eps)   \\
            &                      &    15    & Hydrostatic distribution in polygon / nodes (eps)   \\
            &                      &    16    & Bilinear distribution within rectangle by coordinates   \\
            &                      &    17    & Hydrostatic distribution in rectangle / coordinates (eps)   \\
%
  \hline
%
  N2        & {\footnotesize node} &    0     & Overwrite    \\
            &                      &    1     & Superimpose  \\
%
  \hline
%
  N3        & {\footnotesize curve}&    int   & Curve number    \\
  \hline   \hline
  if N1=0   & {\footnotesize begin\_node}    &      & Node number                       \\
            & {\footnotesize values[0]}      &      & Nodal value        \hfill (AH)    \\
%
  \hline
%
  if N1=1   & {\footnotesize x[0] y[0] z[0]} &      & Node coordinates                  \\
            & {\footnotesize radius[0]}      &      & Radius ($\>$0.0)                     \\
            & {\footnotesize values[0]}      &      & Nodal value        \hfill (AH)    \\
%
  \hline
%
  if N1=2   & {\footnotesize x[0] y[0] z[0]} &      & First node coordinates            \\
            & {\footnotesize x[1] y[1] z[1]} &      & First nodal value                 \\
            & {\footnotesize radius[0]}      &      & "Radius", epsilon                 \\
            & {\footnotesize values[0]}      &      & Second node coordinates           \\
            & {\footnotesize values[1]}      &      & Second nodal value \hfill RFE-Data incorrect (AH)    \\
%
  \hline
%
\end{tabular}

\newpage
{\it Continuation from the previous page}

\begin{tabular}{|p{1.5cm}|p{2.75cm}|c|p{8.5cm}|}
%
  \hline
%
  if N1=3   & {\footnotesize begin\_node}    &      & First node                        \\
            & {\footnotesize end\_node}      &      & Second node                       \\
            & {\footnotesize step\_nodes}    &      & Increment                         \\
            & {\footnotesize values[0]}      &      & First nodal value                 \\
            & {\footnotesize value[1]}       &      & Second nodal value  \hfill  (RFE-Data has to be adjusted) (AH) \\
%
  \hline
%
  if N1=4   & {\footnotesize x[0] y[0] z[0]} & \PlaceHolder{Values} & First node coordinates            \\
            & {\footnotesize x[1] y[1] z[1]} &      & Second node coordinates           \\
            & {\footnotesize x[2] y[2] z[2]} &      & Third node coordinates            \\
            & {\footnotesize radius[0]}      &      & "Radius", epsilon                 \\
            & {\footnotesize values[0]}      &      & Areal value  \hfill          (AH) \\
%
  \hline
%
  if N1=5   & {\footnotesize count\_of\_points}        & & Number of polygon points       \\
            & {\footnotesize nodes[count\_of\_points]} & & Polygon nodes                  \\
            & {\footnotesize values[count\_of\_points]}& & Nodal values   \hfill (AH/CT)  \\
%
  \hline
%
%
  if N1=6   & {\footnotesize count\_of\_points}        & & Number of surface polygon points \\
            & {\footnotesize nodes[count\_of\_points]} & & Surface polygon nodes            \\
            & {\footnotesize radius [0]}               & & "Radius", epsilon \hfill (does not work properly!)  \\
            & {\footnotesize value[0]}                 & & Areal value \hfill (RFE-Data not correct)  (AH) \\
%
  \hline
%
%
  if N1=7   & {\footnotesize x[0] y[0] z[0]}           & & First node coordinates           \\
            & {\footnotesize x[1] y[1] z[1]}           & & Second node coordinates          \\
            & {\footnotesize x[2] y[2] z[2]}           & & Third node coordinates           \\
            & {\footnotesize z[3]}                     & & Reference elevation              \\
            & {\footnotesize values[0]}                & & Reference value \hfill  (not yet tested)  (AH) \\
%
  \hline
%
%
  if N1=8   & {\footnotesize count\_of\_points}          & & Number of polygon points         \\
            & {\footnotesize x[0] y[0] z[0]}           & & Polygon node coordinates         \\
            & {\footnotesize ...}                      & & ...                              \\
            & {\footnotesize x[count\_of\_points-1]}     & &                                  \\
            & {\footnotesize y[count\_of\_points-1]}     & &                                  \\
            & {\footnotesize z[count\_of\_points-1]}     & &                                  \\
            & {\footnotesize radius[0]}                & & "Radius", epsilon  \hfill (does not work properly!)              \\
            & {\footnotesize values[0]}                & & Value \hfill  (RFE-Data not correct) (AH)\\
%
  \hline
%
  if N1=9   & {\footnotesize x[0] y[0] z[0]} &      & Rectangle node coordinates  \\
            & {\footnotesize values[0]}      &      & Value                                 \\
            & {\footnotesize x[1] y[1] z[1]} &      & ...                                   \\
            & {\footnotesize values[1]}      &      &                                       \\
            & {\footnotesize x[2] y[2] z[2]} &      &                                       \\
            & {\footnotesize values[2]}      &      &                                       \\
            & {\footnotesize x[3] y[3] z[3]} &      &                                       \\
            & {\footnotesize values[3]}      &      & \hfill (does not work properly!) (RK) \\
%
  \hline
%
%
  if N1=10  & {\footnotesize x[0] y[0] z[0]} &      & Node coordinates                      \\
            & {\footnotesize values[0]}      &      & Value                                 \\
            & {\footnotesize x[1] y[1] z[1]} &      & ...                                   \\
            & {\footnotesize values[1]}      &      &                                       \\
            & {\footnotesize x[2] y[2] z[2]} &      &                                       \\
            & {\footnotesize values[2]}      &      &                                       \\
            & {\footnotesize x[3] y[3] z[3]} &      &                                       \\
            & {\footnotesize values[3]}      &      & \hfill (does not work properly!) (RK) \\
%
  \hline
%
  if N1=11  & {\footnotesize x[0] y[0] z[0]} &      &                        \\
            & {\footnotesize values[0]}      &      &                        \\
            & {\footnotesize epsilon}        &      &         \hfill (CT)    \\
%
  \hline
%
\end{tabular}

\newpage
{\it Continuation from the previous page}

\begin{tabular}{|p{1.5cm}|p{2.75cm}|c|p{8.5cm}|}
%
  \hline
%
  if N1=12  & {\footnotesize count\_of\_points}         &      &                        \\
            & {\footnotesize x[count\_of\_points]}      &      &                        \\
            & {\footnotesize y[count\_of\_points]}      &      &                        \\
            & {\footnotesize z[count\_of\_points]}      &      &                        \\
            & {\footnotesize values[count\_of\_points]} &      &                        \\
            & {\footnotesize epsilon}                 &      &       \hfill (CT)      \\
%
  \hline
%
  if N1=13  & {\footnotesize count\_of\_points}         &  \PlaceHolder{Values} &                        \\
            & {\footnotesize nodes[count\_of\_points]}  &      &                        \\
            & {\footnotesize values[count\_of\_points]} &      &                        \\
            & {\footnotesize epsilon}                   &      &      \hfill (CT)       \\
%
  \hline
%
  if N1=14  & {\footnotesize values[0]}            &      &                        \\
            & {\footnotesize values[1]}            &      &                        \\
            & {\footnotesize phase}                &      &                        \\
            & {\footnotesize count\_of\_points}    &      &                        \\
            & {\footnotesize x[count\_of\_points]} &      &                        \\
            & {\footnotesize y[count\_of\_points]} &      &                        \\
            & {\footnotesize z[count\_of\_points]} &      &                        \\
            & {\footnotesize epsilon}              &      & \hfill (not yet tested) (CT)  \\
%
  \hline
%
  if N1=15  & {\footnotesize values[0]}            &      &                        \\
            & {\footnotesize values[1]}            &      &                        \\
            & {\footnotesize phase}                &      &                        \\
            & {\footnotesize count\_of\_points}    &      &                        \\
            & {\footnotesize nodes[count\_of\_points]} &  &                        \\
            & {\footnotesize epsilon}              &      & \hfill (not yet tested) (CT)  \\
%
  \hline
%
  if N1=16  & {\footnotesize x[0] y[0] z[0]}     &      &                        \\
            & {\footnotesize values[0]}          &      &                        \\
            & {\footnotesize x[1] y[1] z[1]}     &      &                        \\
            & {\footnotesize values[1]}          &      &                        \\
            & {\footnotesize x[2] y[2] z[2]}     &      &                        \\
            & {\footnotesize values[2]}          &      &                        \\
            & {\footnotesize x[3] y[3] z[3]}     &      &                        \\
            & {\footnotesize values[3]}          &      &                        \\
            & {\footnotesize epsilon}            &      & \hfill  (CT)           \\
%
  \hline
%

  if N1=17  & {\footnotesize values[0]}          &      &                        \\
            & {\footnotesize values[1]}          &      &                        \\
            & {\footnotesize x[0] y[0] z[0]}     &      &                        \\
            & {\footnotesize x[1] y[1] z[1]}     &      &                        \\
            & {\footnotesize x[2] y[2] z[2]}     &      &                        \\
            & {\footnotesize x[3] y[3] z[3]}     &      &                        \\
            & {\footnotesize epsilon}            &      & \hfill (not yet tested) (CT)  \\
%
  \hline
%
\end{tabular}


\subsection*{Specific boundary conditions}

\#BOUNDARY\_CONDITIONS\_FREE\_OUTFLOW

This keyword is intended for open boundaries under multiphase flow conditions.

For a system with two or more phases it is necessary to specify BCs for the saturations on each boundary
which is regarded "open" by specifying a pressure BC. If the user knows in advance that this will be a free
outflow boundary, the free outflow BC can replace the saturation BCs and the saturations at the boundary will
adjust freely in the simulation due to the conditions inside the system.

If the simplified Richards solution is used, this BC specifies the free outflow of fluid over a boundary with
no pressure BC specified. In the iterative process the pressure will be set to zero (which is the reference
pressure for the transition between full and partially saturated conditions) if thereby an outflow is
enabled, otherwise the boundary is regarded as impermeable.


\LastModified{MK June 13, 2003}
