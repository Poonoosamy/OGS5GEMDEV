
                                                                                                                                                                                                                                                                                                                                                                                                                                                                                                                                                                                                                                                                                                                                                                                                                                                                                                                                                                                                                                                                                                                                                                                                                                                                                                                                                                                                                                                                                                                                                                                                                                                                                                                                                                                                                                                         \appendix                                                                                                \chapter{Fluid property equations}
%#######################################################################
\section{Viscosity}
\subsection{Carbon dioxide \cite{FenWakVes:98}}

Equation \eqref{eq-three-term-visc} 
\begin{equation}
\eta (\rho,T) = \eta_{0} (T) + \Delta \eta (\rho,T) + \Delta \eta_c (\rho,T)
	\label{eq-three-term-visc-A}
\end{equation}
in section \ref{sec-viscosity}  will be explained in the following. The influence of the term $\Delta \eta_c (\rho,T)$ is very low so that it can be neglected ($\Delta \eta_c (\rho,T) = \unit[0]{}$). For the zero-density viscosity $\eta_0(T)$ it can be written
\begin{equation}
	\eta_0(T) = \frac{1.00697\,T^\frac{1}{2}}{\mathfrak{S}_{\eta}^*(T^*)}.
	\label{eq-visc-co2-a}
\end{equation}
For $\theta_{\eta}^*(T^*)$ can be set
\begin{equation}
	\ln\mathfrak{S}_{\eta}^*(T^*) = \sum_{i=0}^4 a_i(\ln\,T^*)^i
	\label{eq-visc-co2-b}
\end{equation}
and for the reduced temperature $T^*$ 
\begin{equation}
	T^* = \frac{kT}{\varepsilon}
	\label{eq-visc-co2-c}
\end{equation}
with $\varepsilon/k = \unit[251.196]{K}$. Values for the coefficient $a_i$ of \eqref{eq-visc-co2-b} are listed in Tab.~\ref{tab-visc-co2-ai}.
\begin{table}[H]
  \caption{\label{tab-visc-co2-ai}Values for the coefficient $a_i$ in \eqref{eq-visc-co2-b}.}
  \begin{center}
  \begin{tabular}{lr@{.}l}
  \toprule
  $i$\hspace{1cm}		& \multicolumn{2}{c}{$a_i$}\\
  \midrule
  0	\hspace{1cm}	& $0$&$235\,156$\\
  1	\hspace{1cm}	& $-0$&$491\,266$\\  
  2 \hspace{1cm}  	& $5$&$211\,155\cdot 10^-2$\\ 
  3	\hspace{1cm}	& $5$&$347\,906\cdot 10^-2$\\  
  4 \hspace{1cm}  	& $-1$&$537\,102\cdot 10^-2$\\
  \bottomrule
 \end{tabular}
 \end{center}
\end{table}

Instead of $\Delta \eta (\rho,T)$ it can be written
\begin{equation}
	\Delta \eta (\rho,T) = \sum_{i=1}^n b_i(T)\rho^i,
	\label{eq-visc-co2-d}
\end{equation}
with 
\begin{equation}
	b_i = \sum_{j=1}^m \frac{d_{ij}}{T^{*(j-1)}}.
	\label{eq-visc-co2-e}
\end{equation}
For the coefficient $d_{ij}$ the values are given in Tab.~\ref{tab-visc-co2-dij}.  
\begin{table}[H]
  \caption{\label{tab-visc-co2-dij}Values for the coefficient $d_{ij}$ in \eqref{eq-visc-co2-e}.}
  \begin{center}
  \begin{tabular}{lr@{.}l}
  \toprule
  $d_{ij}$\hspace{1cm}		& \multicolumn{2}{c}{value}\\
  \midrule
  $d_{11}$	\hspace{1cm}	& $0$&$407\,111\,9\cdot 10^-2$\\
  $d_{21}$	\hspace{1cm}	& $0$&$719\,803\,7\cdot 10^-4$\\  
  $d_{64}$ \hspace{1cm}  	& $0$&$241\,169\,7\cdot 10^-16$\\ 
  $d_{81}$	\hspace{1cm}	& $0$&$297\,107\,2\cdot 10^-22$\\  
  $d_{82}$ \hspace{1cm}  	& $-0$&$162\,788\,8\cdot 10^-22$\\
  \bottomrule
 \end{tabular}
 \end{center}
\end{table}
%################################################################
\subsection{Nitrogen  \cite{SteKraLae:87}}

For nitrogen exists the equation 
\begin{equation}
\eta (\rho,T) = \eta_{0} (T) + \eta_{ex} (\rho)
	\label{eq-visc-n2-a}
\end{equation}
which can be separated in two terms. It can be written for 
\begin{equation}
	\eta_{0} (T) = \frac{5}{16} \left[\frac{MkT}{\pi N_A}\right]^{0.5}(\sigma^2 \Omega(T^*))^{-1} 
	\label{eq-visc-n2-b}
\end{equation}
with
\begin{equation}
	\ln \Omega(T^*) = \sum_{i=0}^4 A_i(\ln T^*)^i 
	\label{eq-visc-n2-c}
\end{equation}
and 
\begin{equation}
	T^* = \frac{Tk}{\epsilon}.
	\label{eq-visc-n2-d}
\end{equation}
The adjustable parameters $\epsilon/k$ and $\sigma$ have energy-scaling and length-scaling funktions with the weihgted least-square fitted values of $\epsilon/k = \unit[100.016\,54]{K}$ and $\sigma = \unit[0.365\,024\,96]{nm}$. In \eqref{eq-visc-n2-b} $k$ represents the Boltzmann`s constant, $N_A$ the Avogadro's number and $M$ the molecular weight. For values for $A_i$ in \eqref{eq-visc-n2-c} see Tab.~\ref{tab-visc-n2-ai-c}.

The term $\eta_{ex} (\rho)$ will be described as 
\begin{equation}
	\frac{\Delta\eta_{ex} (\rho)}{\eta_c} = \frac{C_1}{\chi-C_2}+\frac{C_1}{C_2}+\sum_{i=3}^5C_i\chi^{i-2}
	\label{eq-visc-n2-e}
\end{equation}
where $\eta_c$ respresents the critical viscosity faktor and $\chi$ is the reduced density with $\chi = \rho/\rho_c$. The values for parameter $C_i$ are listed in Tab.~\ref{tab-visc-n2-ai-c}.
\begin{table}[H]
  \caption{\label{tab-visc-n2-ai-c}Values for the parameters $A_i$ in \eqref{eq-visc-n2-c} and $C_i$ in \eqref{eq-visc-n2-e}.}
  \begin{center}
  \begin{tabular}{lr@{.}lr@{.}l}
  \toprule
  $i$	& \multicolumn{2}{c}{$A_i$}				&\multicolumn{2}{c}{$C_i$} \\
  \midrule
  0		& $0$&$466\,49$			&\multicolumn{2}{c}{}\\
  1		& $-0$&$572\,15$		& $-20$&$099\,970$\\  
  2  	& $0$&$191\,64$			& $3$&$437\,641\,6$\\ 
  3		& $-0$&$037\,08$		& $-14$&$447\,005\,1$\\  
  4  	& $0$&$002\,41$			& $-0$&$027\,766\,561$\\
  5  	& \multicolumn{2}{c}{}	& $-0$&$216\,623\,62$\\
  \bottomrule
 \end{tabular}
 \end{center}
\end{table}
%####################################################################
\subsection{Methane \cite{FriElyIng:89} and Ethane \cite{FriIngEly:91}}
\label{sec-visc-ch4-c2h6}

For the viscosity of methane and ethane the following equation is given:
\begin{equation}
\eta (\rho,T) = \eta_{0} (T) + \eta_{ex} (\rho,T).
	\label{eq-visc-ch4-a}
\end{equation}
Instead of $\eta_{0} (T)$ it can be written:

for methane
\begin{equation}
	\eta_{0} (T) = \frac{5\sqrt{\pi uM_r kT}}{16\pi\sigma^2\Omega^{(2.2)^*}(t)} = 10.50\frac{\sqrt{t}}{\Omega^{(2.2)^*}}(t)\ \unit[]{\mu Pa\,s},
	\label{eq-visc-ch4-b}
\end{equation}
and for ethane
\begin{equation}
	\eta_{0} (T) = \frac{5\sqrt{\pi uM_r kT}}{16\pi\sigma^2\Omega^{(2.2)^*}(t)} = 12.0085\frac{\sqrt{t}}{\Omega^{(2.2)^*}}(t)\ \unit[]{\mu Pa\,s}.
	\label{eq-visc-ch4-b1}
\end{equation}
$M_r$ stands for the molecular mass with $M_{r,Methane} = \unit[16.043]{g/mol}$ and $M_{r,Ethane} = \unit[30.070]{g/mol}$, $u$ for the unified atomic mass unit with $u = \unit[1.660\,540\,2\cdot 10^{-27}]{kg}$ and $k$ for the Boltzmann constant with $k = \unit[1.380\,658\cdot 10^{-23}]{JK^{-1}}$. The term $\Omega^{(2.2)^*}$ is a function of the reduced temperature $t=kT/\epsilon$ with $\epsilon/k=\unit[174]{K}$. In can be given as
\begin{equation}
	\Omega^{(2.2)^*} = \left[\sum_{i=1}^9C_it^{[(i-1)/3-1]}\right]^{-1}.
	\label{eq-visc-ch4-c}
\end{equation}
The values for the coefficient $C_i$ in \eqref{eq-visc-ch4-c} are shown in Tab.~\ref{tab-visc-ch4-c}.
\begin{table}[h]
\caption{\label{tab-visc-ch4-c}Values for the coefficient $C_i$ in \eqref{eq-visc-ch4-c} for both methane and ethane.}
\begin{center}
\begin{tabular}{cr@{.}lr@{.}l}
\toprule
i		& \multicolumn{2}{c}{$C_i$}\\
\midrule
1		& $-3$&$032\,813\,828\,1$\\
2		& $16$&$918\,880\,086$\\
3		& $-37$&$189\,364\,917$\\
4		& $41$&$288\,861\,858$\\
5		& $-24$&$615\,921\,14$\\
6		& $8$&$948\,843\,096$\\
7		& $-1$&$873\,924\,504\,2$\\
8		& $0$&$209\,661\,013\,90$\\
9		& $-9$&$657\,043\,707\,4\cdot10^-3$\\
\bottomrule
\end{tabular}
\end{center}
\end{table}

The term $\eta_{ex} (\rho,T)$ can be expressed

for methane as
\begin{equation}
\begin{split}
	\eta_{ex} (\rho,T)& = \frac{P_c^{\frac{2}{3}}(M_ru)^{\frac{1}{2}}}{(T_ck)^\frac{1}{6}}\left[\sum_{i=1}^9g_i\delta^{r_i}\tau^{s_i}\right] \left[1+\sum_{i=10}^{11}g_i\delta^{r_i}\tau^{s_i}\right]^{-1}\\
		& = 12.149\left[\sum_{i=1}^9g_i\delta^{r_i}\tau^{s_i}\right] \left[1+\sum_{i=10}^{11}g_i\delta^{r_i}\tau^{s_i}\right]^{-1}\ \unit[]{\mu Pa\,s},
	\label{eq-visc-ch4-d} 
\end{split}
\end{equation}
and for ethane as
\begin{equation}
\begin{split}
	\eta_{ex} (\rho,T)& = \frac{P_c^{\frac{2}{3}}(M_ru)^{\frac{1}{2}}}{(T_ck)^\frac{1}{6}}\left[\sum_{i=1}^9g_i\delta^{r_i}\tau^{s_i}\right] \left[1+\sum_{i=10}^{11}g_i\delta^{r_i}\tau^{s_i}\right]^{-1}\\
		& = 15.977\left[\sum_{i=1}^9g_i\delta^{r_i}\tau^{s_i}\right] \left[1+\sum_{i=10}^{11}g_i\delta^{r_i}\tau^{s_i}\right]^{-1}\ \unit[]{\mu Pa\,s}.
	\label{eq-visc-ch4-d1} 
\end{split}
\end{equation}
In this context $\delta$ is the reduced density with $\delta=\rho/\rho_c$ and $\tau$ is the temperature reduced by the critical temperature with $\tau=T_c/T$. The exponents $r_i$, $s_i$ and the dimensionless fitted coefficient $g_i$ from \eqref{eq-visc-ch4-d} and \eqref{eq-visc-ch4-d1} are given in Tab.~\ref{tab-visc-ch4-rsg}
\begin{table}[H]
  \caption{\label{tab-visc-ch4-rsg}Values for the exponents $r_i$, $s_i$ and the coefficient $g_i$ in \eqref{eq-visc-ch4-d} and \eqref{eq-visc-ch4-d1} for excess transport property funcion.}
  \begin{center}
  \begin{tabular}{lllr@{.}lr@{.}l}
  \toprule
  		&			&			& \multicolumn{2}{c}{methane \eqref{eq-visc-ch4-d}}	& \multicolumn{2}{c}{ethane \eqref{eq-visc-ch4-d1}}\\
   \cmidrule(lr){4-5} \cmidrule(lr){6-7}
  $i$	& $r_i$		&$s_i$		& \multicolumn{2}{c}{$g_i$} 	& \multicolumn{2}{c}{$g_i$}\\
  \midrule
  1		& 1			& 0			& $0$&$412\,501\,37$			& $0$&$471\,770\,03$\\  
  2  	& 1			& 1			& $-0$&$143\,909\,12$			& $-0$&$239\,503\,11$\\ 
  3		& 2			& 0			& $0$&$103\,669\,93$			& $0$&$398\,083\,01$\\  
  4  	& 2			& 1			& $0$&$402\,874\,64$			& $-0$&$273\,433\,35$\\
  5  	& 2			& 1.5		& $-0$&$249\,035\,24$			& $0$&$351\,922\,60$\\
  6		& 3			& 0			& $-0$&$129\,531\,31$			& $-0$&$211\,013\,08$\\
  7		& 3			& 2			& $0$&$065\,757\,76$			& $-0$&$004\,785\,79$\\
  8		& 4			& 0			& $0$&$025\,666\,28$			& $0$&$073\,781\,29$\\
  9		& 4			& 1			& $-0$&$037\,165\,26$			& $-0$&$030\,425\,255$\\
  10	& 1			& 0			& $-0$&$387\,983\,41$			& $-0$&$304\,352\,86$\\
  11	& 1			& 1			& $0$&$035\,338\,15$			& $0$&$001\,215\,675$\\
  \bottomrule
 \end{tabular}
 \end{center}
\end{table}
%###################################################################
\subsection{Water\cite{IAPWS:08a}}
\label{subsec-visc-h2o}

The viscosity for water is given in equation
\begin{equation}
	\overline{\eta} = \overline{\eta}_0(\overline{T}) \cdot \overline{\eta}_1(\overline{T},\overline{\rho}) \cdot \overline{\eta}_2(\overline{T},\overline{\rho}).
	\label{eq-visc-h2o-a}
\end{equation}
In this connection it is used the dimensionless term for temperature $T$, pressure $p$, density $\rho$ and viscosity $\eta$ defined by the overline. They are given  in the following:
\begin{equation}
	\overline{T} = T/T^*,
	\label{eq-visc-h2o-b}
\end{equation}
\begin{equation}
	\overline{p} = p/p^*,
	\label{eq-visc-h2o-c}
\end{equation}
\begin{equation}
	\overline{\rho} = \rho/\rho^*,
	\label{eq-visc-h2o-d}
\end{equation}
\begin{equation}
	\overline{\eta} = \eta/\eta^*.
	\label{eq-visc-h2o-e}
\end{equation}
In this case the * stands for reference values which agree with the values for the critical point of $T$, $p$ and $\rho$. For $\eta$ the reference constant has no physical significance. The reference values are given in Tab.~\ref{tab-visc-h2o-Tprho}. 
\begin{table}[h]
  \caption{\label{tab-visc-h2o-Tprho}Reference values for the variables $T$, $p$, $\rho$ and $\eta$ for \eqref{eq-visc-h2o-b} - \eqref{eq-visc-h2o-e}.}
  \begin{center}
  \begin{tabular}{ll}
  \toprule
  variable	& reference value\\
  \midrule
  $T^*$		& $\unit[647.096]{K}$\\
  $p^*$		& $\unit[22.064]{MPa}$\\
  $\rho^*$	& $\unit[322]{kg \cdot m^{-3}}$\\
  $\eta^*$	& $\unit[1.00\cdot10^{-6}]{Pa\cdot s}$\\
  \bottomrule
 \end{tabular}
 \end{center}
\end{table}
For pressure $p$ and temperature $T$ are the following ranges given:
\begin{align*} 
0 				& < p < p_t 				 & and \ \ \ \ \ \unit[273.16]{K} &\leq T \leq \unit[1173.15]{K}\\ 
p_t 			& \leq p\leq \unit[300]{MPa} & and \ \ \ \ \ \ \ \ T_m(p) 	  &\leq T \leq \unit[1173.15]{K}\\
\unit[300]{MPa} & < p \leq \unit[350]{MPa}   & and \ \ \ \ \ \ \ \ T_m(p) 	  &\leq T \leq \unit[873.15]{K}\\
\unit[350]{MPa} & < p \leq \unit[500]{MPa}   & and \ \ \ \ \ \ \ \ T_m(p) 	  &\leq T \leq \unit[433.15]{K}\\
\unit[500]{MPa} & < p \leq \unit[1000]{MPa}  & and \ \ \ \ \ \ \ \ T_m(p) 	  &\leq T \leq \unit[373.15]{K}
\end{align*}
where $p_t$ is the triple-point pressure and $T_m(p)$ is the pressure-dependent melting temperature.

Based on \eqref{eq-visc-h2o-a} it can be written for the viscosity in the dilute-gas limit $\overline{\eta}_0(\overline{T})$
\begin{equation}
	\overline{\eta}_0(\overline{T}) = \frac{100\sqrt{\overline{T}}}{ \sum\limits_{i=0}^3\frac{H_i}{\overline{T}^i}}
	\label{eq-visc-h2o-f}
\end{equation}	
with the coefficient $H_i$, whose values are shown in Tab.~\ref{tab-visc-h2o-hi}.													
\begin{table}[h]
\caption{\label{tab-visc-h2o-hi}Values for the coefficient $H_i$ in \eqref{eq-visc-h2o-f}.}
\begin{center}
\begin{tabular}{lr@{.}l}
\toprule
$i$ & \multicolumn{2}{c}{$H_i$}\\
\midrule
$0$ & $1$&$677\,52$\\
$1$ & $2$&$204\,62$\\
$2$ & $0$&$636\,656\,4$\\
$3$ & $-0$&$241\,605$\\
\bottomrule
\end{tabular}
\end{center}
\end{table}

The term $\overline{\eta}_1(\overline{T},\overline{\rho})$ in \eqref{eq-visc-h2o-a} represents the contribution to viscosity due to finite density and is given by
\begin{equation}
	\overline{\eta}_1(\overline{T},\overline{\rho}) = exp\left[\overline{\rho}\sum_{i=0}^5\left(\frac{1}{\overline{T}}-1\right)\sum_{j=0}^6H_{ij}(\overline{\rho}-1)^j\right].
	\label{eq-visc-h2o-g}
\end{equation}
The values for the coefficient $H_{ij}$ are shown in Tab.~\ref{tab-visc-h2o-hij}. The coefficients $H_{ij}$ which were omitted from Tab.~\ref{tab-visc-h2o-hij} are identically equal to zero.
\begin{table}[ht]
  \caption{\label{tab-visc-h2o-hij}Values for coefficients $H_{ij}$ in \eqref{eq-visc-h2o-g}.}
  \begin{center}
  \begin{tabular}{llr@{.}l}
  \toprule
  $i$	& $j$		&\multicolumn{2}{c}{$H_{ij}$} \\
  \midrule
  0		& 0			& $5$&$200\,94\cdot 10^-1$\\  
  1  	& 0			& $8$&$508\,95\cdot 10^-2$\\ 
  2		& 0			& $-1$&$083\,74$\\  
  3  	& 0			& $-2$&$895\,55\cdot 10^-1$\\
  0  	& 1			& $2$&$225\,31\cdot 10^-1$\\
  1		& 1			& $9$&$991\,15\cdot 10^-1$\\
  2		& 1			& $1$&$887\,97$\\
  3		& 1			& $1$&$266\,13$\\
  5		& 1			& $1$&$205\,73\cdot 10^-1$\\
  0		& 2			& $-2$&$813\,78\cdot 10^-1$\\
  1		& 2			& $-9$&$068\,51\cdot 10^-1$\\
  2		& 2			& $-7$&$724\,79\cdot 10^-1$\\
  3		& 2			& $-4$&$898\,37\cdot 10^-1$\\
  4		& 2			& $-2$&$570\,40\cdot 10^-1$\\
  0		& 3			& $1$&$619\,13\cdot 10^-1$\\
  1		& 3			& $2$&$573\,99\cdot 10^-1$\\
  0		& 4			& $-3$&$253\,72\cdot 10^-2$\\
  3		& 4			& $6$&$984\,52\cdot 10^-2$\\
  4		& 5			& $8$&$721\,02\cdot 10^-3$\\
  3		& 6			& $-4$&$356\,73\cdot 10^-3$\\
  5		& 6			& $-5$&$932\,64\cdot 10^-4$\\
  \bottomrule
 \end{tabular}
 \end{center}
\end{table}

The third factor $\overline{\eta}_2(\overline{T},\overline{\rho})$ in \eqref{eq-visc-h2o-a} represents the critical enhancement of the viscosity. It is only significant in a very small region in density and temperature around the critical point. Exactly at the critical point the viscosity is infinite, but the enhancement term contributes an amount greater than 2~\% of the full viscosity only within the following boundaries:
\begin{equation}
	\unit[645.91]{K} < T < \unit[650.77]{K},\hspace{1cm} \unit[245.8]{kg m^{-3}} < \rho < \unit[405.3]{kgm^{-3}}. 
	\label{eq-visc-h2o-h}
\end{equation}
Only within the boundaries of \eqref{eq-visc-h2o-h} the critical enhancement is significant. Outside the region given in \eqref{eq-visc-h2o-h} the enhancement is less than the uncertainty in the formulation and allows a simplification to reduce complexity and computing time:
\begin{equation}
	\overline{\eta}_2 = 1.
	\label{eq-visc-h2o-i}
\end{equation} 
For the function $\overline{\eta}_2$ defined for the entire ranges of states it can be refered to \cite{IAPWS:08a}.
%############################################################################

\subsection{Propane}\cite{VogKueBi:98}

For the viscosity of propane exists the equation
\begin{equation}
	\eta(\rho,T) = \eta_0(T)+ \eta_1(T)\rho + \Delta\eta_h(\rho,T) + \Delta\eta_c(\rho,T).
	\label{eq-visc-c3h8-a}
\end{equation}
For the viscosity in the zero-density limit it can be written
\begin{equation}
	\eta_0(T) = \frac{0.021\,357(MT)^\frac{1}{2}}{\sigma^2 \mathfrak{S}^*_\eta(T^*)},
	\label{eq-visc-c3h8-b}
\end{equation}
whereas $M$ represents the molar mass and $\sigma$ is a scaling parameter with $\sigma=\unit[0.497\,48]{nm}$. The expression $\ln \mathfrak{S}^*_\eta(T^*)$ can be replaced by
\begin{equation}
	\ln \mathfrak{S}^*_\eta(T^*) = \sum_{i=0}^4 a_i(\ln T^*)^i,
	\label{eq-visc-c3h8-c}
\end{equation}
and $T^*$ by
\begin{equation}
	T^* = \frac{k_BT}{\varepsilon}.
	\label{eq-visc-c3h8-d}
\end{equation}
Here $\varepsilon/k_B$ is a scaling parameter with $\varepsilon/k_B=\unit[263.88]{K}$. Furthermore the values for the coefficient $a_i$ ar shown in Tab.~\ref{tab-visc-c3h8-bi-ai}.
\begin{table}[h]
\caption{\label{tab-visc-c3h8-bi-ai}Values for the coefficient $b_i$ in \eqref{eq-visc-c3h8-g}.}
\begin{center}
\begin{tabular}{lr@{.}lr@{.}l}
\toprule
i	& \multicolumn{2}{c}{$b_i$}	& \multicolumn{2}{c}{$a_i$}\\
\midrule
0	& $-19$&$572\,881$			& $0$&$251\,045\,74$\\
1	& $219$&$739\,99$			& $-0$&$472\,712\,38$\\
2	& $-1015$&$322\,6$			& \multicolumn{2}{c}{}\\
3	& $2471$&$012\,51$			& $0$&$060\,836\,515$\\
4	& $-3375$&$171\,7$			& \multicolumn{2}{c}{}\\
5	& $2491$&$659\,7$			&\multicolumn{2}{c}{}\\
6	& $-787$&$260\,86$			&\multicolumn{2}{c}{}\\
7	& $14$&$085\,455$			&\multicolumn{2}{c}{}\\
8	& $-0$&$346\,641\,58$		&\multicolumn{2}{c}{}\\
\bottomrule
\end{tabular}
\end{center}
\end{table}

The term $\eta_1(T)\rho$ should be calculated by the following equations:
\begin{equation}
	B_\eta^*(T^*) = \sum_{i=0}^6 b_i(T^*)^{-0.25i} + b_i(T^*)^{-2.5} + b_8(T^*)^{-5.5},
	\label{eq-visc-c3h8-g}
\end{equation}
\begin{equation}
	B_\eta^*(T^*) = \frac{B_\eta(T)}{N_A \sigma^3}\ \ \ \longrightarrow\ \ \ B_\eta(T) = B_\eta^*(T^*) N_A \sigma^3,
	\label{eq-visc-c3h8-f}
\end{equation}
\begin{equation}
	B_\eta(T) = \frac{\eta_1(T)}{\eta_0(T)}\ \ \ \longrightarrow\ \ \ \eta_1(T) = B_\eta(T)\, \eta_0(T)
	\label{eq-visc-c3h8-e}
\end{equation}
whereas $B_\eta$ represents the viscosity virial coefficient and $N_A$ is the Avogadro's number with $N_A=6.022\,141\,79 \cdot 1023 \unit[]{mol^{-1}}$. For values of $b_i$ see Tab.~\ref{tab-visc-c3h8-bi-ai}.

For the third term in \eqref{eq-visc-c3h8-a} $\Delta\eta_h(\rho,T)$ there exists the following expression, but this is related to the reduced terms of density and temperature $\delta=\rho/\rho_c$ and $\tau=T/T_c$
\begin{equation}
	\Delta\eta_h(\delta,\tau) = \sum_{i=2}^5 \sum_{i=0}^2 e_{ij} \frac{\delta^i}{\tau^j} - f_k \left(\frac{\delta}{\delta_0(\tau)-\delta} - \frac{\delta}{\delta_0(\tau)} \right).
	\label{eq-visc-c3h8-f}
\end{equation}
with $f_k=1616.884\,053\,74$ and
\begin{equation}
	\delta_0(\tau) = g_1(1+g_2\tau^{\frac{1}{2}}).
	\label{eq-visc-c3h8-g}
\end{equation}
with $g_1=2.500\,539\,388\,63$ and $g_2=0.860\,516\,059\,264$. For $f_k$ in \eqref{eq-visc-c3h8-f} there are values given in Tab.~\ref{tab-visc-c3h8-eij}.
\begin{table}[h]
\caption{\label{tab-visc-c3h8-eij}Values for the coefficient $e_{ij}$ in \eqref{eq-visc-c3h8-f}.}
\begin{center}
\begin{tabular}{llr@{.}l}
\toprule
i	& j	& \multicolumn{2}{c}{$e_{ij}$}\\
\midrule
2	& 0	& $35$&$987\,303\,019\,5$\\
2	& 1	& $-180$&$512\,188\,564$\\
2	& 2	& $87$&$712\,488\,822\,3$\\
3	& 0	& $-105$&$773\,052\,525$\\
3	& 1	& $205$&$319\,740\,877$\\
3	& 2	& $-129$&$210\,932\,610$\\
4	& 0	& $58$&$949\,158\,775\,9$\\
4	& 1	& $-129$&$740\,033\,100$\\
4	& 2	& $76$&$628\,041\,997\,1$\\
5	& 0	& $-9$&$594\,078\,684\,75$\\
5	& 1	& $21$&$072\,698\,659\,8$\\
5	& 2	& $-14$&$397\,196\,818\,7$\\
\bottomrule
\end{tabular}
\end{center}
\end{table}

The term $\Delta\eta_c(\rho,T)$ is neglectible and is set to zero.
%############################################################################

\section{Thermal conductivity}

%############################################################################
\subsection{Carbon Dioxide \cite{VesWak:90}}

For carbon dioxide the following equation exists:
\begin{equation}
	\lambda(\rho,T) = \lambda^0(T)+\Delta\lambda(\rho,T)+\Delta_c\lambda(\rho,T).
	\label{ew-tc-co2-a}
\end{equation}
The term $\Delta_c\lambda(\rho,T)$ is neglectible because of the validity only in a very small region around the critical point. For $\lambda^0(T)$ it can be written
\begin{equation}
	\lambda^0(T) = \frac{0.475589(T)^{\frac{1}{2}}(1+r^2)}{\Omega_\lambda^*(T^*)}.
	\label{eq-tc-co2-b}
\end{equation}
The term $\Omega_\lambda^*(T^*)$ can be replaced by
\begin{equation}
	\Omega_\lambda^*(T^*) = \sum_{i=0}^7b_i/T^{*i}.
	\label{eq-tc-co2-c}
\end{equation}
For $b_i$ there are given values shown in Tab.~\ref{tab-tc-co2-bi-ci}. The parameter $T^*$ can be replaced by
\begin{equation}
	T^* = \frac{kT}{\epsilon},
	\label{eq-tc-co2-d}
\end{equation}
where $\epsilon/k=\unit[251.196]{K}$ with $k$ as the Boltzmann constant $k = \unit[1.380\,658\cdot 10^{-23}]{JK^{-1}}$. 
\begin{table}[h]
\caption{\label{tab-tc-co2-bi-ci}Values for the coefficient $b_i$ in \eqref{eq-tc-co2-c}.}
\begin{center}
\begin{tabular}{cr@{.}lr@{.}l}
\toprule
i		& \multicolumn{2}{c}{$b_i$}		& \multicolumn{2}{c}{$c_i$}\\
\midrule
0		& $0$&$422\,615\,9$				& \multicolumn{2}{c}{}\\
1		& $0$&$628\,011\,5$				& $2$&$387\,869\cdot10^{-2}$\\
2		& $-0$&$538\,766\,1$			& $4$&$350\,794$\\
3		& $0$&$673\,594\,1$				& $-10$&$334\,04$\\
4		& $0$&$0$						& $7$&$981\,590$\\
5		& $0$&$0$						& $-1$&$940\,558$\\
6		& $-0$&$436\,267\,7$			& \multicolumn{2}{c}{}\\
7		& $0$&$225\,538\,8$				& \multicolumn{2}{c}{}\\
\bottomrule
\end{tabular}
\end{center}
\end{table}
For $r$ in \eqref{eq-tc-co2-b} it can be written
\begin{equation}
	r = \left(\frac{2c_{int}}{5k}\right)^{1/2},
	\label{eq-tc-co2-e}
\end{equation}
where
\begin{equation}
	\frac{c_{int}}{k} = 1.0+e^{-183.5/T} \sum_{i=1}^5c_i(T/100)^{2-i}.
	\label{eq-tc-co2-f}
\end{equation}
The values for the coefficient $c_i$ are given in Tab.~\ref{tab-tc-co2-bi-ci}.

%######################################################################
\subsection{Nitrogen \cite{SteKraLae:87}}

For the thermal conductivity of nitrogen the following equation exists:
\begin{equation}
	\lambda(\rho,T) = \lambda_0(T) + \Delta\lambda_R(\rho).
	\label{eq-tc-n2-a}
\end{equation}
The term $\lambda_0(T)$ stands for the conductivity in the zero-density limit and $\Delta\lambda_R(\rho)$ for the residual thermal conductivity. $\lambda_0(T)$ is given by
\begin{equation}
	\lambda_0(T) = \lambda_{tr} + \lambda_{in}
	\label{eq-tc-n2-b}
\end{equation}
with 
\begin{equation}
	\lambda_{tr} = -2.5F(1.5-X_1)
	\label{eq-tc-n2-c}
\end{equation}
as the part due to the translational energy transfer and
\begin{equation}
	\lambda_{in} = FX_2 [c_{v0}(T)/(kN_A)+X_1]
	\label{eq-tc-n2-d}
\end{equation}
as the part due to the internal degrees of freedom of the molecule. Thereby $X_1$ and $X_2$ are adjustable parameters with $X_1=0.951\,852\,02$ and $X_2=1.020\,542\,2$. For $F$ it can be written 
\begin{equation}
	F = kN_A\eta_0(T)/M.
	\label{eq-tc-n2-e}
\end{equation}
For $\eta_0$ and $k$, $N_A$ and $M$ see \eqref{eq-visc-n2-b} and the corresponding explanation. The term $c_{v0}$ in \eqref{eq-tc-n2-d} represents the ideal isochoric heat capacity. It can be describes by the following equations (\eqref{eq-tc-n2-f}-\eqref{eq-tc-n2-h}):
\begin{equation}
	c_{v0} = R(B_iT^{i-4}+\frac{(B_8UU(D+1))}{D^2})-1
	\label{eq-tc-n2-f}
\end{equation}
with 
\begin{equation}
	U = \frac{B_9}{T}
	\label{eq-tc-n2-g}
\end{equation}
and
\begin{equation}
	D = e^U-1.
	\label{eq-tc-n2-h}
\end{equation}
The variables $B_8$ in \eqref{eq-tc-n2-f} and $B_9$ in \eqref{eq-tc-n2-g} are given with $B_8=0.100\,773\,735\,767 \cdot 10^1$ and $B_9=0.335\,340\,610 \cdot 10^4$. For the other parameters $B_i$ in \eqref{eq-tc-n2-f} the values are shown in Tab.~\ref{tab-tc-n2-b}.
\begin{table}[h]
\caption{\label{tab-tc-n2-b}Values for the parameter $B_i$ in \eqref{eq-tc-n2-f}.}
\begin{center}
\begin{tabular}{lr@{.}l}
\toprule
i		&\multicolumn{2}{c}{$B_i$}\\
\midrule
1		& $-0$&$837\,079\,888\,737 \cdot 10^3$\\
2		& $0$&$379\,147\,114\,487 \cdot 10^2$ \\
3		& $-0$&$601\,737\,844\,275$\\
4		& $0$&$350\,418\,363\,823 \cdot 10^1$\\
5		& $-0$&$874\,955\,653\,028 \cdot ^10^{-5}$\\
6		& $0$&$148\,968\,607\,239 \cdot 10^{-7}$\\
7		& $-0$&$256\,370\,354\,277 \cdot10^{-11}$\\
\bottomrule
\end{tabular}
\end{center}
\end{table}

The second term $\Delta\lambda_R(\rho)$ in \eqref{eq-tc-n2-a} can be replaced by the following expression:
\begin{equation}
	\Delta\lambda_R(\rho)/\lambda_c = \sum_{i=1}^4C_i\chi^i,
	\label{eq-tc-n2-i}
\end{equation}
where $\lambda_c$ is the critical thermal conductivity with $\lambda_c=4.17$  and $\chi=\rho/\rho_c$ is the reduced density with $\rho_c=314$. The values for $C_i$ are given in Tab.~\ref{tab-tc-n2-ci}. 
\begin{table}[h]
\caption{\label{tab-tc-n2-ci}Values for $C_i$ in \eqref{eq-tc-n2-i}.}
\begin{center}
\begin{tabular}{ll}
\toprule
i		&$C_i$\\
\midrule
1		&3.337\,354\,2\\
2		&0.370\,982\,51\\
3		&0.899\,134\,56\\
4		&0.169\,725\,05\\
\bottomrule
\end{tabular}
\end{center}
\end{table}
%#######################################################################
\subsection{Methane \cite{YouEly:87}}

For the thermal conductivity of methan the following equation exists:
\begin{equation}
	\lambda = \lambda_0+(F_0+F_1\rho)\rho/(1-F_2\rho)+\lambda_c.
	\label{eq-tc-ch4-a}
\end{equation}
Here the term $\lambda_0$ can be replaced by 
\begin{equation}
	\lambda_0 = 1000\eta_0(C_p^0-5R/2)[G_t(1)+G_t(2)\epsilon/kT],
	\label{eq-tc-ch4-b}
\end{equation}
whereas for $\eta_0$ it can be refered to \eqref{eq-visc-ch4-b} with the corresponding explanations. For the coefficient $G_t(1)$ and $G_t(2)$ are the values given $G_t(1) = 1.346\,953\,698$ and $G_t(2) = -0.325\,467\,775\,3$. For $\epsilon/k$ it can be assumed $\epsilon/k=\unit[174]{K}$. The ideal gas spezific heat $C_p^0$ is computed by
\begin{equation}
	\frac{C_p^0}{R} = \sum_{n=1}^7G_i(n)T^{(n-4)}+\frac{G_i(8)u^2e^u}{(e^u-1)^2}
	\label{eq-tc-ch4-c}
\end{equation}
with
\begin{equation}
	u = \frac{G_i(9)}{T}.
	\label{eq-tc-ch4-d}
\end{equation}
For the coefficient $G_i(n)$ are existing the following values, which are shown in Tab.~\ref{tab-tc-ch4-gin}.
\begin{table}[h]
\caption{\label{tab-tc-ch4-gin}Values for the coefficient $G_i(n)$ in \eqref{eq-tc-ch4-c} and \eqref{eq-tc-ch4-d}.}
\begin{center}
\begin{tabular}{cr@{.}l}
\toprule
n		& \multicolumn{2}{c}{$G_i(n)$}\\
\midrule
1		& $-1$&$804\,475\,050\,7\cdot10^6$\\
2		& $7$&$742\,666\,639\,3\cdot10^4$\\
3		& $-1$&$324\,165\,875\,4\cdot10^3$\\
4		& $1$&$543\,814\,959\,5\cdot10^1$\\
5		& $-5$&$147\,900\,525\,7\cdot19^-2$\\
6		& $1$&$080\,917\,219\,6\cdot10^-4$\\
7		& $-6$&$550\,178\,343\,7\cdot10^-8$\\
8		& $-6$&$749\,005\,617\,1$\\
9		& \multicolumn{2}{l}{$3000$}\\
\bottomrule
\end{tabular}
\end{center}
\end{table}
The terms $F_0$, $F_1$ and $F_2$ in \eqref{eq-tc-ch4-a} are given by the following expressions:
\begin{equation}
	F_0 = \sum_{n=1}^3E_t(n)T^{(1-n)},
	\label{eq-tc-ch4-e}
\end{equation}
\begin{equation}
	F_1 = \sum_{n=4}^6E_t(n)T^{(1-n)}
	\label{eq-tc-ch4-f}
\end{equation}
and
\begin{equation}
	F_2 = \sum_{n=7}^8E_t(n)T^{(1-n)}.
	\label{eq-tc-ch4-g}
\end{equation}
For the coefficient $E_t(n)$ in \eqref{eq-tc-ch4-e} till \eqref{eq-tc-ch4-g} are existing values shown in Tab.~\ref{tab-tc-ch4-etn}
\begin{table}[H]
\caption{\label{tab-tc-ch4-etn}Values for the coefficient $E_t(n)$ in \eqref{eq-tc-ch4-e} till \eqref{eq-tc-ch4-g}.}
\begin{center}
\begin{tabular}{cr@{.}l}
\toprule
n		& \multicolumn{2}{c}{$E_t(n)$}\\
\midrule
1		& $0$&$232\,580\,081\,9\cdot10^-2$\\
2		& $-0$&$247\,792\,799\,9$\\
3		& $0$&$388\,059\,371\,3\cdot10^2$\\
4		& $-0$&$157\,951\,914\,6\cdot10^-6$\\
5		& $0$&$371\,799\,132\,8\cdot10^-2$\\
6		& $-0$&$961\,698\,943\,4$\\
7		& $-0$&$301\,735\,277\,4\cdot10^-1$\\
8		& $0$&$429\,815\,338\,6$\\
\bottomrule
\end{tabular}
\end{center}
\end{table}
%#########################################################################
\subsection{Ethane \cite{FriIngEly:91}}

The thermal conductivity for ethane is given in the following equation
\begin{equation}
	\lambda(\rho,T) = \lambda_0(T)+\lambda_{ex}(\rho,T)+\lambda_{cr}(\rho,T).
	\label{eq-tc-c2h6-a}
\end{equation}
For $\lambda_0$ it can be written
\begin{equation}
\begin{split}
	\lambda_0(T) & = \frac{\eta_0(T)} {M_r u N_A}\left[\frac{15R}{4}+f_{int}(C_p^{id}-5R/2)\right] \\& = 0.276\,505\,\eta_0(T)\left[3.75-f_{int}(\tau^2 \phi^{id}_{\tau\tau}+1.5)\right]\ \unit[]{mW\cdot m^{-1}\cdot K^{-1}} 
	\label{eq-tc-c2h6-b}
\end{split}
\end{equation}
where $R$ represents the Gas constant with $R = \unit[8.314\,472]{J/ mol \cdot K}$, $N_A$ the Avogadro's number with $N_A=6.022\,141\,79 \cdot 1023 \unit[]{mol^{-1}}$, $M_r$ is the molecular weight and $u$ is the unified atomic mass unit. For $\eta_0$ it can be refered to \eqref{eq-visc-ch4-b1} in section \ref{sec-visc-ch4-c2h6}. The term $C_p^{id}$ represents the (temperature dependent) ideal gas contribution to the molar isobaric heat capacity. For this and for the term $\tau^2 \phi^{id}_{\tau\tau}$ it is refered to \cite{FriIngEly:91}. The expression $f_{int}$ is a dimensionless function describing the energy exchange. It is given as 
\begin{equation}
	f_{int} = f_1+ \frac{f_2}{t}
	\label{eq-tc-c2h6-c}
\end{equation}
with the coefficients $f_1=1.710\,414\,7$ and $f_2=-0.693\,648\,2$.

For the excess thermal conductivity $\lambda_{ex}$ it can be written
\begin{equation}
\begin{split}
	\lambda_{ex} & = \frac{P_c^{\frac{2}{3}}k^{\frac{5}{6}}}{T_c^{\frac{1}{6}}(M_ru)^{\frac{1}{2}}} \sum_{i=1}^7j_i\delta^{r_i}\tau^{s_i}\\ & = 4.417\,86 \left[\sum_{i=1}^7j_i\delta^{r_i}\tau^{s_i}\right]\ \unit[]{mW\cdot m^{-1}\cdot K^{-1}}
	\label{eq-tc-c2h6-d}
\end{split}
\end{equation}
where $k$ represents the Boltzmann constant with $k = \unit[1.380\,658\cdot 10^{-23}]{JK^{-1}}$. For $\delta$ and $\tau$ see the comments on \eqref{eq-visc-ch4-d} and \eqref{eq-visc-ch4-d1}  in section \ref{sec-visc-ch4-c2h6}. The values for the coefficients $r_i$, $s_i$ and $j_i$ are shown in Tab.~\ref{tab-tc-c2h6-rsj}.
\begin{table}[h]
\caption{\label{tab-tc-c2h6-rsj}Values for the coefficients $r_i$, $s_i$ and $j_i$ from \eqref{eq-tc-c2h6-d}.}
\begin{center}
\begin{tabular}{lllr@{.}l}
\toprule
i	& $r_i$	& $s_i$	&\multicolumn{2}{c}{$j_i$}\\
\midrule
1	& 1		& 0		&$0$&$960\,843\,22$\\
2	& 2		& 0		&$2$&$750\,023\,5$\\
3	& 3		& 0		&$-0$&$026\,609\,289$\\
4	& 4		& 0		&$-0$&$078\,146\,729$\\
5	& 5		& 0		&$0$&$218\,813\,39$\\
6	& 1		& 1.5	&$2$&$384\,956\,3$\\
7	& 3		& 1		&$-0$&$751\,139\,71$\\
\bottomrule
\end{tabular}
\end{center}
\end{table}

For the term $\lambda_{cr}(\rho,T)$ it is given the following equation:
\begin{equation}
\begin{split}
	\lambda_{cr}(\rho,T) & = \frac{\Lambda kT\rho C_p}{6\pi\eta(\rho,T)\xi}F(\rho,T)\\ & = 1.55 \frac{\delta}{\tau}\frac{C_p}{\eta\xi}F(\delta,\tau) \ \unit[]{mW\cdot m^{-1}\cdot K^{-1}}. 
	\label{eq-tc-c2h6-e}
\end{split}
\end{equation}
The constant $\Lambda$ has the value $\Lambda=1.01$. The viscosity $\eta(\rho,T)$ is represented in \eqref{eq-visc-ch4-a} in section \ref{sec-visc-ch4-c2h6}. For $\xi$ in \eqref{eq-tc-c2h6-e} is given
\begin{equation}
\begin{split}
\xi & = \xi_0 \Gamma_0^{\frac{-\nu}{\gamma}} \left(\frac{P_c}{\rho_c^2}\right)^{\frac{\nu}{\gamma}}\left[\rho\left(\frac{\partial\rho}{\partial P}\vert_T - \frac{2T_c}{T}\frac{\partial\rho}{\partial P}\vert_{T=2T_c}\right)\right]^{\frac{\nu}{\gamma}} \\ 
& = 0.428 (\delta\tau)^{0.507}  \bigg[(1+2\delta\phi_\delta^r + \delta^2 \phi_{\delta\delta}^r)^{-1} - (1+2\delta\phi_\delta^r(\frac{1}{2}) \\ & + \delta^2 \phi_\delta\delta^r(\frac{1}{2}))^{-1}\bigg]^{0.507} \ \unit[]{nm}.
\label{eq-tc-c2h6-f}
\end{split}
\end{equation}
The critical exponents $\gamma$ and $\nu$ are given with $\gamma=1.242$ and $\nu=0.63$. For $\xi_0$ stands the value $\xi_0=\unit[0.19]{nm}$ and for $\Gamma_0$ stands $\Gamma_0=0.0563$. The term $F$ from \eqref{eq-tc-c2h6-e} can be written as
\begin{equation}
\begin{split}
	F & = \frac{2}{\pi} \bigg[e^{-q_D\xi[1+(q_D\xi)^3/(3\delta^2)]^{-1}} \\ & - 1+\frac{C_p-C_V}{C_p}\left(\tan^{-1}(q_D\xi)+\frac{C_v}{C_p-C_v}q_D\xi \right)  \bigg]
	\label{eq-tc-c2h6-g}
\end{split}
\end{equation}
Here for $q_D^{-1}$ it is given the value $q_D^{-1}=\unit[0.545]{nm}$. For the molar isochoric heat capacity $C_v$ and the molar isobaric heat capacity $C_p$ it is refered to \cite{FriIngEly:91}.
%##############################################################################
\subsection{Propane \cite{MaPerRa:02}}
For the thermal conductivity of propane exists the equation
\begin{equation}
	\lambda(\rho,T) = \lambda_0(T) + \Delta\lambda_r(\rho,T) + \Delta\lambda_c(\rho;T).
	\label{eq-tc-c3h8-a}
\end{equation}
The term $\lambda_0(T)$ can be replaced by 
\begin{equation}
	\lambda_0(T) = A_1 + A_2\left(\frac{T}{T_c}\right) + A_3\left(\frac{T}{T_c}\right)^2,
	\label{eq-tc-c3h8-b}
\end{equation}
where $A_i$ is a coefficient whose values are shown Tab.~\ref{tab-tc-c3h8-ai}.
\begin{table}[h]
\caption{\label{tab-tc-c3h8-ai}Values for the coefficient $A_i$ in \eqref{eq-tc-c3h8-b}.}
\begin{center}
\begin{tabular}{lr@{.}l}
\toprule
i	& \multicolumn{2}{c}{$\unit[A_i]{[Wm^{-1}K^{-1}]}$}\\
\midrule
1	& $-1$&$247\,78 \cdot 10^{-3}$\\
2	& $8$&$163\,71 \cdot 10^{-3}$\\
3	& $1$&$993\,74 \cdot 10^{-2}$\\
\bottomrule
\end{tabular}
\end{center}
\end{table}
For $\Delta\lambda_r(\rho;T)$ it can be written
\begin{equation}
	\Delta\lambda_r(\rho;T) = \sum_{i=1}^5\left(B_{i,1} + B_{i,2}\left(\frac{T}{T_c}\right)\right)\left(\frac{\rho}{\rho_c}\right)^i,
	\label{eq-tc-c3h8-c}
\end{equation}
with the coefficients $B_i$ whose values are shown in Tab.~\ref{tab-tc-c3h8-bi}.
\begin{table}[h]
\caption{\label{tab-tc-c3h8-bi}Values for the coefficients $B_i$ in \eqref{eq-tc-c3h8-c}.}
\begin{center}
\begin{tabular}{lr@{.}lr@{.}l}
\toprule
&\multicolumn{4}{c}{$\unit[B_{i}]{[Wm^{-1}K^{-1}]}$}\\
\cmidrule{2-5}
i	& \multicolumn{2}{c}{$B_{i,1}$}	& \multicolumn{2}{c}{$B_{i,2}$}\\
\midrule
1	& $-3$&$511\,52 \cdot 10^{-2}$							& $4$&$691\,95 \cdot 10^{-2}$\\
2	& $1$&$708\,90 \cdot 10^{-1}$							& $-1$&$486\,16 \cdot 10^{-1}$\\
3	& $-1$&$476\,88 \cdot 10^{-3}$							& $1$&$324\,57 \cdot 10^{-1}$\\
4	& $5$&$192\,83 \cdot 10^{-2}$							& $-4$&$856\,36 \cdot 10^{-2}$\\
5	& $-6$&$186\,62 \cdot 10^{-3}$							& $6$&$604\,14 \cdot 10^{-3}$\\
\bottomrule
\end{tabular}
\end{center}
\end{table}
Instead of $\Delta\lambda_c(\rho;T)$ the following equation can be used
\begin{equation}
	\Delta\lambda_c(\rho,T) = \frac{C_1}{C_2+ \lvert \Delta T_c\rvert } \exp[-(C_3\Delta \rho_c)^2].
	\label{eq-tc-c3h8-d}
\end{equation}
Here $\Delta T_c$ and $\Delta \rho_c$ are given as $\Delta T_c=\frac{T}{T_c}-1$ and $\Delta \rho_c=\frac{\rho}{\rho_c}-1$. For the values of coefficient $C_i$ see Tab.~\ref{tab-tc-c3h8-ci}.
\begin{table}[h]
\caption{\label{tab-tc-c3h8-ci}Values for the coefficient $C_i$ in \eqref{eq-tc-c3h8-d}.}
\begin{center}
\begin{tabular}{lr@{.}l}
\toprule
i	& \multicolumn{2}{c}{$\unit[C_i]{[Wm^{-1}K^{-1}]}$}\\
\midrule
1	& $3$&$664\,86 \cdot 10^{-4}$\\
2	& $-2$&$216\,96 \cdot 10^{-3}$\\
3	&$2$&$642\,13 \cdot 10^0$\\
\bottomrule
\end{tabular}
\end{center}
\end{table}
%##############################################################################
\subsection{Water\cite{IAPWS:08b}}

The thermal conductivity for water is given in equation
\begin{equation}
	\overline{\lambda} = \overline{\lambda}_0(\overline{T}) \cdot \overline{\lambda}_1(\overline{T},\overline{\rho}) + \overline{\lambda}_2(\overline{T},\overline{\rho}).
	\label{eq-tc-h2o-a}
\end{equation}
In this connection it is used the dimensionless term for temperature $T$, pressure $p$, density $\rho$, thermal conductivity $\lambda$ and the (symmetrized) compressibility $\overline{\chi}_T$ defined by the overline. They are given  in the following:
\begin{equation}
	\overline{T} = T/T^*,
	\label{eq-tc-h2o-b}
\end{equation}
\begin{equation}
	\overline{p} = p/p^*,
	\label{eq-tc-h2o-c}
\end{equation}
\begin{equation}
	\overline{\rho} = \rho/\rho^*,
	\label{eq-tc-h2o-d}
\end{equation}
\begin{equation}
	\overline{\lambda} = \lambda/\lambda^*,
	\label{eq-tc-h2o-e}
\end{equation}
\begin{equation}
	\overline{\chi}_T=\overline{\rho}\left[\frac{\partial\overline{\rho}}{\partial\overline{p}}\right]^2_{\overline{T}}
	\label{eq-tc-h2o-f}
\end{equation}
In this case the * stands for reference values which are close to but not identical with the critical constants of $T$, $p$ and $\rho$.  The reference values are given in Tab.~\ref{tab-tc-h2o-Tprho}. 
\begin{table}[h]
  \caption{\label{tab-tc-h2o-Tprho}Reference values for the variables $T$, $p$, $\rho$ and $\lambda$.}
  \begin{center}
  \begin{tabular}{ll}
  \toprule
  variable	& reference value\\
  \midrule
  $T^*$		& $\unit[647.226]{K}$\\
  $p^*$		& $\unit[22.115\cdot10^{-6}]{Pa}$\\
  $\rho^*$	& $\unit[317.763]{kg \cdot m^{-3}}$\\
  $\lambda^*$	& $\unit[0.4945]{W\cdot m^{-1}\cdot K^{-1}}$\\
  \bottomrule
 \end{tabular}
 \end{center}
\end{table}
For pressure $p$ and temperature $T$ are the following ranges of validity given:
\begin{align} 
p \leq \unit[400]{MPa}	& \hspace{2cm} for \hspace{2.5cm} \ \ \ \unit[0]{^{\circ}C}	&\leq T \leq \unit[125]{^{\circ}C}\\ 
p \leq \unit[200]{MPa}  & \hspace{2cm} for \hspace{2.5cm} \unit[125]{^{\circ}C} &< T \leq \unit[250]{^{\circ}C}\\
p \leq \unit[150]{MPa}  & \hspace{2cm} for \hspace{2.5cm} \unit[250]{^{\circ}C}	&< T \leq \unit[400]{^{\circ}C}\\
p \leq \unit[100]{MPa}  & \hspace{2cm} for \hspace{2.5cm} \unit[400]{^{\circ}C}	&< T \leq \unit[800]{^{\circ}C}
\end{align}
Based on \eqref{eq-tc-h2o-a} it can be written for the thermal conductivity in the dilute-gas limit $\overline{\lambda}_0(\overline{T})$
\begin{equation}
	\overline{\lambda}_0(\overline{T}) = \frac{\sqrt{\overline{T}}}{ \sum\limits_{i=0}^3\frac{L_i}{\overline{T}^i}}
	\label{eq-tc-h2o-g}
\end{equation}	
with the coefficient $L_i$, whose values are shown in Tab.~\ref{tab-tc-h2o-li}.													
\begin{table}[h]
\caption{\label{tab-tc-h2o-li}Values for the coefficient $L_i$ in \eqref{eq-tc-h2o-g}.}
\begin{center}
\begin{tabular}{lr@{.}l}
\toprule
$i$ & \multicolumn{2}{c}{$L_i$}\\
\midrule
$0$ & $1$&$000\,000$\\
$1$ & $6$&$978\,267$\\
$2$ & $2$&$599\,096$\\
$3$ & $-0$&$998\,254$\\
\bottomrule
\end{tabular}
\end{center}
\end{table}

The term $\overline{\lambda}_1(\overline{T},\overline{\rho})$ in \eqref{eq-tc-h2o-a} can be written as
\begin{equation}
	\overline{\lambda}_1(\overline{T},\overline{\rho}) = exp\left[\overline{\rho}\sum_{i=0}^4\sum_{j=0}^5 L_{ij}\left(\frac{1}{\overline{T}}-1\right)^i(\overline{\rho}-1)^j\right].
	\label{eq-tc-h2o-h}
\end{equation}
The values for the coefficient $L_{ij}$ in \eqref{eq-tc-h2o-h} are shown in Tab.~\ref{tab-tc-h2o-lij}. 
\begin{table}[ht]
  \caption{\label{tab-tc-h2o-lij}Values for coefficients $L_{ij}$ in \eqref{eq-tc-h2o-h}.}
  \begin{center}
  \begin{tabular}{lr@{.}lr@{.}lr@{.}lr@{.}lr@{.}l}
  \toprule
$j$	&\multicolumn{2}{c}{$L_{0j}$} &\multicolumn{2}{c}{$L_{1j}$} &\multicolumn{2}{c}{$L_{2j}$} &\multicolumn{2}{c}{$L_{3j}$} &\multicolumn{2}{c}{$L_{4j}$} \\
  \midrule
  0		& $1$&$329\,304\,6$		& $1$&$701\,836\,3$		& $5$&$224\,615\,8$ 	& $8$&$712\,767\,5$ 	& $-1$&$852\,599\,9$\\  
  1  	& $-0$&$404\,524\,37$	& $-2$&$215\,684\,5$	& $-10$&$124\,111$		& $-9$&$500\,061\,1$	& $0$&$934\,046\,90$\\ 
  2		& $0$&$244\,094\,90$	& $1$&$651\,105\,7$		& $4$&$987\,468\,7$		& $4$&$378\,660\,6$		& 0\\  
  3  	& $0$&$018\,660\,751$	& $-0$&$767\,360\,02$	& $-0$&$272\,976\,94$	& $-0$&$917\,837\,82$	& 0\\
  4  	& $-0$&$129\,610\,68$	& $0$&$372\,833\,44$	& $-0$&$430\,833\,93$	& 0						& 0\\
  5		& $0$&$044\,809\,953$	& $-0$&$112\,031\,60$	& $0$&$133\,338\,49$	& 0						& 0\\
  \bottomrule
 \end{tabular}
 \end{center}
\end{table}

The term $\overline{\lambda}_2(\overline{T},\overline{\rho})$ in \eqref{eq-tc-h2o-a} accounts for an enhancement of the thermal conductivity in the critical region and is defined by
\begin{equation}
\begin{split}
	\overline{\lambda}_2(\overline{T},\overline{\rho}) & =\frac{(55.071)(0.001\,384\,8)}{\overline{\eta}_0(\overline{T})\cdot \overline{\eta}_1(\overline{T},\overline{\rho})}\left(\frac{\overline{T}}{\overline{\rho}}\right)^2 \left(\frac{\partial\overline{p}}{\partial\overline{T}}\right)^2_{\overline{\rho}} \overline{\chi}^{0.4678}_T\ \overline{\rho}^{\frac{1}{2}} \\[0.3cm] 
 &\ \ \ \ \cdot exp \left[-18.66(\overline{T}-1)^2 -(\overline{\rho}-1)^4\right].
	\label{eq-tc-h2o-j}
	\end{split}
\end{equation}
The functions $\overline{\eta}_0(\overline{T})$ and $\overline{\eta}_1(\overline{T},\overline{\rho})$ are those defined in \eqref{eq-visc-h2o-a} in section \ref{subsec-visc-h2o}.

%###########################################################################################
\section{Free Helmholtz Energy}
\subsection{Theory\cite{WatDoo:96}}
All thermodynamic properties of a pure substance can be derived from 
\begin{equation}
	\frac{f(\rho,T)}{RT} = \phi(\delta,\tau) = \phi^0(\delta,\tau)+\phi^r(\delta,\tau),
	\label{eq-fhe-a}
\end{equation}
where $R=\unit[0.461\,518\,05]{kJ\,kg^{-1}\,K^{-1}}$, $\delta=\rho/\rho_c$ and $\tau=T_c/T$ with $\rho_c=\unit[322]{kg\,m^{-3}}$ and $T_c=\unit[647.096]{K}$. Thereby it is used the appropriate combination of the ideal-gas part $\phi^0$  
\begin{equation}
\phi^{o}=\operatorname{ln}\delta + n^{o}_{1}+n^{o}_{2}\tau +n^{o}_{3}\operatorname{ln}\tau +\sum^{8}_{i=4} n^{o}_{i} \operatorname{ln}\left[ 1-e^{-\gamma^{o}_{i}\tau}\right] 
\label{eq-fhe-b}
\end{equation}
and the residual part $\phi^r$  of the dimensionless Helmholtz free energy and their derivatives
\begin{equation}
\begin{split}
\phi^{r}&=\sum^{k_1}_{i=1}n_i\delta^{d_i}\tau^{t_i}+\sum^{k_2}_{i=k_1+1}n_i\delta^{d_i}\tau^{t_i}e^{-\delta^{c_i}}\\&
+\sum^{k_3}_{i=k_2+1}n_i\delta^{d_i}\tau^{t_i}e^{-\alpha_i\left(\delta-\epsilon_i\right)^2-\beta\left(\tau-\gamma_i\right)^2}\\& +\sum^{k_4}_{i=k_3+1} n_i\Delta^{b_i}\delta e^{-C_i\left(\delta-1\right)^2-D_i\left(\tau-1\right)^2}.
\end{split}
\label{eq-fhe-c}
\end{equation}
with 
\begin{equation}
	\Delta=\lbrace \left(1-\tau\right)+A_i[\left(\delta-1\right)^2]^{1/(2\beta_i)}\rbrace^2+B_i[(\delta-1)^2]^{a_i}.
	\label{eq-fhe-d}
\end{equation} 
Relations between thermodynamic properties and $\phi^0$ and $\phi^r$ and their derivatives are summarized in Tab.~\ref{tab-fhe-prop}.
\begin{table}[H]
\caption{\label{tab-fhe-prop}Relations of thermodynamic properties to the ideal-gas part $\phi^0$ and the residual part $\phi^r$ of the dimensionless Helmholtz free energy and their derivatives.}
\begin{center}
\begin{tabular}{ll}
\toprule
Property					& Relation\\
\midrule
pressure					& $\frac{p(\delta,\tau)}{\rho RT}=1+\delta\phi^r_\delta$\\
\cmidrule{2-2}
isochoric heat capacity		& $\frac{c_v(\delta,\tau)}{R}=-\tau^2(\phi^0_{\tau\tau}+\phi^r_{\tau\tau})$\\
\cmidrule{2-2}
isobaric heat capacity		& $\frac{c_p(\delta,\tau)}{R}=-\tau^2(\phi^0_{\tau\tau}+\phi^r_{\tau\tau})+\frac{(1+\delta\phi^r_\delta-\delta\tau\phi^r_{\delta\tau})^2}{1+2\delta\phi^r_\delta+\delta^2\phi^r_{\delta\delta}}$\\
\bottomrule
\end{tabular}
\end{center}
\end{table}
For the coefficients $n_i^0$, $\gamma_i^0$ from \eqref{eq-fhe-b} and $c_i$, $d_i$, $t_i$, $n_i$, $A_i$, $B_i$, $C_i$, $D_i$, $a_i$, $b_i$, $\alpha_i$, $\beta_i$, $\gamma_i$ and $\epsilon_i$  from \eqref{eq-fhe-c} are existing different values depending from the substances. In the following these values will be presented for some liquides and gases.

%++++++++++++++++++++++++++++++++++++++++++++++++++++++++++++++++++++++++++++++++++++++++++++++++++++++++++++++++++++++++++++++++++++++++

%\clearpage
\subsection{Water}
For the limit $k$ of the summator in \eqref{eq-fhe-c} the following values are given for water:
\begin{compactitem}
	\item $k_1=7$
	\item $k_2=51$
	\item $k_3=54$
	\item $k_4=56$.
\end{compactitem}
\begin{table}[h]
\caption{\label{tab-fhe-h2o-ideal}Values for $n_i^o$ and $\gamma_i^o$ in \eqref{eq-fhe-b} for water.}
\begin{center}
\begin{tabular}{lr@{.}lr@{.}l}
\toprule
i	& \multicolumn{2}{c}{$n_i^o$}	& \multicolumn{2}{c}{$\gamma_i^o$}\\
\midrule
1	& $-8$&$320\,446\,482\,01$		& \multicolumn{2}{c}{}\\
2	& $6$&$683\,210\,526\,8$		& \multicolumn{2}{c}{}\\
3	& $3$&$006\,32$					& \multicolumn{2}{c}{}\\
4	& $0$&$012\,436$				& $1$&$287\,289\,67$\\
5	& $0$&$973\,15$					& $3$&$537\,342\,22$\\
6	& $1$&$279\,50$					& $7$&$740\,737\,08$\\
7	& $0$&$969\,56$					& $9$&$244\,377\,96$\\
8	& $0$&$248\,73$					& $27$&$507\,510\,5$\\
\bottomrule
\end{tabular}
\end{center}
\end{table}

\begin{table}[h]
\caption{\label{tab-fhe-h2o-resi-a}Values for the coefficients $a_i$, $b_i$, $A_i$, $B_i$, $C_i$, $D_i$, $\alpha_i$, $\beta_i$, $\gamma_i$ and $\epsilon_i$ from \eqref{eq-fhe-c} and \eqref{eq-fhe-d} for water.}
\begin{center}
\begin{tabular}{lcccccccccc}
\toprule
i	& $a_i$	& $b_i$		& $A_i$		& $B_i$		& $C_i$		& $D_i$	 & $\alpha_i$	& $\beta_i$	& $\gamma_i$	& $\epsilon_i$	\\
\midrule
52	&		&			&			&			&			&		 & 20			& 150		& 1.21			&	 1	\\
53	&		&			&			&			&			&		 & 20			& 150		& 1.21			&	 1	\\
54	&		&			&			&			&			&		 & 20			& 250		& 1.25			&	 1	\\
55	& 3.5	& 0.85		& 0.32		& 0.2		& 28		& 700	 &				& 0.3		&				&		\\			
56	& 3.5	& 0.95		& 0.32		& 0.2		& 32		& 800	 &				& 0.3		&				&		\\
\bottomrule
\end{tabular}
\end{center}
\end{table}

\begin{table}
\caption{\label{tab-fhe-h2o-resi-b}Values for the coefficients $n_i$, $d_i$, $t_i$ and $c_i$ from \eqref{eq-fhe-c} for water.}
\begin{center}
\scriptsize       
\begin{longtable}{lr@{.}lccc}
\toprule
i	& \multicolumn{2}{c}{$n_i$}						&	$d_i$	& $t_i$			& $c_i$ \\	
\midrule
1	& $1$&$253\,354\,793\,552\,3 \cdot 10^{-2}$		&	1		& -0.5			&		\\
2	& $7$&$895\,763\,472\,282\,8$			 		&	1		& 0.875			&		\\
3	& $-8$&$780\,320\,330\,356\,1$			 		&	1		& 1				&		\\
4	& $3$&$180\,250\,934\,541\,8 \cdot 10^{-1}$		&	2		& 0.5			&		\\
5	& $-2$&$614\,553\,385\,935\,8 \cdot 10^{-1}$	&	2		& 0.75			&		\\
6	& $-7$&$819\,975\,168\,798\,1 \cdot 10^{-3}$	&	3		& 0.375			&		\\
7	& $8$&$808\,949\,310\,213\,4 \cdot 10^{-3}$		&	4		& 1				&		\\
8	& $-6$&$685\,657\,230\,796\,5 \cdot 10^{-1}$	&	1		& 4				&	1	\\
9	& $2$&$043\,381\,095\,096\,5 \cdot 10^{-1}$		&	1		& 6				&	1	\\	
10	& $-6$&$621\,260\,503\,968\,7 \cdot 10^{-5}$	&	1		& 12			&	1	\\
11	& $-1$&$923\,272\,115\,600\,2 \cdot 10^{-1}$	&	2		& 1				&	1	\\
12	& $-2$&$570\,904\,300\,343\,8 \cdot 10^{-1}$	&	2		& 5				&	1	\\
%13	& $1$&$607\,486\,848\,625�?,1 \cdot 10^{-1}$		&	3		& 4				&	1	\\%Fehler NB
14	& $-4$&$009\,282\,892\,580\,7 \cdot 10^{-2}$	&	4		& 2				&	1	\\
15	& $3$&$934\,342\,260\,325\,4 \cdot 10^{-7}$		&	4		& 13			&	1	\\
16	& $-7$&$594\,137\,708\,814\,4 \cdot 10^{-6}$	&	5		& 9				&	1	\\
17	& $5$&$625\,097\,935\,188\,8 \cdot 10^{-4}$		&	7		& 3				&	1	\\
18	& $-1$&$560\,865\,225\,713\,5 \cdot 10^{-5}$	&	9		& 4				&	1	\\
19	& $1$&$153\,799\,642\,295\,1 \cdot 10^{-9}$		&	10		& 11			&	1	\\
20	& $3$&$658\,216\,514\,420\,4 \cdot 10^{-7}$		&	11		& 4				&	1	\\
21	& $-1$&$325\,118\,007\,466\,8 \cdot 10^{-12}$	&	13		& 13			&	1	\\
22	& $-6$&$263\,958\,691\,245\,4 \cdot 10^{-10}$	&	15		& 1				&	1	\\
23	& $-1$&$079\,360\,090\,893\,2 \cdot 10^{-1}$	&	1		& 7				&	2	\\
24	& $1$&$761\,149\,100\,875\,2 \cdot 10^{-2}$		&	2		& 1				&	2	\\
25	& $2$&$213\,229\,516\,754\,6 \cdot 10^{-1}$		&	2		& 9				&	2	\\
26	& $-4$&$024\,766\,976\,352\,8 \cdot 10^{-1}$	&	2		& 10			&	2	\\
27	& $5$&$808\,339\,998\,575\,9 \cdot 10^{-1}$		&	3		& 10			&	2	\\
28	& $4$&$996\,914\,699\,080\,6 \cdot 10^{-3}$		&	4		& 3				&	2	\\
29	& $-3$&$135\,870\,071\,254\,9 \cdot 10^{-2}$	&	4		& 7				&	2	\\
30	& $-7$&$431\,592\,971\,034\,1 \cdot 10^{-1}$	&	4		& 10			&	2	\\
31	& $4$&$780\,732\,991\,548\,0 \cdot 10^{-1}$		&	5		& 10			&	2	\\
32	& $2$&$052\,794\,089\,594\,8 \cdot 10^{-2}$		&	6		& 6				&	2	\\
33	& $-1$&$363\,643\,511\,034\,3 \cdot 10^{-1}$	&	6		& 10			&	2	\\
34	& $1$&$418\,063\,440\,061\,7 \cdot 10^{-2}$		&	7		& 10			&	2	\\
35	& $8$&$332\,650\,488\,071\,3 \cdot 10^{-3}$		&	9		& 1				&	2	\\
36	& $-2$&$905\,233\,600\,958\,5 \cdot 10^{-2}$	&	9		& 2				&	2	\\
37	& $3$&$861\,508\,557\,420\,6 \cdot 10^{-2}$		&	9		& 3				&	2	\\
38	& $-2$&$039\,348\,651\,370\,4 \cdot 10^{-2}$	&	9		& 4				&	2	\\
39	& $-1$&$655\,405\,006\,373\,4 \cdot 10^{-3}$	&	9		& 8				&	2	\\
40	& $1$&$995\,557\,197\,954\,1 \cdot 10^{-3}$		&	10		& 6				&	2	\\	
41	& $1$&$587\,030\,832\,415\,7 \cdot 10^{-4}$		&	10		& 9				&	2	\\
42	& $-1$&$638\,856\,834\,253\,0 \cdot 10^{-5}$	&	12		& 8				&	2	\\
43	& $4$&$361\,361\,572\,381\,1 \cdot 10^{-2}$		&	3		& 16			&	3	\\
44	& $3$&$499\,400\,546\,376\,5 \cdot 10^{-2}$		&	4		& 22			&	3	\\
45	& $-7$&$678\,819\,784\,462\,1 \cdot 10^{-2}$	&	4		& 23			&	3	\\
46	& $2$&$244\,627\,733\,200\,6 \cdot 10^{-2}$		&	5		& 23			&	3	\\
47	& $-6$&$268\,971\,041\,468\,5 \cdot 10^{-5}$	&	14		& 10			&	4	\\
48	& $-5$&$571\,111\,856\,564\,5 \cdot 10^{-10}$	&	3		& 50			&	6	\\
49	& $-1$&$990\,571\,835\,440\,8 \cdot 10^{-1}$	&	6		& 44			&	6	\\
50	& $3$&$177\,749\,733\,073\,8 \cdot 10^{-1}$		&	6		& 46			&	6	\\
51	& $-1$&$184\,118\,242\,598\,1 \cdot 10^{-1}$	&	6		& 50			&	6	\\
52	& $-3$&$130\,626\,032\,343\,5 \cdot 10^{1}$		&	3		& 0				&		\\
53	& $3$&$154\,614\,023\,778\,1 \cdot 10^{1}$		&	3		& 1				&		\\
54	& $-2$&$521\,315\,434\,169\,5 \cdot 10^{3}$		&	3		& 4				&		\\
55	& $-1$&$487\,464\,085\,672\,4 \cdot 10^{-1}$	&			& 				&		\\
56	& $3$&$180\,611\,087\,844\,4 \cdot 10^{-1}$		&			& 		 		&		\\
\bottomrule
\end{longtable}
\end{center}
\end{table}

%\begin{table}
%\caption{\label{tab-fhe-h20-resi-a}Values for the coefficients $n_i$, $d_i$, $t_i$ and $c_i$ from \eqref{eq-fhe-c}.}
%\begin{center}
%\scriptsize       
%\begin{longtable}{lr@{.}llr@{.}ll|lr@{.}llr@{.}ll}
%\toprule
%i	& \multicolumn{2}{c}{n}							&	d	& \multicolumn{2}{c}{t}	& c	&i	& \multicolumn{2}{c}{n}							&	d	& \multicolumn{2}{c}{t}	& c  \\	
%\midrule
%\endhead
%1	& $1$&$253\,354\,793\,552\,3 \cdot 10^{-2}$		&	1	& $-0$&$5$	&		& 29	& $-3$&$135\,870\,071\,254\,9 \cdot 10^{-2}$	&	4	& $7$&$0$	&	2	\\
%2	& $7$&$895\,763\,472\,282\,8$			 		&	'V (V )V *V +V                                                                                                                                                                                                                                                                                                                                                                                                                                                                                                                                                                                                                                                                                                                                                                                                                                                                                                                                                                                                                                                                                                                                                                                                                                                                                                                                                                                                                                                                                                                                                                                                                                                                                                                                                                                                                                                                                                                                                                                                                                                                                                                                                                                                                                                                                                                                                                                                                                                                                                                                                                                                                                                                                                                                                                                                                                                                                                                                                                                                                                                                                                                                                                                                                                                                                                                                                                                                                                                                                                                                                                                                                                                                                                                                                                                                                                                                                                                                                                                                                                                                                                                                                                                                                                             1	& $875$&$0$	&		& 30	& $-7$&$431\,592\,971\,034\,1 \cdot 10^{-1}$	&	4	& $10$&$0$	&	2	\\
%3	& $-8$&$780\,320\,330\,356\,1$			 		&	1	& $1$&$0$	&		& 31	& $4$&$780\,732\,991\,548\,0 \cdot 10^{-1}$		&	5	& $10$&$0$	&	2	\\
%4	& $3$&$180\,250\,934\,541\,8 \cdot 10^{-1}$		&	2	& $0$&$5$	&		& 32	& $2$&$052\,794\,089\,594\,8 \cdot 10^{-2}$		&	6	& $6$&$0$	&	2	\\
%5	& $-2$&$614\,553\,385\,935\,8 \cdot 10^{-1}$	&	2	& $0$&$75$	&		& 33	& $-1$&$363\,643\,511\,034\,3 \cdot 10^{-1}$	&	6	& $10$&$0$	&	2	\\
%6	& $-7$&$819\,975\,168\,798\,1 \cdot 10^{-3}$	&	3	& $375$&$0$	&		& 34	& $1$&$418\,063\,440\,061\,7 \cdot 10^{-2}$		&	7	& $10$&$0$	&	2	\\
%7	& $8$&$808\,949\,310\,213\,4 \cdot 10^{-3}$		&	4	& $1$&$0$	&		& 35	& $8$&$332\,650\,488\,071\,3 \cdot 10^{-3}$		&	9	& $1$&$0$	&	2	\\
%8	& $-6$&$685\,657\,230\,796\,5 \cdot 10^{-1}$	&	1	& $4$&$0$	&	1	& 36	& $-2$&$905\,233\,600\,958\,5 \cdot 10^{-2}$	&	9	& $2$&$0$	&	2	\\
%9	& $2$&$043\,381\,095\,096\,5 \cdot 10^{-1}$		&	1	& $6$&$0$	&	1	& 37	& $3$&$861\,508\,557\,420\,6 \cdot 10^{-2}$		&	9	& $3$&$0$	&	2	\\	
%10	& $-6$&$621\,260\,503\,968\,7 \cdot 10^{-5}$	&	1	& $12$&$0$	&	1	& 38	& $-2$&$039\,348\,651\,370\,4 \cdot 10^{-2}$	&	9	& $4$&$0$	&	2	\\
%11	& $-1$&$923\,272\,115\,600\,2 \cdot 10^{-1}$	&	2	& $1$&$0$	&	1	& 39	& $-1$&$655\,405\,006\,373\,4 \cdot 10^{-3}$	&	9	& $8$&$0$	&	2	\\
%12	& $-2$&$570\,904\,300\,343\,8 \cdot 10^{-1}$	&	2	& $5$&$0$	&	1	& 40	& $1$&$995\,557\,197\,954\,1 \cdot 10^{-3}$		&	10	& $6$&$0$	&	2	\\	
%13	& $1$&$607\,486\,848\,625�?,1 \cdot 10^{-1}$		&	3	& $4$&$0$	&	1	& 41	& $1$&$587\,030\,832\,415\,7 \cdot 10^{-4}$		&	10	& $9$&$0$	&	2	\\
%14	& $-4$&$009\,282\,892\,580\,7 \cdot 10^{-2}$	&	4	& $2$&$0$	&	1	& 42	& $-1$&$638\,856\,834\,253\,0 \cdot 10^{-5}$	&	12	& $8$&$0$	&	2	\\
%15	& $3$&$934\,342\,260\,325\,4 \cdot 10^{-7}$		&	4	& $13$&$0$	&	1	& 43	& $4$&$361\,361\,572\,381\,1 \cdot 10^{-2}$		&	3	& $16$&$0$	&	3	\\
%16	& $-7$&$594\,137\,708\,814\,4 \cdot 10^{-6}$	&	5	& $9$&$0$	&	1	& 44	& $3$&$499\,400\,546\,376\,5 \cdot 10^{-2}$		&	4	& $22$&$0$	&	3	\\
%17	& $5$&$625\,097\,935\,188\,8 \cdot 10^{-4}$		&	7	& $3$&$0$	&	1	& 45	& $-7$&$678\,819\,784\,462\,1 \cdot 10^{-2}$	&	4	& $23$&$0$	&	3	\\
%18	& $-1$&$560\,865\,225\,713\,5 \cdot 10^{-5}$	&	9	& $4$&$0$	&	1	& 46	& $2$&$244\,627\,733\,200\,6 \cdot 10^{-2}$		&	5	& $23$&$0$	&	3	\\
%19	& $1$&$153\,799\,642\,295\,1 \cdot 10^{-9}$		&	10	& $11$&$0$	&	1	& 47	& $-6$&$268\,971\,041\,468\,5 \cdot 10^{-5}$	&	14	& $10$&$0$	&	4	\\
%20	& $3$&$658\,216\,514\,420\,4 \cdot 10^{-7}$		&	11	& $4$&$0$	&	1	& 48	& $-5$&$571\,111\,856\,564\,5 \cdot 10^{-10}$	&	3	& $50$&$0$	&	6	\\
%21	& $-1$&$325\,118\,007\,466\,8 \cdot 10^{-12}$	&	13	& $13$&$0$	&	1	& 49	& $-1$&$990\,571\,835\,440\,8 \cdot 10^{-1}$	&	6	& $44$&$0$	&	6	\\
%22	& $-6$&$263\,958\,691\,245\,4 \cdot 10^{-10}$	&	15	& $1$&$0$	&	1	& 50	& $3$&$177\,749\,733\,073\,8 \cdot 10^{-1}$		&	6	& $46$&$0$	&	6	\\
%23	& $-1$&$079\,360\,090\,893\,2 \cdot 10^{-1}$	&	1	& $7$&$0$	&	2	& 51	& $-1$&$184\,118\,242\,598\,1 \cdot 10^{-1}$	&	6	& $50$&$0$	&	6	\\
%24	& $1$&$761\,149\,100\,875\,2 \cdot 10^{-2}$		&	2	& $1$&$0$	&	2	& 52	& $-3$&$130\,626\,032\,343\,5 \cdot 10^{1}$		&	3	& $0$&$0$	&		\\
%25	& $2$&$213\,229\,516\,754\,6 \cdot 10^{-1}$		&	2	& $9$&$0$	&	2	& 53	& $3$&$154\,614\,023\,778\,1 \cdot 10^{1}$		&	3	& $1$&$0$	&		\\
%26	& $-4$&$024\,766\,976\,352\,8 \cdot 10^{-1}$	&	2	& $10$&$0$	&	2	& 54	& $-2$&$521\,315\,434\,169\,5 \cdot 10^{3}$		&	3	& $4$&$0$	&		\\
%27	& $5$&$808\,339\,998\,575\,9 \cdot 10^{-1}$		&	3	& $10$&$0$	&	2	& 55	& $-1$&$487\,464\,085\,672\,4 \cdot 10^{-1}$	&		&			&		\\
%28	& $4$&$996\,914\,699\,080\,6 \cdot 10^{-3}$		&	4	& $3$&$0$	&	2	& 56	& $3$&$180\,611\,087\,844\,4 \cdot 10^{-1}$		&		&			&		\\
%\bottomrule
%\end{longtable}
%\end{center}
%\end{table}

%++++++++++++++++++++++++++++++++++++++++++++++++++++++++++++++++++++++++++++++++++++++++++++++++++++++++++++++++++++++++++++++++++++++++

\clearpage
\subsection{Carbon Dioxide}
For the limit $k$ of the summator in \eqref{eq-fhe-c} the following values are given for carbon dioxide:
\begin{compactitem}
	\item $k_1=7$
	\item $k_2=34$
	\item $k_3=39$
	\item $k_4=42$.
\end{compactitem}
\begin{table}[h]
\caption{\label{tab-fhe-co2-ideal}Values for $n_i^o$ and $\gamma_i^o$ in \eqref{eq-fhe-b} for carbon dioxide.}
\begin{center}
\begin{tabular}{lr@{.}lr@{.}l}
\toprule
i	& \multicolumn{2}{c}{$n_i^o$}		& \multicolumn{2}{c}{$\gamma_i^o$}	\\
\midrule
1	& $8$&$373\,044\,56$			& \multicolumn{2}{c}{}	\\
2	& $-3$&$704\,543\,04$			& \multicolumn{2}{c}{}	\\
3	& $2$&$5$						& \multicolumn{2}{c}{}	\\
4	& $1$&$994\,270\,42$			& $3$&$151\,63$	\\
5	& $0$&$621\,052\,48$			& $6$&$111\,9$	\\
6	& $0$&$411\,952\,93$			& $6$&$777\,08$	\\
7	& $1$&$040\,289\,22$			& $11$&$323\,84$ \\
8	& $0$&$083\,276\,78$			& $27$&$087\,92$	\\
\bottomrule
\end{tabular}
\end{center}
\end{table}

\begin{table}[h]
\caption{\label{tab-fhe-co2-resi-a}Values for the coefficients $a_i$, $b_i$, $A_i$, $B_i$, $C_i$, $D_i$, $\alpha_i$, $\beta_i$, $\gamma_i$ and $\varepsilon_i$ from \eqref{eq-fhe-c} and \eqref{eq-fhe-d} for carbon dioxide.}
\begin{center}
\begin{tabular}{lcccccccccc}
\toprule
i	& $a_i$	& $b_i$	& $A_i$	& $B_i$	& $C_i$	& $D_i$	& $\alpha_i$	& $\beta_i$	& $\gamma_i$	& $\varepsilon_i$	\\
\midrule
35	&		&		&		&		&		&		& 25	& 325	& 1.16	&	1	\\
36	&		&		&		&		&		&		& 25	& 300	& 1.19	&	1	\\
37	&		&		&		&		&		&		& 25	& 300	& 1.19	&	1	\\
38	&		&		&		&		&		&		& 15	& 275	& 1.25	&	1	\\
39	&		&		&		&		&		&		& 20	& 275	& 1.22	&	1	\\
40	& 3.5	& 0.875	& 0.7	& 0.3	& 10	& 275	&		& 0.3	&		&		\\
41	& 3.5	& 0.925	& 0.7	& 0.3	& 10	& 275	&		& 0.3	&		&		\\
42	& 3		& 0.875	& 0.7	& 1		& 12.5	& 275	&		& 0.3	&		&		\\
\bottomrule
\end{tabular}
\end{center}
\end{table}

\begin{table}
\caption{\label{tab-fhe-co2-resi-b}Values for the coefficients $n_i$, $d_i$, $t_i$ and $c_i$ from \eqref{eq-fhe-c} for carbon dioxide.}
\begin{center}
\begin{tabular}{lr@{.}lccc}
\toprule
i	& \multicolumn{2}{c}{$n_i$}						& $d_i$	& $t_i$ 	& $c_i$	\\
\midrule		
1	& $3$&$885\,682\,320\,316\,1 \cdot 10^{-1}$		&	  1	&	 0		&		\\
2	& $2$&$938\,547\,594\,274\,0$					&	  1	&	 0.75	&		\\
3	& $-5$&$586\,718\,853\,493\,4$					&	  1	&	 1		&		\\
4	& $-7$&$675\,319\,959\,247\,7 \cdot 10^{-1}$	&	  1	&	 2		&		\\
5	& $3$&$172\,900\,558\,041\,6 \cdot 10^{-1}$		&	  2	&	 0.75	&		\\
6	& $5$&$480\,331\,589\,776\,7 \cdot 10^{-1}$		&	  2	&	 2		&		\\
7	& $1$&$227\,941\,122\,033\,5 \cdot 10^{-1}$		&	  3	&	 0.75	&		\\
8	& $2$&$165\,896\,154\,322\,0$					&	  1	&	 1.5	&	 1	\\
9	& $1$&$584\,173\,510\,972\,4$					&	  2	&	 1.5	&	 1	\\
10	& $-2$&$313\,270\,540\,550\,3 \cdot 10^{-1}$	&	  4	&	 2.5	&	 1	\\
11	& $5$&$811\,691\,643\,143\,6 \cdot 10^{-2}$		&	 5	&	 0		&	 1	\\
12	& $-5$&$536\,913\,720\,538\,2 \cdot 10^{-1}$	&	 5	&	 1.5	&	 1	\\
13	& $4$&$894\,661\,590\,942\,2 \cdot 10^{-1}$		&	 5	&	 2		&	 1	\\
14	& $-2$&$427\,573\,984\,350\,1 \cdot 10^{-2}$	&	 6	&	 0		&	 1	\\
15	& $6$&$249\,479\,050\,167\,8 \cdot 10^{-2}$		&	 6	&	 1		&	 1	\\
16	& $-1$&$217\,586\,022\,524\,6 \cdot 10^{-1}$	&	 6	&	 2		&	 1	\\
17	& $-3$&$705\,568\,527\,008\,6 \cdot 10^{-1}$	&	 1	&	 3		&	 2	\\
18	& $-1$&$677\,587\,970\,042\,6 \cdot 10^{-2}$	&	 1	&	 6		&	 2	\\
19	& $-1$&$196\,073\,663\,798\,7 \cdot 10^{-1}$	&	 4	&	 3		&	 2	\\
20	& $-4$&$561\,936\,250\,877\,8 \cdot 10^{-2}$	&	 4	&	 6		&	 2	\\
%21	& $3$&$561�?,278\,927\,034\,6 \cdot 10^{-2}$		&	 4	&	 8		&	 2	\\ %Fehler
22	& $-7$&$442\,772\,713\,205\,2 \cdot 10^{-3}$	&	 7	&	 6		&	 2	\\
23	& $-1$&$739\,570\,490\,243\,2 \cdot 10^{-3}$	&	 8	&	 0		&	 2	\\
24	& $-2$&$181\,012\,128\,952\,7 \cdot 10^{-2}$	&	 2	&	 7		&	 3	\\
%25	& $2$&$433\,216\,655�?,923\,6 \cdot 10^{-2}$		&	 3	&	 12		&	 3	\\
26	& $-3$&$744\,013\,342\,346\,3 \cdot 10^{-2}$	&	 3	&	 16		&	 3	\\
27	& $1$&$433\,871\,575\,687\,8 \cdot 10^{-1}$		&	 5	&	 22		&	 4	\\
28	& $-1$&$349\,196\,908\,328\,6 \cdot 10^{-1}$	&	 5	&	 24		&	 4	\\
29	& $-2$&$315\,122\,505\,348\,0 \cdot 10^{-2}$	&	 6	&	 16		&	 4	\\
%30	& $1$&$236\,312�?,549\,290\,1 \cdot 10^{-2}$		&	 7	&	 24		&	 4	\\
31	& $2$&$105\,832\,197\,294\,0 \cdot 10^{-3}$		&	 8	&	 8		&	 4	\\
32	& $-3$&$395\,851\,902\,636\,8 \cdot 10^{-4}$	&	 10	&	 2		&	 4	\\
33	& $5$&$599\,365\,177\,159\,2 \cdot 10^{-3}$		&	 4	&	 28		&	 5	\\
34	& $-3$&$033\,511\,805\,564\,6 \cdot 10^{-4}$	&	 8	&	 14		&	 6	\\
35	& $-2$&$136\,548\,868\,832\,0 \cdot 10^{2}$		&	 2	&	 1		&		\\
36	& $2$&$664\,156\,914\,927\,2 \cdot 10^{4}$		&	 2	&	 0		&		\\
37	& $-2$&$402\,721\,220\,455\,7 \cdot 10^{4}$		&	 2	&	 1		&		\\
38	& $-2$&$834\,160\,342\,399\,9 \cdot 10^{2}$		&	 3	&	 3		&		\\
39	& $2$&$124\,728\,440\,017\,9 \cdot 10^{2}$		&	 3	&	 3		&		\\
40	& $-6$&$664\,227\,654\,075\,1 \cdot 10^{-1}$	&		&			&		\\
41	& $7$&$260\,863\,234\,989\,7 \cdot 10^{-1}$		&		&			&		\\
42	& $5$&$506\,866\,861\,284\,2 \cdot 10^{-2}$		&		&			&		\\
\bottomrule
\end{tabular}
\end{center}
\end{table}

%++++++++++++++++++++++++++++++++++++++++++++++++++++++++++++++++++++++++++++++++++++++++++++++++++++++++++++++++++++++++++++++++++++++++

\clearpage
\subsection{Methane}
For the limit $k$ of the summator in \eqref{eq-fhe-c} the following values are given for methane:
\begin{compactitem}
	\item $k_1=13$
	\item $k_2=36$
	\item $k_3=40$
\end{compactitem}

\begin{table}[h]
\caption{\label{tab-fhe-ch4-ideal}Values for $n_i^o$ and $\gamma_i^o$ in \eqref{eq-fhe-b} for methane.}
\begin{center}
\begin{tabular}{lr@{.}lr@{.}l}
\toprule
i	& \multicolumn{2}{c}{$n_i^o$}		& \multicolumn{2}{c}{$\gamma_i^o$}	\\
\midrule
1	& $9$&$912\,439\,72$			& \multicolumn{2}{c}{}	\\
2	& $-6$&$332\,700\,87$			& \multicolumn{2}{c}{}	\\
3	& $3$&$001\,6$					& \multicolumn{2}{c}{}	\\
4	& $0$&$008\,449$				& $3$&$400\,432\,4$	\\		
5	& $4$&$694\,2$					& $10$&$269\,515\,75$	\\
6	& $3$&$486\,5$					& $20$&$439\,327\,47$	\\
7	& $1$&$657\,2$					& $29$&$937\,448\,84$	\\
8	& $1$&$411\,5$					& $79$&$133\,519\,45$	\\
\bottomrule
\end{tabular}
\end{center}
\end{table}

\begin{table}[h]
\caption{\label{tab-fhe-ch4-resi-a}Values for the coefficients $\alpha_i$, $\beta_i$, $\gamma_i$ and $\epsilon_i$ from \eqref{eq-fhe-c} and \eqref{eq-fhe-d} for methane.}
\begin{center}
\begin{tabular}{lcccc}
\toprule
i	& $\alpha_i$	& $\beta_i$		& $\gamma_i$	& $\epsilon_i$	\\
\midrule							
37	& 20			& 200			& 1.07			& 1	\\
38	& 40			& 250			& 1.11			& 1	\\
39	& 40			& 250			& 1.11			& 1	\\
40	& 40			& 250			& 1.11			& 1	\\
\bottomrule
\end{tabular}
\end{center}
\end{table}

%\begin{table}
%\caption{\label{tab-fhe-ch4-resi-b}Values for the coefficients $n_i$, $d_i$, $t_i$ and $c_i$ from \eqref{eq-fhe-c} for methane.}
%\begin{center}
%\begin{tabular}{lr@{.}lccc}
%\toprule
%i	& \multicolumn{2}{c}{$n_i$}	& $d_i$	& $t_i$	& $c_i$	\\
%\midrule									
%1	& $4$&$368 \cdot 10^{-2}$	&	1	&	-0.5	 		\\
%2	& $6$&$709 \cdot 10^{-1}$	&	1	&	0.5	 		\\
%3	& $-1$&$766$				&	1	&	1	&		\\
%4	& $8$&$582 \cdot 10^{-1}$	&	2	&	0.5	&		\\
%5	& $-1$&$207$				&	2	&	1	&		\\
%6	& $5$&$120 \cdot 10^{-1}$	&	2	&	1.5	&		\\
%7	& $-4$&$000 \cdot 10^{-4}$	&	2	&	4.5	&		\\
%8	& $-1$&$248 \cdot 10^{-2}$	&	3	&	0	&		\\
%9	& $3$&$100 \cdot 10^{-2}$	&	4	&	1	&		\\
%10	& $1$&$755 \cdot 10^{-3}$	&	4	&	3	&		\\
%11	& $-3$&$172 \cdot 10^{-6}$	&	8	&	1	&		\\
%12	& $-2$&$240 \cdot 10^{-6}$	&	9	&	3	&		\\
%13	& $2$&$947 \cdot 10^{-7}$	&	10	&	3	&		\\
%14	& $1$&$830 \cdot 10^{-10}$	&	1	&	0	&	1	\\
%15	& $1$&$512 \cdot 10^{-1}$	&	1	&	1	&	1	\\
%16	& $-4$&$289 \cdot 10^{-1}$	&	1	&	2	&	1	\\
%17	& $6$&$894 \cdot 10^{-2}$	&	2	&	0	&	1	\\
%18	& $-1$&$408 \cdot 10^{-2}$	&	4	&	0	&	1	\\
%19	& $-3$&$063 \cdot 10^{-2}$	&	5	&	2	&	1	\\
%20	& $-2$&$970 \cdot 10^{-2}$	&	6	&	2	&	1	\\
%21	& $-1$&$932 \cdot 10^{-2}$	&	1	&	5	&	2	\\
%22	& $-1$&$106 \cdot 10^{-1}$	&	2	&	5	&	2	\\
%23	& $9$&$953 \cdot 10^{-2}$	&	3	&	5	&	2	\\
%24	& $8$&$548 \cdot 10^{-3}$	&	4	&	2	&	2	\\
%25	& $-6$&$151 \cdot 10^{-2}$	&	4	&	4	&	2	\\
%26	& $-4$&$292 \cdot 10^{-2}$	&	3	&	12	&	3	\\
%27	& $-1$&$813 \cdot 10^{-2}$	&	5	&	8	&	3	\\
%28	& $3$&$446 \cdot 10^{-2}$	&	5	&	10	&	3	\\
%29	& $-2$&$386 \cdot 10^{-3}$	&	8	&	10	&	3	\\
%30	& $-1$&$159 \cdot 10^{-2}$	&	2	&	10	&	4	\\
%31	& $6$&$642 \cdot 10^{-2}$	&	3	&	14	&	4	\\
%32	& $-2$&$372 \cdot 10^{-2}$	&	4	&	12	&	4	\\
%33	& $-3$&$962 \cdot 10^{-2}$	&	4	&	18	&	4	\\
%34	& $-1$&$387 \cdot 10^{-2}$	&	4	&	22	&	4	\\
%35	& $3$&$389 \cdot 10^{-2}$	&	5	&	18	&	4	\\
%36	& $-2$&$927 \cdot 10^{-3}$	&	6	&	14	&	4	\\
%37	& $9$&$325 \cdot 10^{-5}$	&	2	&	2	&		\\
%38	& $-6$&$287$				&	0	&	0	&		\\
%39	& $1$&$271 \cdot 10^{1}$	&	0	&	1	&		\\
%40	& $-6$&$424$				&	0	&	2	&		\\
%\bottomrule
%\end{tabular}
%\end{center}
%\end{table}


\begin{table}[h]
\caption{\label{tab-fhe-ch4-resi-b}Values for the coefficients $n_i$, $d_i$, $t_i$ and $c_i$ from \eqref{eq-fhe-c} for methane.}
\begin{center}
\begin{tabular}{lr@{.}lccc|lr@{.}lccc}
\toprule
i	& \multicolumn{2}{c}{$n_i$}	& $d_i$	& $t_i$	& $c_i$	& i	& \multicolumn{2}{c}{$n_i$}	& $d_i$	& $t_i$	& $c_i$\\
\midrule									
1	& $4$&$368 \cdot 10^{-2}$	&	1	&	-0.5&	 	& 21	& $-1$&$932 \cdot 10^{-2}$	&	1	&	5	&	2	\\
2	& $6$&$709 \cdot 10^{-1}$	&	1	&	0.5	& 		& 22	& $-1$&$106 \cdot 10^{-1}$	&	2	&	5	&	2	\\
3	& $-1$&$766$				&	1	&	1	&		& 23	& $9$&$953 \cdot 10^{-2}$	&	3	&	5	&	2	\\
4	& $8$&$582 \cdot 10^{-1}$	&	2	&	0.5	&		& 24	& $8$&$548 \cdot 10^{-3}$	&	4	&	2	&	2	\\
5	& $-1$&$207$				&	2	&	1	&		& 25	& $-6$&$151 \cdot 10^{-2}$	&	4	&	4	&	2	\\
6	& $5$&$120 \cdot 10^{-1}$	&	2	&	1.5	&		& 26	& $-4$&$292 \cdot 10^{-2}$	&	3	&	12	&	3	\\
7	& $-4$&$000 \cdot 10^{-4}$	&	2	&	4.5	&		& 27	& $-1$&$813 \cdot 10^{-2}$	&	5	&	8	&	3	\\
8	& $-1$&$248 \cdot 10^{-2}$	&	3	&	0	&		& 28	& $3$&$446 \cdot 10^{-2}$	&	5	&	10	&	3	\\
9	& $3$&$100 \cdot 10^{-2}$	&	4	&	1	&		& 29	& $-2$&$386 \cdot 10^{-3}$	&	8	&	10	&	3	\\
10	& $1$&$755 \cdot 10^{-3}$	&	4	&	3	&		& 30	& $-1$&$159 \cdot 10^{-2}$	&	2	&	10	&	4	\\
11	& $-3$&$172 \cdot 10^{-6}$	&	8	&	1	&		& 31	& $6$&$642 \cdot 10^{-2}$	&	3	&	14	&	4	\\
12	& $-2$&$240 \cdot 10^{-6}$	&	9	&	3	&		& 32	& $-2$&$372 \cdot 10^{-2}$	&	4	&	12	&	4	\\
13	& $2$&$947 \cdot 10^{-7}$	&	10	&	3	&		& 33	& $-3$&$962 \cdot 10^{-2}$	&	4	&	18	&	4	\\
14	& $1$&$830 \cdot 10^{-10}$	&	1	&	0	&	1	& 34	& $-1$&$387 \cdot 10^{-2}$	&	4	&	22	&	4	\\
15	& $1$&$512 \cdot 10^{-1}$	&	1	&	1	&	1	& 35	& $3$&$389 \cdot 10^{-2}$	&	5	&	18	&	4	\\
16	& $-4$&$289 \cdot 10^{-1}$	&	1	&	2	&	1	& 36	& $-2$&$927 \cdot 10^{-3}$	&	6	&	14	&	4	\\
17	& $6$&$894 \cdot 10^{-2}$	&	2	&	0	&	1	& 37	& $9$&$325 \cdot 10^{-5}$	&	2	&	2	&		\\
18	& $-1$&$408 \cdot 10^{-2}$	&	4	&	0	&	1	& 38	& $-6$&$287$				&	0	&	0	&		\\
19	& $-3$&$063 \cdot 10^{-2}$	&	5	&	2	&	1	& 39	& $1$&$271 \cdot 10^{1}$	&	0	&	1	&		\\
20	& $-2$&$970 \cdot 10^{-2}$	&	6	&	2	&	1	& 40	& $-6$&$424$				&	0	&	2	&		\\
\bottomrule
\end{tabular}
\end{center}
\end{table}

%++++++++++++++++++++++++++++++++++++++++++++++++++++++++++++++++++++++++++++++++++++++++++++++++++++++++++++++++++++++++++++++++++++++++

\clearpage
\subsection{Nitrogen}
For the limit $k$ of the summator in \eqref{eq-fhe-c} the following values are given for nitrogen:
\begin{compactitem}
	\item $k_1=6$
	\item $k_2=32$
	\item $k_3=36$
\end{compactitem}

\begin{table}[h]
\caption{\label{tab-fhe-n2-ideal}Values for $n_i^o$ and $\gamma_i^o$ in \eqref{eq-fhe-b} for nitrogen.}
\begin{center}
\begin{tabular}{lr@{.}lr@{.}l}
\toprule
i	& \multicolumn{2}{c}{$n_i^o$}	& \multicolumn{2}{c}{$\gamma_i^o$}	\\
\midrule					
1	& $9$&$912\,644$			& \multicolumn{2}{c}{}	\\
2	& $-6$&$333\,133$			& \multicolumn{2}{c}{}	\\
3	& $3$&$001\,6$				& \multicolumn{2}{c}{}	\\
4	& $0$&$008\,449$			& $3$&$400\,664$	\\
5	& $4$&$694\,2$				& $10$&$270\,22$	\\
%6	& $3$&$486\,5$				& $20$&$440�?,72$	\\ %Fehler
7	& $1$&$657\,2$				& $29$&$939\,49$	\\
8	& $1$&$411\,5$				& $79$&$138\,92$	\\
\bottomrule
\end{tabular}
\end{center}
\end{table}

\begin{table}[h]
\caption{\label{tab-fhe-n2-resi-a}Values for the coefficients $\alpha_i$, $\beta_i$, $\gamma_i$ and $\epsilon_i$ from \eqref{eq-fhe-c} and \eqref{eq-fhe-d} for nitrogen.}
\begin{center}
\begin{tabular}{lcccc}
\toprule
i	& $\alpha_i$	& $\beta_i$	& $\gamma_i$	& $\epsilon_i$	\\
\midrule								
33	& 20			& 325		& 1.16			& 1	\\
34	& 20			& 325		& 1.16			& 1	\\
35	& 15			& 300		& 1.13			& 1	\\
36	& 25			& 275		& 1.25			& 1	\\
\bottomrule
\end{tabular}
\end{center}
\end{table}

\begin{table}
\caption{\label{tab-fhe-n2-resi-b}Values for the coefficients $n_i$, $d_i$, $t_i$ and $c_i$ from \eqref{eq-fhe-c} for nitrogen.}
\begin{center}
\begin{tabular}{lr@{.}lccc}
\toprule
i	& \multicolumn{2}{c}{$n_i$}							& $d_i$	& $t_i$		& $c_i$	\\
\midrule								
1	& $0$&$924\,803\,575\,275$							&	 1	&	0.25	&	 0	\\
2	& $-0$&$492\,448\,489\,428$							&	 1	&	875		&	 0	\\
3	& $0$&$661\,883\,336\,938$							&	 2	&	0.5		&	 0	\\
4	& $-0$&$192\,902\,649\,201 \cdot 10^1$				&	 2	&	875		&	 0	\\
5	& $-0$&$622\,469\,309\,629 \cdot 10^{-1}$			&	 3	&	375		&	 0	\\
6	& $0$&$349\,943\,957\,581$							&	 3	&	0.75	&	 0	\\
7	& $0$&$564\,857\,472\,498$							&	 1	&	0.5		&	 1	\\		
8	& $-0$&$161\,720\,005\,987 \cdot 10^1$				&	 1	&	0.75	&	 1	\\
9	& $-0$&$481\,395\,031\,883$						 	&	 1	&	2		&	 1	\\
10	& $0$&$421\,150\,636\,384$							&	 3	&	1.25	&	 1	\\
11	& $-0$&$161\,962\,230\,825 \cdot 10^{-1}$			&	3	&	3.5		&	 1	\\
12	& $0$&$172\,100\,994\,165$							&	4	&	1		&	 1	\\
13	& $0$&$735\,448\,924\,933 \cdot 10^{-2}$			&	6	&	0.5		&	 1	\\
14	& $0$&$168\,077\,305\,479 \cdot 10^{-1}$			&	6	&	3		&	 1	\\
15	& $-0$&$107\,626\,664\,179 \cdot 10^{-2}$			&	7	&	0		&	 1	\\
16	& $-0$&$137\,318\,088\,513 \cdot 10^{-1}$			&	7	&	2.75	&	 1	\\
17	& $0$&$635\,466\,899\,859 \cdot 10^{-3}$			&	8	&	0.75	&	 1	\\
18	& $0$&$304\,432\,279\,419 \cdot 10^{-2}$			&	8	&	2.5		&	 1	\\
19	& $-0$&$435\,762\,336\,045 \cdot 10^{-1}$			&	1	&	4		&	 2	\\
20	& $-0$&$723\,174\,889\,316 \cdot 10^{-1}$			&	2	&	6		&	 2	\\
21	& $0$&$389\,644\,315\,272 \cdot 10^{-1}$			&	3	&	6		&	 2	\\
22	& $-0$&$212\,201\,363\,910 \cdot 10^{-1}$			&	4	&	3		&	 2	\\
23	& $0$&$408\,822\,981\,509 \cdot 10^{-2}$			&	5	&	3		&	 2	\\
24	& $-0$&$551\,990\,017\,984 \cdot 10^{-4}$			&	8	&	6		&	 2	\\
25	& $-0$&$462\,016\,716\,479 \cdot 10^{-1}$			&	4	&	16		&	 3	\\
26	& $-0$&$300\,311\,716\,011 \cdot 10^{-2}$			&	5	&	11		&	 3	\\
27	& $0$&$368\,825\,891\,208 \cdot 10^{-1}$			&	5	&	15		&	 3	\\
28	& $-0$&$255\,856\,846\,220 \cdot 10^{-2}$			&	8	&	12		&	 3	\\
29	& $0$&$896\,915\,264\,558 \cdot 10^{-2}$			&	3	&	12		&	 4	\\
30	& $-0$&$441\,513\,370\,350 \cdot 10^{-2}$			&	5	&	7		&	 4	\\
31	& $0$&$133\,722\,924\,858 \cdot 10^{-2}$			&	6	&	4		&	 4	\\
32	& $0$&$264\,832\,491\,957 \cdot 10^{-3}$			&	9	&	16		&	 4	\\
33	& $0$&$196\,688\,194\,015 \cdot 10^2$				&	1	&	0		&	 2	\\
34	& $-0$&$209\,115\,600\,730 \cdot 10^2$				&	1	&	1		&	 2	\\
35	& $0$&$167\,788\,306\,989 \cdot 10^{-1}$			&	3	&	2		&	 2	\\
36	& $0$&$262\,767\,566\,274 \cdot 10^4$				&	2	&	3		&	 2	\\
\bottomrule
\end{tabular}
\end{center}
\end{table}







