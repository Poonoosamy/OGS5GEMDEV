\subsubsection*{Theory}

Radioactive decay is the change in the composition of a core by emitting particles and/or electro-magnetic radiation. Different kinds of radioactive decay are i.e. decay as a result of emission of negatrons or positrons and decay under emission of $\gamma$-rays.

\begin{tabbing}
\=xxxxxxxxxxxxxxxxxxxxxxx \=xxxxxxxxxxxxxxxxx \kill
\> $\alpha$-decay
\> ${^{232}_{\phantom{2}90}}\mathrm{Th}\rightarrow\;{^{228}_{\phantom{2}88}}\mathrm{Ra}+{^{4}_{2}}\mathrm{He}$ \\[1.5ex]
\> $\beta$-decay
\> ${^{228}_{\phantom{2}88}}\mathrm{Ra}\rightarrow\;{^{228}_{\phantom{2}89}}\mathrm{Ac}+{^{\phantom{-}0}_{-1}}\mathrm{e}$ \\[1.5ex]
\> $\gamma$-decay
\> ${^{236}_{\phantom{2}92}}\mathrm{U}^{\star}\rightarrow\;{^{236}_{\phantom{2}92}}\mathrm{U}+\gamma$
\end{tabbing}

The above given examples show that the radioactive decay is an irreversible process. The following differential equation describes the decay as first order reaction (without chain development):
\begin{equation}
\frac{\partial C}{\partial t}\,=\,-\lambda\cdot C
\label{eq55}
\end{equation}
{\small
with $\lambda$ - decay rate (s$^{-1}$).
}

The integration of this equation causes an exponential decay term in the following form.
\begin{equation}
C(t)\,=\,C_0\cdot e^{-\lambda\cdot t}
\label{eq56}
\end{equation}
{\small
with $C_0$ - initial concentration (kg$\cdot$m$^{-3}$).
}

The decay values are commonly expressed as the so-called half life ($t_{1/2}$). This is the point of time when half of the substance is degraded. The relation between the half-life $T$ and the decay rate results from:
\begin{equation}
e^{-\lambda\cdot t}\,=\,\frac{1}{2}\;\Rightarrow\;
\lambda\,=\,\frac{\ln(2)}{T}\,\cong\,\frac{0.693}{T}
\label{eq57}
\end{equation}
