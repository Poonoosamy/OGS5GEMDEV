\section{Theory}

The mass transport in a homogeneous, saturated aquifer can be influenced by convection, diffusion, decay and biodegradation, sorption and chemical reactions. For a steady state 1-dimensional flow through a homogeneous isotropic medium with constant material parameters the following differential equation \ref{eq51} is applied.
\begin{equation}
\frac{\partial C}{\partial t}\,+\, \frac{\rho_b}{R}\cdot\frac{\partial S}{\partial t}\,+\, \frac{q}{R}\cdot\frac{\partial C}{\partial x}\,=\,
D_{xx}\cdot\frac{\partial^2C}{\partial x^2}\,-\,\lambda\cdot C
\label{eq51}
\end{equation}
{\small
with
\begin{itemize}
\item[$C$] -- dissolved concentration (kg$\cdot$m$^{-3}$),
\item[$S$] -- sorbed concentration(kg$\cdot$kg$^{-1}$),
\item[$t$] -- time (s),
\item[$\rho_b$] -- bulk density (kg$\cdot$m$^{-3}$),
\item[$R$] -- retardation factor (-),
\item[$q$] -- flow rate (m$\cdot$s$^{-1}$),
\item[$x$] -- distance (m),
\item[$D_{xx}$] -- dispersion coefficient in x-direction (m$^2\cdot$s$^{-1}$),
\item[$\lambda$] -- decay rate (s$^{-1}$).
\end{itemize}
}

This equation is used to calculate the concentration distribution under consideration of decay and sorption with the linear Henry-isotherm. The retardation coefficient $R$ for the Henry isotherm is related to the Henry sorption coefficient $K_D$ in the following way (equ. \ref{eq52}).
\begin{equation}
R\,=\,1\,+\,\frac{\rho_b}{\Phi}\,K_D\,=\,
1\,+\,\frac{1-\Theta}{\Phi}\,\rho_s K_D
\label{eq52}
\end{equation}
{\small
with
\begin{itemize}
\item[$\Phi$] -- porosity (-),
\item[$\rho_s$] -- density (kg$\cdot$m$^{-3}$),
\end{itemize}
with the initial and boundary conditions
\begin{displaymath}
C(x,t=0)\,=\,C_I\quad\forall x
\end{displaymath}
with $C_I$ - concentration at time $I$.
\begin{displaymath}
C(x=0,t)\,=\,C_0\quad\forall t,\qquad
\frac{\partial C}{\partial x}(x\rightarrow\infty,t)\,=\,C_I
\quad\forall t>0
\end{displaymath}
with $C_0$ - initial concentration.
}

The following analytical solution is significant:
\begin{eqnarray}
C=C_1&+&\left(C_0-C_1\right)\cdot\frac{1}{2}
\mbox{\hspace{-1.5ex}}\cdot\mbox{\hspace{-1.5ex}}
\left[
\exp
\left(
\frac{v\cdot x(1-\gamma)}{2\cdot D_{xx}}
\right)
\cdot\mathrm{erfc}
\left(
\frac{x-v\cdot\gamma\cdot t/R}{2\cdot \sqrt{D_{xx}\cdot t/R}}
\right)
\right. \nonumber\\[2.5ex]
& &
+\left.
\exp
\left(
\frac{v\cdot x(1+\gamma)}{2\cdot D_{xx}}
\right)
\cdot\mathrm{erfc}
\left(
\frac{x+v\cdot\gamma\cdot t/R}{2\cdot \sqrt{D_{xx}\cdot t/R}}
\right)
\right]
\label{eq53}
\end{eqnarray}
{\small
with $v$ - velocity
}

\begin{equation}
\gamma\,=\,
\sqrt{1+4\cdot\lambda\cdot R\cdot D_{xx}/v^2}\,.
\label{eq54}
\end{equation}

Equation \ref{eq53} is the basis for the verification of the RockFlow-simulation results for the 1-dimensional mass transport. All described equations and all analytical solutions of equation \ref{eq53} are taken from \cite{Hab:01}. 