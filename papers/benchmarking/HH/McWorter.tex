%-------------------------------------------------------------------------
\subsection{\upshape\textbf{McWhorter problem}\label{MWPIsoTwoPhaseFlow}}
\subsubsection*{\upshape\textbf{Theory}}
It is assumed that the flow of both wetting and non-wetting phases can be adequately described by the Darcy's law if the phases are immiscible and incompressible.
\begin{equation}
n\frac{\partial S^{\gamma}}{\partial t} + \nabla \cdot \mathbf{q}^{\gamma} = 0, \gamma=w, nw
\label{eq:mcwtMassEq}
\end{equation}
\begin{equation}
\mathbf{q}^{\gamma}=-{\mathbf K} \lambda^{\gamma} \nabla p^{\gamma}
\label{eq:mcwtFluxEq}
\end{equation}
where $\lambda_w$ and  $\lambda_{nw}$ are mobility of wetting and non-wetting fluid. Both phase are linked by the state equation $S_w+S_{nw}=1$ and $p_c=p_g-p_w$. Here total flux, $\mathbf {q}_t=\mathbf {q}_w + \mathbf {q}_{nw}$ once $p_c$ is function of the $S_w$.


A formulation that is often used for two phase flow problem is so called fractional flow model. One of the attractive of this formulation is that the model become more accessible to analysis. Subtracting equation ($18.1.24$) for both phases we have.
\begin{equation}
\mathbf {q}_w=f \mathbf {q}_t- D \frac{\partial S_w}{\partial x}
\label{eq:McWhorterWetFlux}
\end{equation}
where 
\begin{equation*}
f=\frac{1}{1 + \frac{\lambda_{nw}}{\lambda_w}},~~~D=-\lambda_{nw} f \frac{\partial p_c}{\partial S_w}
\end{equation*}
First term on the right of equation (\ref{eq:McWhorterWetFlux}) is dicteted by rate at which flux is injected on the boundary and second term represent the addition impelling force due to gradient of capillary pressure. Put equation (\ref{eq:McWhorterWetFlux}) in equation ($18.1.23$) for wetting phase and assume that total flux, $\mathbf q_t$ is space invariant.
\begin{equation}
\frac{\partial }{\partial x}\left( D\frac{\partial S_w}{\partial x}\right) - \mathbf q_t \frac{\partial f}{\partial S_w}\frac{\partial S_w}{\partial x}=n \frac{\partial S_w}{\partial t}
\label{eq:McWhorterAnal}
\end{equation}
In the last benchmark (Buckley and Leveret) it is assume that force due to gradient of capillary pressure is very small as consequence of total flux, $\mathbf q_t$ is large hence suppressed the second order term in the equation.


Knowing  the capillarity effect, model verification need a comparison with an analytical solution based on one by McWhorter and Sunada ($1990$). They developed an exact quasi-analytical solution of equation (\ref{eq:McWhorterAnal}) for unidirectional displacement where non-wetting phase by wetting phase using the concept of a fractional flow function.


The fractional flow function is defined as ratio of wetting phase flux, $\mathbf q_w$ to the total flux, $\mathbf q_t$. It has shown that this ratio is function of $S_w$ only, when $\mathbf q_t$ is inversely related to square root of the time.
\newpage
\textbf{Problem definition}\\
The test benchmark problem for capillary effects is formulated as if the instantaneous displacement occurs in one-dimensional horizontal reservoir initially occupied by oil. Solution has been obtain through solving the governing equations ($18.1.12$) and ($18.1.13$) by pressure-pressure scheme described in sec (sec.\ref{sec:pp-scheme}). Different from the Buckley-Leverett problem, here flow is governed by capillary force when water saturation at the left end of the horizontal column is kept to be one, while the right end is kept to be no flux at all. So for no source term is accounted.
\vspace{-0.3cm}
\begin{figure}[H]
\begin{center}
\epsfig{figure=HH/figures/Schematic.eps,height=3cm}
\end{center}
\vspace{-0.6cm}
\caption{Schematic of the benchmark formulated to test McWhorter and Sunada's analytical solution.}
\label{mcwt:config}
\end{figure}
\vspace{-0.4cm}
\textbf{Results}\\
Based on the above discussion GeoSys produces agreeable solution. Fig. $18.1.10$ shows water saturation profile, $S_w$ with a fine grid along with $2.6m$ long horizontal column for different time steps. Line elements has been used with time and space discretization $\delta t=0.5s$ and $\delta x=0.05m$ respectively.
\begin{figure}[H]
\begin{center}
\epsfig{figure=HH/figures/pp_1d.eps,height=5cm}
\end{center}
\vspace{-0.4cm}
\caption{Water saturation, $S_w$ profile of the present result along with analytical solution based on one by McWhorter.}
\label{mcwt:ppModel}
\end{figure}
Here, we have solved exactly same problem using the total-pressure-based saturation model in sequential iterative coupling scheme.
\begin{figure}[H]
\begin{center}
\epsfig{figure=HH/figures/TPSMcWhorter.eps,height=5cm}
\end{center}
\caption{Water saturation, $S_w$ profile in sequential iterative coupling scheme.}
\label{mcwt:psModel}
\end{figure}
Unlike the pressure-pressure model, one downside for the total-pressure-based saturation model is less accurate for the problems dominated by capillarity (see Fig. $18.1.11$). Since the pressure-pressure model directly solves for capillary pressure as a primary variable, the model has an advantage for the capillary related problems. On the other hand, the total-pressure-based saturation model is limited to the problems when $d P_c/d S_w$ is close to zero. The condition for $d P_c/d S_w$ close to zero caused physically in the cases of fractures, shear zones and transitions between heterogeneities.
\begin{table}[!htb]
\begin{tabular}{lccr}
\hline\hline\noalign{\smallskip}
Property & Symbol & Value & Unit \\
\noalign{\smallskip}\hline\noalign{\smallskip}
Column length & $L$ & $m$ & $2.6$  \\
wetting dynamic viscosity &  $\mu_w$ & $Pa.s$ & $1.0\times10^{-3}$ \\
non-wetting dynamic viscosity & $\mu_{nw}$ & $Pa.s$ & $1.0\times10^{-3}$ \\
wetting phase density &  $\rho_w$ &$kg.m^{-3}$ & $1.0\times10^{3}$ \\
Non-wetting phase density &  $\rho_{nw}$ & $kg.m^{-3}$ & $1.0\times10^{3}$ \\
Permeability & $\mathbf K$ & $ m^2$ & $1.0\times 10^{-10}$ \\
Porosity & $n$ & $--$ & $3.0\times10^{-1}$ \\
Residual saturation of water &  $S_{rw}$ & $--$ & $0$ \\
Residual saturation of oil &  $S_{nrw}$ & $--$ & $0$ \\
Entry pressure &  $p_d$ & $Pa$ & $5.0\times10^{3}$ \\
Soil distribution index &  $\lambda$ & $--$ & $2.0$ \\
Capillary pressure & $p^c(S_{eff})$ & $Pa$ & Brooks-Corey model\\
Relative permeability & $\kappa_{rel}(S_{eff})$ & $--$ & Brooks-Corey model \\
\noalign{\smallskip}\hline\hline
\end{tabular}
\caption{Material parameters for the McWhorter problem.}
\end{table}
\clearpage
\subsubsection*{\upshape\textbf{Benchmark deposit}}
\begin{tabular}{|l|l|l|}
\hline
Benchmark & Problem type & Path in benchmark deposit \\
\hline
\emph{mcwt}& H2 & McWhorter \\
\hline
\end{tabular}
\clearpage
