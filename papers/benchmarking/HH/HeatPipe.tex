%-------------------------------------------------------------------------
\subsection{\upshape\textbf{Heat Pipe problem}\label{HPPNonIsoTwoPhaseFlow}}
\subsubsection*{\upshape\textbf{Theory}}
When an unsaturated porous medium subjected to a constant heat flux and the temperature is sufficiently high, water is heated and vaporizes. Vapor flows under its pressure gradient towards cooler end where it condenses. Vaporization and condensation produce a liquid saturation gradient, creating a capillary pressure gradient inside the porous medium. Condensate flows towards the hot end under the influence of a capillary pressure gradient. This is a heat pipe in an unsaturated porous medium


Udell and Fitch derived the pressure gradient of the each phases in two-phase flow with heat transfer. The generalized form of the Darcy's law is used to calculate velocity fields. 
\begin{equation}
\frac{d p^g}{d x} = \frac{\eta q \nu^g}{\mathbf k k_{\mathrm {rg}} H_{\mathrm {vap}}}
\label{eq:HP1}
\end{equation}
\begin{equation}
\frac{d p^l}{d x} =- \frac{\eta q \nu^l}{\mathbf k k_{\mathrm {rl}} H_{\mathrm {vap}}}
\label{eq:HP2}
\end{equation}
where, $\eta$ is the ratio of heat transport caused by convection to the total heat-flux $q$ (see Helming [1997]). $p$ is phase pressure; $\nu^\gamma=\frac{\mu\gamma}{\rho^\gamma}$; $x$ is space coordinate in the x-direction; $\mathbf k$ is intrinsic permeability; $k_{r\gamma}$ is relative permeability and $H_{\mathrm {vap}}$ is latent heat of water. $\gamma$ is the phase superscript and $g, l$ stand for gas and liquid phase, respectively. Gas pressure is the sum of two partial pressure, i.e. $p^g=p^g_a+p^g_w$.

The density of gas phase is sum of air and vapor density. Air density is according to ideal gas equation, i.e.
\begin{equation}
\rho_{\mathrm {ga}}=\frac{M_a p_a}{RT} 
\label{eq:HP3}
\end{equation}
The energy transport is described by Zhou et al. [1990] as
\begin{equation}
q=-\kappa_{\mathrm {app}}\frac{\partial d T}{d x} + \dot m_{\mathrm {vap}} H_{\mathrm {vap}}
\label{eq:HP4}
\end{equation}
where, $T$ is temperature, $\kappa_{\mathrm {app}}$ is apparent thermal conductivity.

Since capillary pressure is the difference of phase pressure, hence from Eq. 1, the capillary pressure gradient is
\begin{equation}
\frac{d p^c}{d x} = \frac{\eta q}{\mathbf k H_{\mathrm {vap}}}\left[\frac{\nu^g}{k_{\mathrm {rg}}} + \frac{\nu^l}{k_{\mathrm {rl}}}\right]
\label{eq:HP5}
\end{equation}
Brooks-Corey has presented water saturation -capillary pressure relation in the following form
\begin{equation}
S=\left(\frac{Pd}{p^c}\right)^\lambda
\label{eq:HP6}
\end{equation}
By comparing this with Leverett's [1941] non-dimensional form we get $Pd=\sigma_0\left(\frac{n}{\mathbf k}\right)^{0.5}$ and $n$ is medium porosity. $\sigma_0$ is interfacial tension at reference temperature $T_0$. Here, $S$ is scaled as following 
\begin{equation}
S=\frac{S_{\mathrm {w}}-S_{\mathrm {lr}}}{1-S_{\mathrm {lr}}-S_{\mathrm {gr}}}
\label{eq:HP7}
\end{equation}
The constant $S_{\mathrm {lr}}; S_{\mathrm {gr}}$ are residual saturations. And for interfacial tension we have used following correlation given by Olivella and Gens[2000].
\begin{equation}
\sigma( T)={0.3258C^{1.256}} - {0.148C^{2.256}};~~ T\le 633.15 \mathrm K
\label{eq:surface_tension}
\end{equation}
where, $C=1.0-\frac{T}{647.3~K}$\\

Brooks-Corey relative permeabilities relation are 
\begin{equation}
\mathbf k_{\mathrm {rg}}=\left(1-S\right)^2 \left(1-S^{\frac{2+\lambda}{\lambda}}\right);~~~\mathbf k_{rl}=S^{\frac{2+3\lambda}{\lambda}}
\label{eq:HP8}
\end{equation}
Using Eqs. (\ref{eq:HP5}-\ref{eq:HP6}), we can write following forms of saturation gradient.
\begin{equation}
\frac{d S}{d x}=\frac{S^{1.5}}{P_d}\frac{2\eta q}{\mathbf k H_{\mathrm {vap}}}\left[\frac{\nu^g}{k_{\mathrm {rg}}} + \frac{\nu^l}{k_{\mathrm {rl}}}\right]
\label{eq:HP9}
\end{equation}
Now Eq. (\ref{eq:HP9}) is integrated over two-phase zone. Where two-phase zone can be defined by imposing the limits of integration (see Udell [1985]): $S=S_0$ at $x=0$ and $S=S_1$ at $x=L$.

The saturation vapor density $\rho_{\mathrm {sat}}$, is depending on temperature, and estimated by following relation
\begin{equation}
\rho_{\mathrm {sat}}=1.0\times10^{-3}\exp\left(a-\frac{b}{T}\right)
\label{eq:HP10}
\end{equation}
where, constants $a=19.81$ and $b=4975.9$.

In the porous medium, we must account for a decrease in vapor density due to capillarity. The amount of decrease in vapor density is describe by Kelvin equation as follow
\begin{equation}
\rho_{\mathrm {gw}}=\rho_{\mathrm {sat}}\exp\left(-\frac{M_{\mathrm w} p^c}{\rho^l RT}\right)
\label{eq:HP11}
\end{equation}
where $M_{\mathrm w}$ is water molecular weight; $\rho^l$ is liquid density and $R$ is universal gas constant. From Eqs. (\ref{eq:HP10}-\ref{eq:HP11}), we get temperature as function of vapor density and capillary pressure as
\begin{equation}
T=\frac{A}{B}
\label{eq:HP12}
\end{equation}
where
\begin{equation*}
 A=b+\frac{M_{\mathrm w} p^c}{\rho^l R}; B=a-3 -\log\left(\rho_{\mathrm {gw}}\right)
 \label{eq:HP20}
\end{equation*}


$\rho_{\mathrm {gw}}$ is changing with temperature which makes difficulty for temperature calculation. Hence we need to know temperature gradient which is possible by Eq. (\ref{eq:HP12}) along with vapor pressure gradient 
\begin{equation}
\frac{d p_{\mathrm{gw}}}{d x} = \frac{\eta q \nu^g_w}{\mathbf k k_{\mathrm {rg}} H_{\mathrm {vap}}}
\label{eq:HP18}
\end{equation}
Form of the temperature gradient
\begin{equation}
\frac{d T}{d x}=\frac{\frac{B M_{\mathrm w}}{\rho^l R} \frac{d p^c}{d x} + \frac{A}{p_{\mathrm{gw}}} \frac{d p_{\mathrm{gw}}}{d x}}{B^2+\frac{A}{T}}
\label{eq:HP13}
\end{equation}
Apparent thermal conductivity can be get from heat flux divided by temperature gradient (see Udell [1985].


The coupled differential Eqs. (\ref{eq:HP1}), (\ref{eq:HP5}), (\ref{eq:HP9}) and (\ref{eq:HP13} ) are integrated by using Euler method with following boundary condition at $x=0$:
\begin{equation}
S=S_0;~~~ p^g=p^g_0;~~~p^c=p^c_0;~~~T=T_0
\label{eq:HP14}
\end{equation}
Material parameters are presented in Table \ref{tab:HP1}.\\

\textbf{Problem definition}\\\\
The test benchmark problem for heat pipe effects is formulated in one-dimensional. 
Horizontal column of length $2.6$~m is filled with fluid subjected to a constant heat flux at the right end where left end temperature maintained below to the saturation temperature.
\begin{figure}[htb]
\begin{center}
\epsfig{figure=HH/figures/Geo.eps,height=1.25cm}
\end{center}
\caption{Schematic of the benchmark.}
\label{Fig:HP1}
\end{figure}\\


\textbf{Results}\\\\
In order to establish the non-isothermal two-phase flow in the OpenGeoSys, we have verified numerical solutions with analytical results. Profile of water saturation $S_{\mathrm w}$, gas phase pressure $p^g$, liquid phase pressure $p^l$ and temperature $T$ are presented in Figs. \ref{Fig:HP2}, and \ref{Fig:HP4}. Found that numerical solutions are agreeable. Line elements has been used with variable time steps and a non uniform space discretization.
\begin{figure}[thbp]
\centerline{
\psfig{figure=HH/figures/S-Pc.eps,height=3.0in,width=3.0in}
\psfig{figure=HH/figures/Pw-Pg.eps,height=3.0in,width=3.0in}}
\caption{Comparison of water saturation and pressure profiles from present solution with analytical solution.}
\label{Fig:HP2}
\end{figure}
A finite element approach has been developed for the nonisothermal two-phase flow model based on the $ppT$ formulation. We used a combined monolithic/ staggered coupling scheme i.e. monolithic for the two-phase flow and staggered for the heat transport.
\begin{figure}[htb]
\begin{center}
\epsfig{figure=HH/figures/Tg.eps,height=8cm}
\end{center}
\caption{Comparison of temperature profile from present solution with analytical solution.}
\label{Fig:HP4}
\end{figure}
\begin{table}[htbp]
\caption{Material parameters for the heat pipe problem.}
\label{tab:HP1}
\begin{tabular}{l*{4}{l}r}
\hline
\textbf{Meaning} & \textbf{Symbol} &  \textbf{Value} &  \textbf{Unit} \\
\hline
Column length & $L$ & $\mathrm m$ & $2.6$  \\
Liquid dynamic viscosity &  $\mu^l$ & $\mathrm {Pa.s}$ & $1.0\times10^{-3}$ \\
Gas dynamic viscosity & $\mu^g$ & $\mathrm {Pa.s}$ & $1.0\times10^{-5}$ \\
Liquid density &  $\rho^l$ &$\mathrm {kg.m^{-3}}$ & $1.0\times10^{3}$ \\
Permeability & $\mathbf k$ & $ \mathrm {m^2}$ & $1.0\times 10^{-13}$ \\
Porosity & $n$ & $--$ & $0.3$ \\
Residual saturation of water &  $S_{\mathrm{rl}}$ & $--$ & $0.2$ \\
Residual saturation of oil &  $S_{\mathrm{rg}}$ & $--$ & $0$ \\
Soil distribution index &  $\lambda$ & $--$ & $2.0$ \\
Capillary pressure & $p^c(S)$ & $\mathrm {Pa}$ & Brooks-Corey model\\
Relative permeability & $\kappa_{\mathrm {r\gamma}}(S)$ & $--$ & Brooks-Corey model \\ \hline
\end{tabular}
\end{table}

\clearpage
\subsubsection*{\upshape\textbf{Benchmark deposit}}
\begin{tabular}{|l|l|l|}
\hline
Benchmark & Problem type & Path in benchmark deposit \\
\hline
\emph{HeatPipe}& H2 & MULTIPHASE \\
\hline
\end{tabular}
\clearpage
