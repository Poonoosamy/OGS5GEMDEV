\section{Introduction}

%\subsection{Motivation}
The numerical analysis of complex multi-field problems is an 
important issue for many engineering problems. A representative 
example is nuclear waste disposal. Nuclear waste repositories are 
constructed in deep geologic underground. Normally, the radioactive 
waste will generate heat for a long period of time with temperatures 
over 100$^\circ$C. Possibly, ground water \rev{flow} may be 
developed and gas will be produced due to the heating of the ground 
water. The coupling of thermal and hydraulic processes can cause 
mechanical damage in the near field of the host rock mass. To assess 
the safety of the underground repositories, the problem needs to be 
addressed as a thermo-hydro-mechanical
 (THM) coupled problem \cite{{LeSch:98},{deBoer:00},{deBoer:05},{StHuJi:02},{EhBl:02},{KolEtAl:2004a},{OKJO:04}}.
Although some commercial tools are already available, there is a 
tremendous demand in the development of fully coupled THM codes. 
Existing concepts couple obtainable codes which are specialized to 
hydraulic, mechanic or deformation problems. The coupling is then 
realized by data exchange between these codes. This procedure causes 
a rigorous restriction of the modeling of coupling phenomena. In 
this work we present a finite element class which can deal with 
thermal, hydraulic as well as mechanic problems.

%\subsection{Programming paradigms}
For the programming paradigms, there are two alternatives for finite 
element code design and development, i.e. procedure-oriented or 
object-oriented. The former does not encapsulate data and methods 
manipulating the data together, the latter does encapsulate data and 
methods and provides regulated communication between data and 
methods to perform tasks \cite{{StrB:00},{Budd:01}}. The object 
oriented paradigm \rev{facilitates} data abstract with its 
capabilities of data encapsulation, polymorphism and inheritance. 
Therefore, it provides us a easy way to develop and to maintain a 
code. This is a reason why more and more researchers are attracted 
to shift from using procedure oriented to  OOP paradigm in numerical 
analysis. Other reasons for its popularity is that software of 
increasing complexity has to be developed by permanently increasing 
programmer teams. OOP has significant advantages by allowing rapid 
software development through \rev{capsulation, inheritance and 
polymorph of data and methods}. 
%
The advantages of object-oriented programming for the development of 
engineering software was described in detail by \cite{Fen:90}.

%\subsection{History}
Although the fundamentals of object-orientated programming (OOP)
were established in the 1960s, it remains a very important concept
to face challenges in scientific computation, such as the solution
of coupled multi-field problems.
%
One of the first applications of the object-oriented paradigm to 
finite element analysis was published in 1990 \cite{FOFOST:90}, 
where essential components of finite element methods such as 
elements, nodes and materials were abstracted into classes. More 
efforts have been made by 
\cite{{FilDev:91},{Mack:92},{ZimYve:92},{YveZim:92}, 
{YveZim:93},{PidHud:93}, {RapKris:93}} in order to demonstrated the 
advantages of OOP over the procedure oriented programming. Moreover, 
the applications to many different physical problems have been 
investigated, such as linear stress analysis 
\cite{{ZimYve:92},{YveZim:92}, {YveZim:93}}, hypersonic shock waves 
\cite{BudPee:93}, structural dynamics \cite{PidHudli:93}, 2D Mises 
plasticity \cite{MenZim:93}, linear static problems 
\cite{{AdeFu:93},{AdeFu:95}}, electro-magnetics \cite{SilMes:94}, 
solidification process \cite{SaZa:99},  heat transfer as well as 
topological buildup \cite{RihKro94}. A process-oriented approach for 
the solution of multi-field problems in porous media is presented in 
\cite{{KolBau:04},{KolEtAl:2004a}}. Numerical objects for algebraic 
calculations in finite element analysis have been developed by 
\cite{{Scholz:92},{ZegHanAit:94},{LuWCD:95}}. \rev{In order to 
provide an automatic coding environment for finite element analysis, 
a symbolic code development concept is presented for the weak forms 
arising from the partial differential 
equations \cite{{ZimEyh:96},{EyhZim:96},{EyhZim:98},{EyhZim:01}}}.

Object design is the fundamental step in object-oriented
programming. The utilization of OOP to finite element analysis is
mainly focused on three aspects: (1) pre/post processing such as
mesh generation and graphical user interface, (2) linear algebra
and (3) finite element methods. In all these aspects, the core
object is the element object. The design of element objects is
associated with other objects corresponding to material
properties, numerical methods, local geometry and topology of
element etc. Specific material objects are described in the
most of references cited above.

In this part we present the design, implementation and application of 
object-orientation in finite element analysis for multi-physics 
problems. The development of a universal object for local 
finite element calculation and assembly is able to 
cope with different kinds of physical problems (i.e. different types 
of partial differential equations) and is, in particular, designed 
for strongly coupled problems. Additionally, object-orientation is used 
in description of mesh topology for the global assembly of system 
equations. The description of the programming semantics of these 
objects is given in C++. All the developments of this work are 
conducted within the framework of the scientific 
software project OGS \cite{geosys}. The the OO-FEM concept is verified by 
numerous test cases dealing with thermo-hydro-mechanical (THM) 
coupled problems in geotechnical as well as hydrological applications (Parts II and III of this book.)
