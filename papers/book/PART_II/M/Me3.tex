\subsection{Plane strain confined compression -- Excavation in heterogeneous media}
\label{subsec:Me3}

\subsubsection{Definition}
\label{subsubsec:Me3_def}

Differing from the homogeneous case, the deformation of the excavtion problem defined in Sec.~\ref{subsec:Me2} is analyzed herewith defining the initial conditions as functions of the coordinates, and assuming four different material domains (cf. Fig.~\ref{Me_fme:excav2}).
\begin{figure}[!thb]
  \begin{center}
  \epsfig{figure=PART_II/M/m_drift_init.eps,width=7cm, height=7cm}
  \end{center}
  \caption{Excavation in heterogeneous rock mass }
  \label{Me_fme:excav2}
\end{figure}

\subsubsection{Solution}
\label{subsubsec:Me3_sol}

The initial stresses are assumed to be linearly distributed within a material domain. The expressions of these distribution are given in Table~\ref{Me_tab:initialStress}.

As depicted in Fig. \ref{Me_fme:excav2}, the domain consists of four different materials denoted by 1, 2, 3 and 4. Within this context, only the Young's modulus is assumed to differ for the material domains under consideration (cf. Table~\ref{Me_tme:el2dHR}).

\clearpage

\begin{table}[!htb]
\centering
\caption{Initial stress distribution as function of coordinates (in kPa; material domains cf. Fig.~\ref{Me_fme:excav2})}
\label{Me_tab:initialStress}
\begin{tabular}{llll}
\toprule
Material & \multicolumn{3}{c}{Functions for stress coefficients} \\
domain   & $\sigma_{xx}$ & $\sigma_{yy}$ & $\sigma_{zz}$ \\
\noalign{\smallskip}\hline\noalign{\smallskip}
   1  & $-23.75-0.2y$        & $-23.75-0.2y$        & cf. $\sigma_{xx}$ \\
   2  & $-24.75-0.5y$        & $-24.75-1.3y$        & cf. $\sigma_{xx}$ \\
   3  & $-26.75-10.0x-12.0y$ & $-26.75-20.0x-16.0y$ & cf. $\sigma_{xx}$ \\
   4  & $-27.75-10.0x-14.0y$ & $-27.75-20.0x-18.0y$ & cf. $\sigma_{xx}$ \\
\bottomrule
\end{tabular}
\end{table}

\begin{table}[!htb]
\centering
\caption{Material parameters (different Young's moduli are given in the order of the material domains)}
\label{Me_tme:el2dHR}
\begin{tabular}{llll}
\toprule
Symbol & Parameter & Value & Unit \\
\midrule
$E$    & Young's modulus & $25.0$; $26.0$; $30.0$; $28.0$   & GPa \\
$\nu$  & Poisson's ratio & $0.3$  & -- \\
$\rho$ & Density         & $2500$ & kg$\cdot$m$^{-3}$ \\
\bottomrule
\end{tabular}
\end{table}

\subsubsection{Results}
\label{subsubsec:Me3_res}

Fig.~\ref{Me_fme:exH_disp} shows the distribution of displacements after excavation, and Fig.~\ref{Me_fme:exH_stress} shows the distribution of different coefficients of the stress tensor after excavation.
\begin{figure}[!thb]
  \begin{center}
  \epsfig{figure=PART_II/M/ex_h_ux.eps,width=6cm, height=6cm}
  \epsfig{figure=PART_II/M/ex_h_uy.eps,width=6cm, height=6cm}
  \end{center}
  \caption{Distribution of displacement (m)}
  \label{Me_fme:exH_disp}
\end{figure}
\begin{figure}[!thb]
  \begin{center}
  \epsfig{figure=PART_II/M/ex_h_sxx.eps,width=6cm, height=6cm}
  \epsfig{figure=PART_II/M/ex_h_sxy.eps,width=6cm, height=6cm}
  \epsfig{figure=PART_II/M/ex_h_syy.eps,width=6cm, height=6cm}
  \epsfig{figure=PART_II/M/ex_h_szz.eps,width=6cm, height=6cm}
  \end{center}
  \caption{Distribution of stresses (kPa)}
  \label{Me_fme:exH_stress}
\end{figure}
