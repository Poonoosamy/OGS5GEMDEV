\section{General finite element formulations}
\label{sec:fem}

The method of weighted residuals is applied to derive the weak
formulation of the balance equations given in section
\ref{sec:balance_equations}.

Assume $\TestS^{\mathrm{n}}\subset
H^1_{\scriptscriptstyle{\Gamma}}(\Omega)^{\mathrm{n}}$ is the test
function space. For all $\test\in \TestS^{\mathrm{1}}$, we have
the weak form of the mass balance equation (\ref{eqn:phase_mass_balance}) as
%
\begin{gather}
\int_\Omega 
\left(
nS^{\gamma} \frac{\partial\rho^{\gamma R}}{\partial t} + nS^{\gamma} \mio{v}{\gamma s}{} \cdot \nabla \rho^{\gamma R}
+
n\rho^{\gamma R}\frac{\partial S^{\gamma}}{\partial t} + n\rho^{\gamma R} \mio{v}{\gamma s}{} \cdot \nabla S^{\gamma}
\nonumber  %[1.5ex]
+
\nabla \cdot
(\rho^{\gamma}nS^{\gamma} \mio{v}{\gamma s}{} )
+S^{\gamma}\rho^{\gamma R}\nabla\cdot\mio{u}{s}{\dot} 
- q^{\gamma}
\right)
\test\dDom = 0
\label{eq:wkmass0}
\end{gather}
%
Applying integration by parts, divergence terms can be rewritten as
%
\begin{gather}
\int_\Omega \nabla\cdot\mio{A}{}{} \omega d\Omega
=
\int_\Omega \mio{A}{}{} \cdot \nabla\omega \dDom
+
\int_\Gamma \mio{A}{}{} \cdot \mathbf{n} \omega \dBdry
\label{eq:wkmass}
\end{gather}

Under the same assumption, the weak form of heat balance equation (\ref{eqn:energy_balance}) can be obtained as
\begin{gather}
\int_\Omega \sum_{\gamma}(\varepsilon^\alpha\rho^\alpha c_p^\alpha)\pD{T}{t}\test\dDom
-
\int_\Omega \miu{j}{\mathrm{th}}{} \cdot \nabla\omega \dDom
+
\int_\Gamma \miu{j}{\mathrm{th}}{} \cdot \mathbf{n} \omega \dBdry
-
\intD Q_{\mbox{\tiny T}}^{\gamma}\test\dDom=0
\label{eq:wkhTmass}
\end{gather}

Taking account of nonlinearity, the weak form of the momentum balance equation
(\ref{eq:4}) must be fulfilled throughout the load history, i.e.,
%
\begin{equation}
\intD \frac{1}{2}(\miu{\sigma}{\mathrm{eff}}{} - \sum_\gamma\sat^\gamma p^\gamma \I ):(\nabla \Test+(\nabla \Test)^{\mathrm T}) \dDom 
-
\intD\Test^{\mathrm T} \cdot \dens \grv \dDom  - \intB
\Test^{\mathrm T} \cdot {\mathbf{t}} \dBdry = 0 \label{eq:wkstress}
\end{equation}
for all $\Test\in \TestS^{\mathrm{n}}, \, \mathrm{n}=2,3$.
In principle, vector form of stress and strain tensor (cf. section \ref{sec:m_properties})
are used to developing  the system equation of the discretized form of (\ref{eq:wkstress}). Under this form,
 the constitutive law for the effective stress tensor can be expressed as
\[
    \miu{\sigma}{\mathrm{eff}}{} = \miu{C}{}{}\, \miu{\varepsilon}{}{}
\]
with the corresponding strain-displacement relationship
\[
    \miu{\varepsilon}{}{} = \mathcal{L} \,\mio{u}{s}{}
\]
where $\mathcal{L}$ is an differential operator.

\begin{equation}
\mathcal{L} =
\left(%
\begin{array}{ccc}
 \partial/\partial x & 0 & 0 \\
 0 & \partial/\partial y & 0 \\
 0 & 0 & \partial/\partial z \\
 \partial/\partial y & \partial/\partial x & 0 \\
 0 & \partial/\partial z & \partial/\partial y \\
 \partial/\partial z & 0 & \partial/\partial x
\end{array}%
\right)
\label{eqn:diffop}
\end{equation}

We use the Galerkin finite element method to solve the weak forms of
balance equations above. All variables are approximated by
admissible finite element functions in the Taylor-Hood finite
element space, i.e, low order interpolation $\Sh_1\in
\mathbb{R}^{\mathrm{n}}$ for pressure and temperature variables and
high order interpolation $\Sh_2 \in \mathbb{R}^{\mathrm{n}}$ for
displacement, respectively. As a result of the finite element
discretization of the weak forms (\ref{eq:wkmass}),
(\ref{eq:wkhTmass}) and (\ref{eq:wkstress}), we obtain local
matrices and vectors for the global system equations
\cite{Kol:02}. Element matrices and vectors can be classified
into following types (Table \ref{tab:types})

\begin{table}[H]
\centering
\begin{tabular}{lll}
\hline
Type & Name & Equations
\\
%-------------------------------------------------------------------------
\hline
$ \intD \Sh_1^{\mathrm{T}} \mathcal{M} \Sh_1 \dDom $ %\label{eq:massM} $
& Mass matrix
& (\ref{eq:wkmass}),(\ref{eq:wkhTmass})
\\
%-------------------------------------------------------------------------
$ \intD (\Sh_1)^{\mathrm{T}} \mathcal{M} \nabla \Sh_1 \dDom $ %\label{eq:LapM} $
& Advection matrix
& (\ref{eq:wkhTmass})
\\
%-------------------------------------------------------------------------
$ \intD (\nabla \Sh_1)^{\mathrm{T}} \mathcal{M} \nabla \Sh_1 \dDom $ %\label{eq:LapM1} $
& Laplace matrix
& (\ref{eq:wkmass}),(\ref{eq:wkhTmass})
\\
%-------------------------------------------------------------------------
$ \intD \mathbf B^{\mathrm{T}} \mathcal{M}  \mathbf B \dDom $ %\label{eq:Tang} $
& Tangential matrix
& (\ref{eq:wkstress})
\\
%-------------------------------------------------------------------------
$ \intD \mathcal{M} {\mathbf B}^{\mathrm{T}} \mathbf{m} \Sh_1 \dDom $ %\label{eq:CDMatrix} $
& Displacement coupling matrix
& (\ref{eq:wkmass})
\\
%-------------------------------------------------------------------------
$ \intD \mathcal{M}\Sh_1^{\mathrm{T}} \mathbf{m}^{\mathrm{T}} \mathbf B\dDom $ %\label{eq:CMatrix} $
& Pressure coupling matrix
& (\ref{eq:wkstress})
\\
%-------------------------------------------------------------------------
$ \intD Q \Sh_1 \dDom \quad \mbox{,} \,\intD Q \Sh_2 \dDom $ %\label{eq:SourceV} $
& Source term vector
& (\ref{eq:wkmass}),(\ref{eq:wkhTmass}),(\ref{eq:wkstress})
\\
%-------------------------------------------------------------------------
$ \intB q \Sh^{\scriptscriptstyle \Gamma} \dBdry $ %\label{eq:NeuV} $
& Neumann vector
& (\ref{eq:wkmass}),(\ref{eq:wkhTmass}),(\ref{eq:wkstress})
\\
%-------------------------------------------------------------------------
\hline
\end{tabular}
\caption{Matrix and vector types} \label{tab:types}
\end{table}

where $\mathcal{M}$ are a process-specific material functions,
$\mathbf B= \mathcal L \Sh_2$ is so called strain-displacement
matrix, $\mathbf m = (1,\,1,\,1,\,0,\,0,\,0)^T$ is mapping vector. Based on
this classification of matrix and vector types the finite element
object is designed (section \ref{sec:ele_fem}).
