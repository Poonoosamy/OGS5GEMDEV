%=========================================================================
\subsection{Process related finite element objects -- ELE-PCS}
\label{sec:ele_pcs}

Only at this stage (last part of element object concept) we
introduce process-related data. The element object
\texttt{CFiniteElementPCS} should work for all processes: fluid
flow, heat transport, deformation and reaction processes
regardless of PDE type and type of unknown field functions (scalar
or vector quantities).

The finite element object ELE-PCS has two tasks: First, calculation 
of element matrices, which are formed by shape functions 
($\Sh_1,\nabla\Sh_1$) and process-specific material properties 
(\texttt{MAT} objects) (Step 2 in Table \ref{tab:alg1}). Second, 
provide local element contributions to the global equation system: 
$A_{ij} x_i = b_j$, where $i,j$ are global node indices (Step 3 in 
Table \ref{tab:alg1}).

The C++ implementation of the process-related finite element
object ELE-PCS is given in Fig. \ref{fig:fem_one}.

\begin{itemize}
%.........................................................................
 \item ELE-FEM relation:
Process related instances are derived from the finite element base
class \texttt{CFiniteElement}. Therefore, they inherit all
necessary geometric and topological data from ELE-GEO, ELE-FEM,
and ELE-MSH objects.
%.........................................................................
 \item PCS relation:
Process related finite element objects need a reference to the
related PCS instance, which is conducted by the ELE-PCS class
constructor.
%.........................................................................
 \item MAT relations:
References to all \texttt{MAT} objects, i.e.
\texttt{CFluidProperties* m\_mfp}, \texttt{CSolidProper\-ties*
m\_msp} and \texttt{CMediumProperties* m\_mmp}, are used to get
the required material parameters of the specified
process(\texttt{CProcess* m\_pcs}). Member function
\texttt{SetMaterial()} prepares the references to process-specific
material properties to accelerate later computations. This insures
that the ELE-FEM objects works properly for all THM processes,
i.e. fluid flow, heat transport and deformation.
%.........................................................................
 \item Local assembly - Element matrices:
Based on geometric and finite element base data (ELE-FEM relation)
and the references to material data (PCS-MAT relation) the
process-specific element matrices can be calculated now
(\texttt{CalcXXXXMatrix()}). Member functions are used to
calculate the material coefficients in the Gauss integration
points (\texttt{CalcXXXXMatrixCoefficients()}). They are defined
as \texttt{inline} types to improve the computation efficiency.
Local element matrices and vectors are stored in the corresponding
symmetric/unsymmetric matrix and vector constructs.
%.........................................................................
 \item Global assembly - Equation system:
The global assembly is conducted by the \texttt{Assemble()}
function. It updates the individual element contributions in the
equation system, i.e. global left-hand-side (LHS) matrix
($A_{ij}$) and global right-hand-side (RHS) vector ($b_j$). To
this purpose the assembly functions needs the relations between
local element node and global mesh node numbers, which is provided
by the ELE-MSH topology (section \ref{sec:ele_msh}).
\texttt{Assemble()} functions are available for different PDE
types. How the \texttt{Assemble()} is implemented for a parabolic
PDE is shown in Fig. \ref{fig:fem_asm}.
\end{itemize}

\begin{figure}[htb!]
\centering
\shadowbox{
\begin{minipage}{\textwidth}
\footnotesize
\begin{verbatim}
class CFiniteElementPCS::public CFiniteElement {
  private: // Member data
    // PCS relation
    CProcess* m_pcs;
    // MAT relations
    CFluidProperties* m_mfp;
    CSolidProperties* m_mfp;
    CMediumProperties* m_mmp;
    // Element matrices
    SymMatrix* MassMatrix;
    SymMatrix* LaplaceMatrix;
    SymMatrix* PressureCouplingMatrix;
    Matrix*    AdvectionMatrix;
    Matrix*    StrainMatrix;
    Matrix*    StrainCouplingMatrix;
    ...
    Matrix*    LHSMatrix;
    Vec*       RHSVector;
  public:  // Member functions
    // Construction
    CFiniteElementPCS(CProcess* m_pcs);
    ~CFiniteElementPCS();
    // MAT functions
    void SetMaterial();
    inline void CalcMassMatrixCoefficient();
    inline void CalcAdvectionMatrixCoefficient();
    inline void CalcLaplaceMatrixCoefficient();
    inline void CalcStrainMatrixCoefficient();
    inline void CalcStrainCouplingMatrixCoefficient();
    inline void CalcPressureCouplingMatrixCoefficient();
    // Element matrices
    inline double InterpolateGPValues(double*); // Interpolation at Gauss points
    void SetMemory();
    void CalcMassMatrix();
    void CalcLumpedMassMatrix();
    void CalcAdvectionMatrix();
    void CalcLaplaceMatrix();
    void CalcStrainMatrix();
    void CalcStrainCouplingMatrix();
    void CalcPressureCouplingMatrix();
    void CalcGravityVector();
    void LocalAssembly(); // LHS element contribution
    ...
    // Element contribution to global equation system
    void GlobalAssembly(); // LHS matrix contribution
}
\end{verbatim}
\normalfont\normalsize

\end{minipage}
}
\caption{Process related finite element class}
\label{fig:fem_one}
\end{figure}


\begin{figure}[H]
\centering
\shadowbox{
\begin{minipage}{0.7\textwidth}
\small
\begin{verbatim}
void CFiniteElementPCS::AssembleParabolicPDEType()
{
  // MAT relations
  SetMaterial();
  // Calculation and assembly of element matrices
  CalcMassMatrix();
  CalcLaplaceMatrix();
  CalcStrainCouplingMatrix();
  // Calculation and assembly of RHS vector
  CalcRHSVector();
}
\end{verbatim}
\normalfont\normalsize

\end{minipage}
}
\caption{Linear element assemble function}
\label{fig:fem_asm}
\end{figure}

%-------------------------------------------------------------------------
%\subsubsection{Finite element objects for vector field functions}

%Deformation process is governed by the Navier's equation. The
%local element matrices like tangential matrix (\ref{eq:Tang}) has
%different expression in addition to those arise from the Laplace
%type equations. Moreover, the non-linear analysis of such problems
%involving complex local calculation. To this purpose, a
%corresponding finite element object is derived from root finite
%element object.
%The C++ implementation of the quadratic FEOs for
%deformation problems is given in Fig. \ref{fig:fem_dm}.

%\input{code/fem_dm.tex}

%In addition to the standard finite element object defined in section
%\ref{sec:finone}, this class needs a reference to solid material
%properties.

%The solid phase material object (\texttt{MSP}) provides the
%constitutive stress-strain relationships for linear as well as
%nonlinear behavior. While fluid material object allows deformation
%finite element object incorporate the coupling calculation.

%Element stiffness matrix, eqn. (\ref{eq:Tang}), pressure coupling
%matrix, eqn. (\ref{eq:CMatrix}), and RHS vector are calculated by
%member function \texttt{LocalAssembly} and assembled to the global
%equation system by member function \texttt{GlobalAssembly}.
%Modified local assemble functions are required for enhanced strain
%element calculations.
