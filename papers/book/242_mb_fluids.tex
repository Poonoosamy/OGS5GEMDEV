Dealing with flow in porous and fractured media we have to
consider the mechanics of fluids in tubes and channels. The
governing equation for flow of incompressible viscous fluids is
the well-known Navier-Stokes equation\index{equation -
Navier-Stokes}
\begin{eqnarray}
\underbrace{
\frac{\partial{\mathbf{v}}}{\partial t}
+
({\mathbf{v}}\cdot\nabla){\mathbf{v}}
}_{\mbox{\small inertial terms}}
=
\mathbf{g}
-
\frac 1\rho \nabla p
+
\underbrace{
\frac{\mu}{\rho} \nabla^2 {\mathbf{v}}
}_{\mbox{\small viscous term}}
\label{eqn:navier_stokes_equations}
\end{eqnarray}

The Navier-Stokes equation can be integrated for laminar flow in
straight, circular tubes when steady-state boundary conditions are
applied.\index{flow - laminar}\index{flow - tube} In this case
convective acceleration term
$({\mathbf{v}}\cdot\nabla){\mathbf{v}}$ becomes zero.
This solution is called the Hagen-Poiseuille equation \cite{Lamb:1932}.
\index{equation - Hagen-Poiseuille}
\begin{eqnarray}
\Delta p
=
\frac{8\mu L}{\pi R^4} \, Q
\label{eqn:hagen_poiseuille}
\end{eqnarray}
where $Q$ is the volumetric flow rate, which is velocity
multiplied by tube cross-section area, $L$ is tube length and $R$
is tube radius. The linear relationship between pressure drop and
flow rate $Q$ breaks down if convective acceleration and/or
transient effects become important. The first case is denoted
\textbf{non-linear laminar flow},\index{flow - non-linear laminar}
when inertial effects become important, e.g. due to curvature of
tubes or channels. The second case is related to \textbf{turbulent
flow},\index{flow - turbulent} when flow pattern become transient
due to velocity fluctuations.

Confusion between
non-linear laminar flow and 'true' turbulent flow
may arise from the fact
that -
concerning the relationship between pressure drop and the flow rate -
inertia effects in laminar flow are expressed in the same fashion
as in turbulent flows
\begin{eqnarray}
\Delta p
=
A Q + B Q^2
\, .
\label{eqn:forchheimer}
\end{eqnarray}
This means if inertia effects or turbulence effects become
significant, the relationship between pressure drop and flow rate
(\ref{eqn:hagen_poiseuille}) is no longer linear. Therefore, we
have to distinguish between three different flow regimes: linear
laminar flow, non-linear laminar flow and 'true' turbulent flow.
Equation (\ref{eqn:forchheimer}) is known as the Forchheimer
equation \cite{For:14}.\index{equation - Forchheimer}


\cite{Dar:56} found that the volume of fluid percolating through a
sand column is proportional to the applied pressure
difference\index{law - Darcy}
\begin{eqnarray}
Q
=
q A
=
A
\frac{k}{\mu}
\frac{\Delta p}{L}
\quad \rightarrow \quad
\Delta p
=
\frac{L}{A}
\frac{\mu}{k}
Q
\, .
\label{eqn:darcy_1}
\end{eqnarray}

Comparing the structure of equations
(\ref{eqn:darcy_1}) and  (\ref{eqn:hagen_poiseuille}),
the analogy between porous media flow and tube flow becomes obvious.
Both equations are characterized by linear relationships
between pressure drop and flow rate.

Darcy's law can be derived from the Navier-Stokes equations. To
this purpose a spatial averaging procedure over a representative
elementary volume (REV) has to be conducted, where microscopic
quantities are transformed into macroscopic ones \cite{Die:85}\index{volume - elementary representative}
\begin{eqnarray}
\langle \psi \rangle
=
\frac{1}{\mbox{\small REV}}
\int_{\mbox{\tiny REV}}
\psi
\, dV
\end{eqnarray}
where
$\psi$ is a local, microscopic quantity
and
$\langle \psi \rangle$ is a spatially averaged macroscopic quantity.
\footnote{See also equation (\ref{eqn:mean}) for general definition of a mean value for a porous medium.
Both notations $\overline \psi$ and $\langle \psi \rangle$ are common in literature.}
%
For fractures the averaging procedure can be splitted into two steps
\begin{eqnarray}
\langle \psi \rangle
=
\frac{1}{2b \, \mbox{\small REA}}
\int_{-b}^{+b}
\int_{\mbox{\tiny REA}}
\psi
\, dx \, dA
\label{eqn:averaging_fracs}
\end{eqnarray}
where $b$ is half fracture aperture and REA is a representative
elementary area. In the following we deal with quantities which
are averaged over fracture thickness and, therefore, are
representative for a certain area of fracture surface.

Darcy's law
is based essentially on the assumption that fluid motion is inertialess,
i.e. inertial terms can be neglected with regard to viscous forces
\begin{eqnarray}
0
=
\langle\mathbf{g}\rangle
-
\frac {1}{\langle\rho\rangle}
\nabla \langle p \rangle
+
\frac{\langle\mu\rangle}{\langle\rho\rangle}
\nabla^2 {\langle\mathbf{v}\rangle}
\, .
\label{eqn:darcy_2}
\end{eqnarray}

Brackets indicate macroscopic quantities. Thus Darcian flow is a
special case of creeping flow for which viscous forces prevail
over inertial forces. A central topic in porous medium theory is
the determination of the viscous drag term. This leads to the
concept of permeability for characterization of the
hydromechanical properties of porous media \cite{Scheidegger:1974}.
Introducing permeability in the following manner\index{material -
permeability}
\begin{eqnarray}
\nabla^2 {\langle\mathbf{v}\rangle}
=
-
{\mathbf k}^{-1}
\, {\mathbf w}
\end{eqnarray}
where
$k$ is the permeability of the porous medium and
% $\DarcyVelocityVector$ is the Darcy or seepage velocity,
$\mathbf{v}$ is the Darcy or seepage velocity,
which are macroscopic quantities by definition.
Substituting this expression into equation (\ref{eqn:darcy_2})
we obtain
\begin{eqnarray}
0
=
{\bf g}
-
\frac {1}{\langle\rho\rangle}
\nabla \langle p \rangle
-
\frac{\langle\mu\rangle}{\langle\rho\rangle}
{\mathbf k}^{-1} {\mathbf w}
\, .
\end{eqnarray}

Rearranging the terms,
we yield the usual form of Darcy's law.
We omit the averaging brackets in the following to keep the notation briefly
\begin{eqnarray}
\mio{w}{}{}
=
-
\frac{\mathbf k}{\mu}
(
\nabla p
-
\rho{\bf g}
)
\, .
\end{eqnarray}

We emphasize
that quantities in the above Darcy equation are macroscopic ones
related to a certain REV of a porous medium,
whereas quantities in the Navier-Stokes equation
(\ref{eqn:navier_stokes_equations}) have local meaning.

Darcy's law has been accepted
as fundamental relationship for porous medium hydraulics.
However, its validity
is restricted to a certain range of geometric and physical conditions.
Deviations from linearity between seepage velocity and pressure drop
are denoted as non-Darcian flow.
Geometric issues are concerned with pore and fracture geometry.

As described above we used the analogy to flow in straight tubes
for explanation of hydromechanical processes in porous media. Tube
bundles model is one approach to hydromechanics of porous media.
However in real geologic materials pores are curved, have varying
cross-sections, may be sealed, and suffer from dead-end effects.
Rock fractures are characterized by rough surfaces. Physical
causes underlying non-linear effects can be high flow rates,
molecular effects, ionic effects and non-Newtonian behavior of the
fluid itself \cite{Dullien:1979}.
